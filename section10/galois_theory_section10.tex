\documentclass[../master_galois_theory]{subfiles}
\begin{document}

\setcounter{section}{9}

\section{ノルムとトレース}

\begin{defi}
  $A:$有限次 $K-alg$とする。
  $x \in A$に対して $x$倍写像
  \begin{eqnarray*}
    T_x : A & \longrightarrow & A \\
    a & \longmapsto & xa
  \end{eqnarray*}
  は $A$が $K-alg$より $K-$線形写像になる。
  よってある $A$の基底によって $\dim_K(A) = n$のとき
  行列 $T_x : K^n \longrightarrow K^n$にできる。

  この行列 $T_x$について
  $x$の\underline{トレース $(\mathrm{trace})$} $\Tr_{A/K}(x)$と
  $x$の\underline{ノルム $(\mathrm{norm})$} $\N_{A/K}(x)$を
  \begin{eqnarray*}
    \Tr_{A/K}(x) & := & \Tr(T_x) \\
    \N_{A/K}(x) & := & \det(T_x)
  \end{eqnarray*}
  とするとこの値は $K$の元であるから
  \begin{eqnarray*}
    \Tr_{A/K} : A & \longrightarrow & K \\
    \N_{A/K} : A & \longrightarrow & K
  \end{eqnarray*}
  という写像になっていて
  $\Tr_{A/K}$は $K-$線形写像、 $\N_{A/K}$は乗法的 $(\N(xy) = \N(x) \N(y))$である。
  とくに、定義域を乗法群 $A^\times$に制限すれば
  \begin{eqnarray*}
    \N_{A/K}|_{A^\times} : A^\times \longrightarrow K
  \end{eqnarray*}
  は群準同型になる。
\end{defi}

\begin{exam}
  $x \in K$のとき $n := [A:K]$として、
  $A$の基底を $\{ e_1 , \dots , e_n \}$とする。
  $T_x = (t_{ij})_{i,j = 1 , \dots , n}$とおいたとき
  行列表示は
  \begin{eqnarray*}
    T_x (e_j) = \sum_{i=1}^n t_{ij} e_i
  \end{eqnarray*}
  とできて $T_x (e_j) = xe_j$で基底の一次独立性から
  $t_{jj} = x , t_{ij} = 0 \  (i \neq j)$となるので
  \begin{eqnarray*}
    T_x =
    \begin{pmatrix}
      x &  &   \\
        & \ddots &  \\
        &  &  x
    \end{pmatrix}
  \end{eqnarray*}
  と書ける。
  したがって
  $\Tr_{A/K}(x) = nx , \N_{A/K}(x) = x^n$
  となる。
\end{exam}

\end{document}
