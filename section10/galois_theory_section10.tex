\documentclass[../master_galois_theory]{subfiles}
\begin{document}

\setcounter{section}{9}

\section{ノルムとトレース}

\subsection{ノルムとトレース}

\begin{defi}
  $A:$有限次 $K-alg$とする。
  $x \in A$に対して $x$倍写像
  \begin{eqnarray*}
    T_x : A & \longrightarrow & A \\
    a & \longmapsto & xa
  \end{eqnarray*}
  は $A$が $K-alg$より $K-$線形写像になる。
  よって $\dim_K(A) = n$のときある $A$の基底 $\{ e_1 , \dots , e_n \}$により、 $T_x = (t_{ij})_{i,j = 1 , \dots , n}$とおいたとき
  行列表示は
  \begin{eqnarray*}
    T_x (e_j) = x e_j = \sum_{i=1}^n t_{ij} e_i
  \end{eqnarray*}
  を満たすような $t_{ij} \in K$で作られてこれにより
  行列 $T_x : K^n \longrightarrow K^n$にできて行列の記法で
  \begin{eqnarray*}
    x(e_1 , \cdots , e_n) = (e_1 , \cdots , e_n) T_x
  \end{eqnarray*}
  と書くことができる。

  この行列 $T_x$について
  $x$の\underline{トレース $(\mathrm{trace})$} $\Tr_{A/K}(x)$と
  $x$の\underline{ノルム $(\mathrm{norm})$} $\N_{A/K}(x)$を
  \begin{eqnarray*}
    \Tr_{A/K}(x) & := & \Tr(T_x) \\
    \N_{A/K}(x) & := & \det(T_x)
  \end{eqnarray*}
  とするとこの値は $K$の元であるから
  \begin{eqnarray*}
    \Tr_{A/K} : A & \longrightarrow & K \\
    \N_{A/K} : A & \longrightarrow & K
  \end{eqnarray*}
  という写像になっていて
  $\Tr_{A/K}$は $K-$線形写像、 $\N_{A/K}$は乗法的 $(\N(xy) = \N(x) \N(y))$である。
  とくに、定義域を乗法群 $A^\times$に制限すれば
  \begin{eqnarray*}
    \N_{A/K}|_{A^\times} : A^\times \longrightarrow K
  \end{eqnarray*}
  は群準同型になる。
\end{defi}

\begin{exam}
  $x \in K$のとき $n := [A:K]$として、
  $A$の基底を $\{ e_1 , \dots , e_n \}$とする。
  $T_x = (t_{ij})_{i,j = 1 , \dots , n}$とおいたとき
  行列表示は
  \begin{eqnarray*}
    T_x (e_j) = \sum_{i=1}^n t_{ij} e_i
  \end{eqnarray*}
  とできて $T_x (e_j) = xe_j$で基底の一次独立性から
  $t_{jj} = x , t_{ij} = 0 \  (i \neq j)$となるので
  \begin{eqnarray*}
    T_x =
    \begin{pmatrix}
      x &        &   \\
        & \ddots &  \\
        &        &  x
    \end{pmatrix}
  \end{eqnarray*}
  と書ける。
  したがって
  $\Tr_{A/K}(x) = nx , \N_{A/K}(x) = x^n$
  となる。
\end{exam}

\begin{exam}
  $A := K[X]/(f)$で $f = X^n + a_1 X^{n-1} + \cdots + a_n \in K[X]$とする。
  $x := X + (f) \in A$についてその $x$倍写像 $T_x$は
  \begin{eqnarray*}
    T_x =
    \begin{pmatrix}
      0 &        &   & - a_n \\
      1 & \ddots &   & \vdots \\
        & \ddots & 0 & \vdots \\
        &        & 1 & - a_1
    \end{pmatrix}
  \end{eqnarray*}
  と書けるから
  $\Tr_{A/K}(x) = -a_1 , \N_{A/K}(x) = (-1)^n a_n$となる。
\end{exam}

\begin{proof}
  $x \in A$はその定義から $f$の根になっている。
  命題 $(\mathrm{\ref{prop:6.1}})$の $(2)$より
  $\{ 1 , x , \dots , x^{n-1} \}$は $A$の基底になっているので
  この基底を用いて $T_x$を行列表示にする。
  $T_x := (t_{ij})_{i,j = 1 , \dots , n}$は $x$の指数を考えれば
  \begin{eqnarray*}
    T_x(x^j) = \sum_{i=0}^{n-1} t_{i+1 j+1} x^i \  (0 \leq j \leq n-1)
  \end{eqnarray*}
  とできる。
  $T_x(x^j) = x^{j+1} \  (0 \leq j \leq n-1)$より $1 \leq j+1 \leq n-1$のとき
  \begin{eqnarray*}
    t_{i+1 j+1} =
    \begin{cases}
      1 & (j+1 = i) \\
      0 & (j+1 \neq i)
    \end{cases}
  \end{eqnarray*}
  $j+1 = n$のとき $x \cdot x^{n-1} = x^n = X^n + (f) = -a_1 X^{n-1} - \cdots -a_n + (f) = -a_1 x^{n-1} - \cdots -a_n$であるので
  \begin{eqnarray*}
    T_x(x^{n-1}) = x^n & = & -a_1 x^{n-1} -a_2 x^{n-2} - \cdots -a_n \\
    = \sum_{i=0}^{n-1} t_{i+1 n} x^i & = & t_{n n} x^{n-1} + t_{n-1 n} x^{n-2} + \cdots + t_{1 n}
  \end{eqnarray*}
  より $t_{n-k n} = -a_{k+1}$となる。
  よって $T_x$は上記の形になる。

  $\Tr_{A/K}(x) = \Tr(T_x) = -a_1$は明らか。
  $\N_{A/K}(x) = \det(T_x)$は $n$列をとなりの列と順番に入れ替えていけば
  入れ替えるごとに $-1$倍されて $1$列まで移動させれば
  行列式の性質より $\det(T_x) = (-1)^{n-1} (-a_n) \det(E_n) = (-1)^n a_n$となる。
\end{proof}

\subsection{正則表現}

\begin{prop}
  体拡大 $L/K$について
  $x$倍写像を作る対応 $T$を $L$の $K$上の基底 $\{ e_1 , \dots , e_n \}$によって $T_x \in M_n(K)$で考えると
  \begin{eqnarray*}
    T : L & \longrightarrow & M_n(K) \\
    x & \longmapsto & T_x
  \end{eqnarray*}
  は $T_x$の成分の定まり方より写像であり、単射環準同型になる。
  この $K$上の写像 $T$を基底 $\{ e_1 , \dots , e_n \}$に関する
  $A/K$の\underline{正則表現}という。
\end{prop}

\begin{proof}
  $T_x , T_y , T_{x + y} , T_{cx} , T_{xy} \in M_n(K) \  (x , y \in A \  c \in K)$についてこれはそれぞれ
  \begin{eqnarray*}
    x(e_1 , \cdots , e_n) & = & (e_1 , \cdots , e_n) T_x \\
    y(e_1 , \cdots , e_n) & = & (e_1 , \cdots , e_n) T_y \\
    (x + y)(e_1 , \cdots , e_n) & = & (e_1 , \cdots , e_n) T_{x + y} \\
    cx(e_1 , \cdots , e_n) & = & (e_1 , \cdots , e_n) T_{cx} \\
    xy(e_1 , \cdots , e_n) & = & (e_1 , \cdots , e_n) T_{xy}
  \end{eqnarray*}
  を満たしている。
  それぞれ演算結果が等しくなることを考えれば
  \begin{eqnarray*}
    T_{x + y} & = & T_x + T_y \\
    T_{cx} & = & c T_x \\
    T_{xy} & = & T_x T_y
  \end{eqnarray*}
  を満たすので $T : L \longrightarrow M_n(K)$は環準同型である。

  また、 $e_j$が基底なので
  $T(x) = T_x = 0 \Leftrightarrow t_ij = 0 ({}^\forall i , j) \Leftrightarrow x e_j = 0 ({}^\forall j) \Leftrightarrow x = 0$
  が成り立つから $\ker(T) = \{ 0 \}$より $T$は単射。
\end{proof}

\begin{prop} \label{prop:2.60}
  $L/K:n$次分離拡大、 $\Omega:K$の代数閉包、 $\sigma_i \in \Hom_{K}(L,\Omega) , (1 \leq i \leq n = [L:K] = [L:K]_s (分離拡大より))$とする。
  このとき $L$の $n$個の元 $e_1 , \dots , e_n$について次は同値。

  $(1)$
  $e_1 , \dots , e_n$は $L/K$の基底。

  $(2)$
  \begin{eqnarray*}
    \det(\sigma_i(e_j)) =
    \begin{vmatrix}
      \sigma_1(e_1) & \cdots & \sigma_1(e_n) \\
      \sigma_2(e_1) & \cdots & \sigma_2(e_n) \\
      \vdots        &        & \vdots        \\
      \sigma_n(e_1) & \cdots & \sigma_n(e_n)
    \end{vmatrix}
    \neq 0
  \end{eqnarray*}
\end{prop}

\begin{proof}
  $(1) \Rightarrow (2)$

  $\det(\sigma_i(e_j)) = 0$と仮定すると $X = (\sigma_i(e_j))$とおいたとき $\vec{x} X = 0$は非自明解 $(c_1 , \dots , c_n) \in \Omega^n$をもつ。
  つまり $\sum_{i=1}^n c_i \sigma_i(e_j) = 0 \  (1 \leq j \leq n)$となるものが存在している。
  このとき任意の元 $\alpha \in L$に対して、
  基底であることより $\alpha = \sum_{i=1}^n a_i e_i$となる
  $a_i \in K$が存在する。
  このとき $\sigma_i$は $K$を動かさないので
  \begin{eqnarray*}
    \sum_{i=1}^n c_i \sigma_i(\alpha) & = & \sum_{i=1}^n c_i \sigma_i \left( \sum_{i=1}^n a_i e_i \right) \\
    & = & \sum_{i=1}^n c_i \sum_{j=1}^n a_j \sigma_i (e_j) \\
    & = & \sum_{j=1}^n a_j \sum_{i=1}^n c_i \sigma_i (e_j) \\
    & = & \sum_{j=1}^n a_j \cdot 0 \\
    & = & 0
 \end{eqnarray*}
 となるが $c_i$は全ては $0$で無いので\rm{Dedekind}の補題
 $(\mathrm{\ref{lemm:2.3}})$に矛盾する。
 よって $\det(\sigma_i(e_j)) \neq 0$

 $(2) \Rightarrow (1)$

 $(2)$を満たすような $e_1 , \dots , e_n$が一次独立であることを示す。
 $c_1 e_1 + \cdots + c_n e_n = 0$となる $c_i \in K$をとる。
 全体に $\sigma_j$をかけると
 $\sum_{i=1}^n c_i \sigma_j (e_i) = 0$であるから
 \begin{eqnarray*}
   \begin{pmatrix}
     \sigma_1 (e_1) & \cdots & \sigma_1 (e_n) \\
     \vdots         &        & \vdots         \\
     \sigma_n (e_1) & \cdots & \sigma_n (e_n)
   \end{pmatrix}
   \begin{pmatrix}
     c_1 \\
     \vdots \\
     c_n
   \end{pmatrix}
   = 0
 \end{eqnarray*}
 となる。
 ここで仮定より $\det(\sigma_i(e_j)) \neq 0$なのでこの連立方程式は
 自明解のみをもつから $c_1 = \cdots = c_n = 0$であるので
 $e_1 , \cdots , e_n$は一次独立。
 $L/K$は $n$次拡大なので基底の個数は $n$個だからこの $e_1 , \cdots , e_n$
 が基底になる。
\end{proof}

\begin{prop} \label{prop:2.61}
  $L/K:n$次分離拡大、 $\Omega$を $K$の代数閉包、
  $\sigma_i : L \longrightarrow \Omega , \alpha \longmapsto \alpha^{\sigma_i} (= \alpha^{(i)}) := \sigma_i(\alpha) , \sigma_i \in \Hom_K(L,\Omega)$
  としたとき $\alpha \in L$について
  \begin{eqnarray*}
    \Tr_{L/K}(\alpha) & = & \sum_{i=1}^n \alpha^{(i)} = \sum_{i=1}^n \alpha^{\sigma_i} \\
    \N_{L/K}(\alpha) & = & \prod_{i=1}^n \alpha^{(i)} = \prod_{i=1}^n \alpha^{\sigma_i}
  \end{eqnarray*}
  となる。
\end{prop}

\begin{proof}
  $L/K$の基底を $e_1 , \dots , e_n$とする。
  任意の $\alpha \in L$についてこの基底による正則表現
  $T : L \longrightarrow M_n(K) , \alpha \longmapsto T_\alpha$は
  $\alpha (e_1 , \cdots , e_n) = (e_1 , \cdots , e_n) T_\alpha$を満たす。
  これに $\sigma_i$をかけると $\sigma_i(T_\alpha) = T_\alpha$であり、
  $\alpha^{(i)} (e_1^{(i)} , \cdots , e_n^{(i)}) = (e_1^{(i)} , \cdots , e_n^{(i)}) T_\alpha$となる。
  これは
  \begin{eqnarray*}
    T_\alpha^\circ :=
    \begin{pmatrix}
      \alpha^{(1)} &        & \\
                   & \ddots & \\
                   &        & \alpha^{(n)}
    \end{pmatrix}
    \\
  \end{eqnarray*}
  と $M := (e_j^{(i)})_{i , j = 1 , \dots , n}$によって
  $T_\alpha^\circ M = M T_\alpha$となる。
  命題 $(\mathrm{\ref{prop:2.60}})$の $(1) \Rightarrow (2)$より
  $\det(M) \neq 0$なので正則行列より $M^{-1}$が存在するから
  $T_\alpha = M^{-1} T_\alpha^\circ M$とできる。
  したがって $\Tr$と $\det$の性質から
  \begin{eqnarray*}
    \Tr_{L/K}(\alpha) & = & \Tr(T_\alpha) = \Tr(M^{-1} T_\alpha^\circ M) = \Tr(T_\alpha^\circ) = \sum_{i=1}^n \alpha^{(i)} = \sum_{i=1}^n \alpha^{\sigma_i} \\
    \N_{L/K}(\alpha) & = & \det(T_\alpha) = \det(M^{-1} T_\alpha^\circ M) = \det(T_\alpha^\circ) = \prod_{i=1}^n \alpha^{(i)} = \prod_{i=1}^n \alpha^{\alpha_i}
  \end{eqnarray*}
  が成り立つ。
\end{proof}

\begin{corl}
  $L/K$が有限次分離拡大なら $\Tr_{L/K}(\alpha) \neq 0$となる $\alpha \in L$が存在する。
\end{corl}

\begin{proof}
  任意の $\alpha \in L$について命題 $(\mathrm{\ref{prop:2.61}})$から
  $\Tr_{L/K}(\alpha) = \sum_{i=1}^n \alpha^{(i)}$であり、
  これが $0$に等しいとすると
  命題 $(\mathrm{\ref{lemm2.3}})$に矛盾するから
  ある $\alpha \in L$で $\Tr_{L/K}(\alpha) \neq 0$となる。
\end{proof}

\end{document}
