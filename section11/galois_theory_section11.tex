\documentclass[../master_galois_theory]{subfiles}
\begin{document}

\setcounter{section}{10}

\section{正規拡大 \  (準Galois拡大)}

\subsection{共役}

\begin{defi}
  $\Omega := \barK : K$の代数閉包とする。
  $L/K , M/K (L , M \subset \Omega)$が $K$上\underline{共役 $(\mathrm{conjugate})$}とはある $\sigma \in \aut_K(\Omega)$
  があって $\sigma (L) = M$となること。

  $x , y \in \Omega$が $K$上\underline{共役 $(\mathrm{conjugate})$}とは
  ある $\sigma \in \aut_K(\Omega)$があって $\sigma(x) = y$となること。
\end{defi}

\begin{exam}
  $z , \overline{z} \in \mathbb{C}$は $\mathbb{R}$の代数閉包であり、
  $G = \aut_{\mathbb{R}}(\mathbb{C}) := \{ \mathrm{Id}_{\mathbb{R}} , \sigma \} , \sigma(z) = \overline{z}$とする。
  このとき $G$の固定体 $\mathbb{R}^G = \mathbb{R}$となるので $\mathbb{C}/\mathbb{R}$は\rm{Galois}である。
  この $\sigma$は複素共役をとる写像であるが $\sigma(z) = \overline{z}$より
  一般の共役の定義にも適している。
\end{exam}

\begin{prop} \label{prop:conjugate}
  $K$の代数閉包 $\Omega$とし、
  $x , y \in \Omega$をとる。
  このとき次は同値。

  $(1)$
  $x$と $y$は $K$上共役。

  $(2)$
  $K-$同型写像 $v : K(x) \longrightarrow K(y)$で
  $v(x) = y$となるものが存在する。

  $(3)$
  $x$と $y$の $K$上の最小多項式は同じ。
\end{prop}

\begin{proof}
  $(1) \Rightarrow (3)$

  $x$の最小多項式を $f \in K[X]$とする。
  $x$と $y$は共役なのである $\sigma \in \aut_K(\Omega)$が存在して
  $\sigma(x) = y$となる。
  $\sigma$は $K-$自己準同型より $K$の元を動かさないので $f$の係数を動かさない。
  よって $f(y) = f(\sigma(x)) = \sigma(f(x)) = \sigma(0) = 0$より
  $f$は $y$を根にもつ。
  $y$の最小多項式を $g \in K[X]$とする。
  $f \neq g$と仮定すると $\deg(g)$の最小性から $g|f$より
  $f = gh$となる $h \in K[X] , \deg(h) > 0$が存在する。
  このとき $f(x) = g(x)h(x) = 0$となり $f$より次数の低い $g$または $h$が $x$を
  根にもつ。
  これは $\deg(f)$の最小性に矛盾するから $f = g$より $x$と $y$の最小多項式は一致する。

  $(3) \Rightarrow (2)$

  $x$と $y$の最小多項式を $f \in K[X]$とする。
  このとき命題 $(\mathrm{\ref{prop:6,7}})$より
  $K(x) \cong K[X]/(f) \cong K(y)$であり、
  \begin{eqnarray*}
    K(x) \longrightarrow & K[X]/(f) & \longrightarrow K(y) \\
    x \longmapsto & X + (f) & \longmapsto y
  \end{eqnarray*}
  となる同型写像が作れる。
  したがって $v : K(x) \longrightarrow K(y) , x \longmapsto v(y)$
  となる $K-$同型写像が存在する。

  $(2) \Rightarrow (1)$

  $\Omega$は $K(x) , K(y)$の代数閉包でもあるので
  系 $(\mathrm{\ref{corl:7.6}})$から $K-$同型 $v : K(x) \longrightarrow K(y)$を
  $\tilde{v} : \Omega \longrightarrow \Omega$に延長できる。
  これは $K-$自己準同型なので $\tilde{v} \in \aut_K(\Omega)$で
  $\tilde{v}(x) = v(x) = y$より定義から $x$と $y$は $K$上共役。
\end{proof}

\begin{corl}
  $x \in K$の最小多項式 $f \in K[X]$で $\sigma \in \aut_K(\Omega)$とする。
  このとき
  \begin{eqnarray*}
    g(X) := \prod_{\sigma \in \aut_K(\Omega)} (X - \sigma(x))
  \end{eqnarray*}
  は $\Omega$において $f$を割る。
  \end{corl}

\begin{proof}
  命題 $(\mathrm{\ref{prop:conjugate}})$の $(1) \Leftrightarrow (3)$
  より $f$は $x$の共役元を根としてすべて含むので $\Omega$において
  $f$が一次因子の積に分解できることより $g$は $f$を割る。
\end{proof}

\subsection{正規}

\begin{defi} \label{defi:normal}
  代数拡大 $L/K$が\underline{正規 $(\mathrm{normal})$}もしくは
  \underline{準\rm{Galois} $(\mathrm{quasi-galois})$}であるとは
  任意の既約多項式 $f \in K[X]$が根を $L$内に一つもてば
  $d$は $L[X]$において一次因子の積に分解することができる。
  $(すべて同じ体の中に根をもつ)$

  $\Leftrightarrow$
  ${}^\forall x \in L$に対してその最小多項式 $f \in K[X]$は $L[X]$において
  一次因子の積に分解できる。

  とくに代数閉包 $\Omega/K$は代数閉体の同値条件の命題 $(\mathrm{\ref{prop:7.1}})$の $(\mathrm{AC}1)$から正規拡大である。
\end{defi}

\begin{prop} \label{prop:11.1}
  代数拡大 $L/K$と代数閉包 $\Omega/K$について次は同値。

  $(1)$
  $L/K$は正規。

  $(2)$
  ${}^\forall x \in L$に対してその任意の共役は $L$に含まれる。

  $(3)$
  ${}^\forall \sigma \in \aut_K(\Omega) , \sigma(L) = L$となる。

  $(4)$
  ${}^\forall \phi \in \Hom_K(L,\Omega) , \phi(L) = L$となる。

  $(5)$
  $L$はある $K$上の多項式族 $(f_i)_{i \in I}$の最小分解体。
\end{prop}

\begin{proof}
  $(1) \Rightarrow (2)$

  $x \in L$の最小多項式 $f \in K[X]$をとる。
  $L/K$が正規で $x$が $L$での $f$の根なので
  $f$は $L[X]$上で一次因子の積に分解できる。
  よって $f$の根はすべて $L$に含まれている。
  ここで命題 $(\mathrm{\ref{prop:conjugate}})$の $(1) \Leftrightarrow (3)$より
  $x$の任意の共役元も最小多項式は $f$なので $f$の根であるからそれは $L$に含まれる。

  $(2) \Rightarrow (1)$

  ${}^\forall x \in L$について $L/K$が代数拡大より最小多項式 $f \in K[X]$がある。
  $f$の他の根 $a_i \in \Omega/K , 1 \leq i \leq n := \deg(f)$も
  $f$を最小多項式として持っているから命題 $(\mathrm{\ref{prop:conjugate}})$
  の $(1) \Leftrightarrow (3)$より $a_i$は $x$の共役元である。
  したがって $a_i \in L$であるから $f$は $L[X]$で $f = \prod_{i=1}^n (X - a_i)$と一次因子の積に分解できるので $L/K$は正規拡大。

  $(1) \Rightarrow (5)$

  ${}^\forall x \in L$の $K$上の最小多項式の族 $(f_i)_{i \in I}$をとり、
  この最小分解体を $M$とする。
  このとき $M[X]$では $f_i$はすべて一次因子の積に分解できるから
  $M \subset L$であり、 $x \in M$でもあるので $M = L$より
  $L$は $(f_i)_{i \in I}$の最小分解体である。

  $(5) \Rightarrow (3)$

  $L$が $(f_i)_{i \in I}$の最小分解体であるとする。
  $f_i$の根を $\alpha_{ij} \in \Omega/K , 1 \leq j \leq n := \deg(f_i)$
  とする。
  この根の集合を $R_i$とおくとき最小分解体の定義から $L = K(\cup_{i \in I} R_i)$
  とかける。
  ${}^\forall \sigma \in \aut_K(\Omega)$をとったときこれは $K$を動かさない。
  また、 $\alpha_{ij}$の最小多項式はすべて $f_i$なのでそれぞれ共役であり
  体の準同型から単射なので $\sigma(R_i) = R_i$となる。
  したがって $\sigma(L) = \sigma(K(\cup_{i \in I}R_i)) = K(\cup_{i \in I}R_i) = L$より成立。

  $(3) \Rightarrow (2)$

  ${}^\forall x \in L$に対してその共役は任意の $\sigma \in \aut_K(\Omega)$による $\sigma(x)$であるが仮定より $\sigma(L) = L$より $\sigma(x) \in L$となる。
  したがって任意の元のすべての共役は $L$に含まれるので成立。

  $(4) \Rightarrow (3)$

  ${}^\forall \sigma \in \aut_K(\Omega)$をとる。
  このとき $\sigma|_L \in \Hom_K(L,\Omega)$なので仮定より
  $\sigma|_L(L) = L$で $\sigma|_L(L) = \sigma(L)$より成立。

  $(3) \Rightarrow (4)$

  $\phi \in \Hom_K(L,\Omega)$にたいして $L , \phi(L)$は $K$の代数拡大
  なので定理 $(\mathrm{\ref{theo:7.3}})$から
  代数閉包 $\Omega$に埋め込めるので $\Omega$はこれらの代数閉包でもある。
  $\phi$は体の準同型より単射なので $\phi : L \longrightarrow \phi(L)$は全単射
  となっているから $L \cong \phi(L)$になっていて系 $(\mathrm{\ref{corl:7.6}})$
  よりこれを延長する $\sigma : \Omega \longrightarrow \Omega$が存在する。
  したがって仮定より $\sigma(L) = L$であり、 $\sigma|_L = \phi$なので
  $\phi(L) = \sigma|_L(L) = \sigma(L) = L$より成立。
\end{proof}

\begin{corl} \label{corl:11.2}
  $L/K:$有限次拡大のとき
  \begin{eqnarray*}
    L/K:正規 \Leftrightarrow [L:K]_s = h_L(L) (:= |\Hom_K(L,L)|)
  \end{eqnarray*}
  が成り立つ。
\end{corl}

\begin{proof}
  系 $(\mathrm{\ref{corl:8.2}})$より $[L:K]_s \leq [L:K]$より
  $L/K$が有限次拡大なので $[L:K]_s$も有限。
  $L \subset \Omega$から一般に $\aut_K(L) \subset \Hom_K(L,\Omega)$である。
  体の準同型は単射なので $\Hom_K(L,L) = \aut_K(L)$とも書ける

  $(\Rightarrow)$

  命題 $(\mathrm{\ref{prop:11.1}})$の $(1) \Leftrightarrow (4)$から
  ${}^\forall \sigma \in \Hom_K(L,\Omega)$をとると $\sigma(L) = L$と
  なっているので $\sigma \in \aut_K(L)$である。
  よって $\Hom_K(L,\Omega) \subset \aut_K(L)$であるので、
  一般に $\aut_K(L) \subset \Hom_K(L,\Omega)$が成り立つことを考えれば
  $\aut_K(L) = \Hom_K(L,\Omega)$だから
  $h_L(L) = [L:K]_s$である。

  $(\Leftarrow)$

  $h_L(L) = [L:K]_s$が有限で成り立っていて $\aut_K(L) \subset \Hom_K(L,\Omega)$より $\aut_K(L) = \Hom_K(L,\Omega)$である。
  ${}^\forall \sigma \in \Hom_K(L,\Omega)$をとると
  $\sigma \in \aut_K(L)$なので $\sigma(L) = L$を満たすから
  命題 $(\mathrm{\ref{prop:11.1}})$の $(1) \Leftrightarrow (4)$から
  $L/K$は正規。
\end{proof}

\clearpage

\end{document}
