\documentclass[../master_galois_theory]{subfiles}
\begin{document}

\setcounter{section}{11}

\section{Galois拡大再論}

\subsection{Galois拡大}

\begin{prop} \label{prop:12.1}
  代数拡大 $L/K$について次は同値

  $(1)$
  $L/K$は\rm{Galois}

  $(2)$
  $L/K$は正規かつ分離的

  $(2)'$
  ${}^\forall x \in L$に対し、その最小多項式は分離的かつ
  $L[X]$において一次因子の積に分解される。

  $(3)$
  $L/K$はある分離多項式族 $(f_i)_{i \in I}$の最小分解体

  さらに、 $L/K$が有限次なら次も同値

  $(4)$
  $[L:K] = h_L(L) (:= |\aut_K(L)|)$
\end{prop}

\begin{proof}
  $\Omega$を $K$の代数閉包とする。

  $(2) \Leftrightarrow (2)'$は正規の定義 $(\mathrm{\ref{defi:normal}})$と系 $(\mathrm{\ref{corl:separable}})$
  と多項式の分離性の定義 $(\mathrm{\ref{prop:9.2}})$から明らか。

  $(1) \Rightarrow (2)$

  ${}^\forall x \in L$とその最小多項式 $f \in K[X]$をとる。
  また、 $Y_x := \{ \sigma(x) | \sigma \in \aut_K(L) \}$
  と定めるとこれは $x$の $\Omega$における共役元の集合の部分集合になり、
  $\sigma \in \aut_K(L)$から $Y \subset L$である。
  命題 $(\mathrm{\ref{prop:conjugate}})$の $(1) \Leftrightarrow (3)$から
  $x$の共役元はすべて $f$の根なので高々 $\deg(f)$個しかないので $Y_x$は有限集合。
  $g := \prod_{y \in Y_x}(X - y) , n := \deg(g)$とする。
  $y$はすべて異なるから単根なので $g$は分離的である。
  また、 $y$は $x$の共役元より $f$の根でもあるから $g$のすべての根は $f$の根より
  $g|f$となる。

  $y \in Y_x \subset L$よりその元から作られる基本対称式は $L$に含まれるので
  $g = \sum_{i=1}^n a_i X^i , a_i \in L$と書ける。
  $\sigma g = \sum_{i=1}^n \sigma(a_i) X^i$とすると係数だけに $\sigma$をかけているから $(\sigma g)(X) = \prod_{y \in Y_x}(X - \sigma(y))$となる。
  ここで $y \in Y_x$より $y = \tau(x) , \tau \in \aut_K(L)$となるものが存在する。
  $\aut_K(L)$は自己同型写像であるから $\sigma \circ \tau \in \aut_K(L)$より $\sigma(y) = \sigma \circ \tau(x)) \in Y_x$となる。
  ここで $Y_x$は有限集合であることと $\sigma$は体の準同型より単射なので
  それぞれの $y$は $\sigma$によりそれぞれ異なる $Y_x$の元に行く。
  したがって $(\sigma g)(X) = \prod_{y \in Y_x}(X - y) = g(X)$となるから
  $a_i$は ${}^\forall \sigma \in \aut_K(L)$によって動かされない。
  $L/K$が\rm{Galois}より $L^{\aut_K(L)} = K$より $a_i \in K$であるから
  $g \in K[X]$である。

  $g , f \in K[X]$で $g|f$より $f$の最小性から $f = g$なので
  任意の $x \in L$の最小多項式は
  $f = \prod_{y \in Y_x}(X - y)$と $L[X]$上で一次因子の積に分解されるので
  $L/K$は正規。
  また、 $g (= f)$は分離的でもあったので任意の最小多項式が分離的より
  系 $(\mathrm{\ref{corl:separable}})$より $L/K$は分離的であるので
  $L/K$は正規かつ分離的。

  $(2) \Rightarrow (1)$

  $L = K$のとき $L^{\aut_K(L)} = K^{\aut_K(K)} = K$で成立。
  $L \neq K$のとき $L \supsetneq K$であるから ${}^\forall x \in L - K$をとる。
  これがある $\sigma \in \aut_K(L)$で $\sigma(x) \neq x$となればよい。

  $x$の最小多項式を $f \in K[X]$とすると $x \in L - K$より $\deg(f) > 1$であり、
  仮定から $L/K$が分離的より系 $(\mathrm{\ref{corl:separable}})$から $f$が単根を持つので
  定義より分離的だから $f(y) = 0$で $y \neq x$であるような元 $y \in \Omega$が存在する。
  $y$の $K$上の最小多項式も $f$なので命題 $(\mathrm{\ref{prop:conjugate}})$の
  $(2) \Leftrightarrow (3)$から $\sigma(x) = y$となるような $\sigma \in \aut_K(\Omega)$が存在する。
  仮定から $L/K$は正規なので命題 $(\mathrm{\ref{prop:11.1}})$の $(1) \Leftrightarrow (3)$から $\sigma(L) = L$より
  $\sigma|_L \in \aut_K(L)$となる。
  この $\sigma$により $\sigma(x) = y \neq x$なので $x$は固定されないから
  固定されるのは $K$の元のみなので $L^{\aut_K(L)} = K$となり
  定義より $L/K$は\rm{Galois}である。

  $(2) \Leftrightarrow (3)$

  命題 $(\mathrm{\ref{prop:11.1}})$の $(1) \Leftrightarrow (5)$より
  「規 $\Leftrightarrow$ある多項式族 $(f_i)_{i \in I}$の最小分解体」が言えている。
  その多項式族は ${}^\forall x \in L$の最小多項式の族であったので
  系 $(\mathrm{\ref{corl:separable}})$より
  「分離的 $\Leftrightarrow$多項式族のすべての多項式が分離的」が言えている。

  $(2) \Leftrightarrow (4)$

  有限次拡大のとき系 $(\mathrm{\ref{corl:11.2}})$から
  「正規 $\Leftrightarrow [L:K]_s = h_L(L)$」が言えている。
  定義より「分離的 $\Leftrightarrow [L:K] = [L:K]_s$」なので
  「正規かつ分離的 $\Leftrightarrow [L:K] = [L:K]_s = h_L(L)$」となり示された。
\end{proof}

\subsection{多項式のGalois群}

\begin{defi} \label{defi:galoispolynomial}
  $K:$体、 $f \in K[X] - K:$分離多項式、 $L_f:f$の $K$上の最小分解体とするとき
  その根をすべて添加しているので命題 $(\mathrm{\ref{prop:6.7}})$から
  $L_f/K$は有限次だから
  命題 $(\mathrm{\ref{prop:12.1}})$の $(1) \Leftrightarrow (3)$から
  $L_f/K$は有限次\rm{Galois}拡大である。
  このとき $\gal(L_f/K)$を\underline{$f$の $K$上の\rm{Galois}群}という。
\end{defi}

\begin{prop} \label{prop:galoispolynomial}
  分離多項式 $f \in K[X] - K$にたいしてその最小分解体 $L_f$を考える。
  $\Omega$を $K$の代数閉包で $L_f$を含むもの、
  $W := \{ x \in \Omega | fの根 \}$とする。
  $f$は分離多項式なので $|W| = n := \deg(f)$となる。
  このとき $\gal(L_f/K)$は $W$に作用し、根の置換を引き起こす。
  したがって $W$の自己同型写像の群、つまり $W$の置換群を $\mathfrak{S}_W$とするとき $|W| = n$から $n$次対称群 $\mathfrak{S}_n$でもあり、
  \begin{eqnarray*}
    \gal(L_f/K) & \longrightarrow & \mathfrak{S}_W (= \mathfrak{S}_n) \\
    \sigma & \longmapsto & \sigma|_W
  \end{eqnarray*}
  という単射群準同型が存在する。 $(\mathfrak{S}_W に \gal(L_f/K)は埋め込める)$

  とくに $|\gal(L_f/K)| = [L_f:K] \leq n!$である。
\end{prop}

\begin{proof}
  ${}^\forall \sigma \in \gal(L_f/K) = \aut_K(L_f)$は
  $f(\sigma(x)) = \sigma(f(x)) = 0$より $\sigma(x) \in W$
  だから $\sigma(W) \subset W$なので
  \begin{eqnarray*}
    \sigma|_W : W & \longrightarrow & W \\
    x & \longmapsto & \sigma(x)
  \end{eqnarray*}
  となり $\sigma$は体の準同型より単射であって $|W| = n$で有限集合なので
  これは全単射である。
  したがって $\sigma|_W$は $W$上の全単射写像の群である $\mathfrak{S}_W$の元となる。
  $\sigma = \tau \in \gal(L_f/K)$のとき、 $\sigma|_W = \tau|_W$であるので
  $\gal(L_f/K) \longrightarrow \mathfrak{S}_W , \sigma \longmapsto \sigma|_W$は写像になっている。
  また、 $\sigma|_W = \tau|_W$のとき、
   $\aut_K(L_f)$の元としての $\sigma , \tau$は $K$を動かさないので
   最小分解体の定義から $L_f = K(W)$なので $W$の動かし方で定まるから $\sigma = \tau$である。
   したがって制限写像 $\gal(L_f/K) \longrightarrow \mathfrak{S}_W$は単射である。

   $L_f/K$は定義 $(\mathrm{\ref{defi:galoispolynomial}})$から有限次\rm{Galois}なので
   命題 $(\mathrm{\ref{prop:12.1}})$の $(1) \Leftrightarrow (4)$から
   $[L_f:K] = h_{L_f}(L_f) = |\aut_K(L_f)| = |\gal(L_f/K)|$である。
   ここで上述のことから $\gal(L_f/K)$は $\mathfrak{S}_W = \mathfrak{S}_n$に
   埋め込めるから $|\gal(L_f/K)| = [L_f:K] \leq |\mathfrak{S}_n| = n!$より示された。
\end{proof}

\begin{corl}
  一般の $n$次多項式 $f \in K[X]$の最小分解体 $L$の拡大次数は $n!$以下である。
\end{corl}

\begin{proof}
  命題 $(\mathrm{\ref{prop:galoispolynomial}})$で $f$は分離多項式とは限らないので
  $|W| \leq n$であるから $|\mathfrak{S}_W| \leq |\mathfrak{S}_n|$である。
  埋め込むことは同様にできるから $\gal(L_f/K)$を $\aut_K(L)$として
  $|\aut_K(L)| \leq |\mathfrak{S}_W| \leq |\mathfrak{S}_n| = n!$より成立。
\end{proof}

\begin{prop}
  分離多項式 $f \in K[X] - K$の根の集合 $W$とその元 $x , y \in W$に対して以下は同値。

  $(1)$
  $x$と $y$は $K$上共役。

  $(2)$
  $x$と $y$は同じ $\gal(L_f/K)-$軌道上に属する。

  $(3)$
  $x$と $y$は $f$の同じ既約成分の根。

  とくに $f$が既約であるためには $W \neq \emptyset$かつ $\gal(L_f/K)$が $W$に推移的に作用することが必要十分である。
  $(群 Gが集合 Xに推移的に作用するとは G-軌道 G(x) := \{ \sigma(x) | \sigma \in G \} とするとき G(x) = Xとなること)$
\end{prop}

\begin{proof}
  $\Omega$を $K$の代数閉包とする。

  $(1) \Leftrightarrow (2)$

  $f$が分離的なので $L_f/K$は有限次\rm{Galois}拡大であるから正規なので
  $\sigma \in \aut_K(\Omega) , \sigma(L_f) = L_f$を満たすから
  $\sigma|_{L_f} \in \aut_K(L_f) = \gal(L_f/K)$となる。
  また、 $\sigma \in \gal(L_f/K)$は系 $(\mathrm{\ref{corl:7.6}})$より
  $\tilde{\sigma} \in \aut_K(\Omega)$に拡張できる。
  これより
  \begin{eqnarray*}
    xと yが K上共役 & \Leftrightarrow & {}^\exists \sigma \in \aut_K(\Omega)
    , x = \sigma(y) \\
    & \Leftrightarrow & y \in \{ \sigma(x) | \sigma \in \gal(L_f/K) \} \\
    & \Leftrightarrow & yは xの \gal(L_f/K)-軌道に含まれる
  \end{eqnarray*}
  となる。

$(1) \Leftrightarrow (3)$

命題 $(\mathrm{\ref{prop:conjugate}})$の $(1) \Leftrightarrow (3)$より
$x$と $y$が $K$上共役 $\Leftrightarrow $ $x$と $y$の $K$上の最小多項式は同じ
なのでその最小多項式を $g \in K[X] - K$とすれば $g$は $f$の既約成分であるので
示された。

もし $\gal(L_f/K)$が $W (\neq \emptyset)$に推移的に作用するとすると、ある $f$の根 $x$に対して
その $\gal(L_f/K)-$軌道は $W$に一致するので任意の $f$の根は $(2) \Leftrightarrow (3)$から $f$の同じ既約成分の根になる。
したがって $f$の根はすべて $f$の既約成分の根になるから $f$は既約。
$f$が既約であるとき $(2) \Leftrightarrow (3)$から
すべての根はある $f$の根 $x$と同じ $\gal(L_f/K)-$軌道上に属するから
$W \subset \gal(L_f/K)-$軌道である。
また、 $x$の軌道はすべて $f$の根になるから $W \supset \gal(L_f/K)-$軌道より
$W = \gal(L_f/K)-$軌道となり推移的である。
\end{proof}

\begin{exam}
  $K:$体、 $L := K(T_1 , \dots , T_n) : n$変数の有理関数体とする。
  $G := \mathfrak{S}_n$として $T_i$の添字の置換とする。
  つまり、 $\sigma \in G$と $f = f(T_1 , \dots , T_n) \in L$に対して、
  $\sigma f := f(T_{\sigma(1)} , \dots , T_{\sigma(n)})$と作用させることとする。
  このとき、 $G$の元は $T_i$を写し、 $K$の元は動かさないので $L$の体の自己同型とみなせるので $G \subset \aut_{体}(L)$となる。

  $M := L^G$とおくとこれは $T_1 , \dots , T_n$の対称有理式の集合になる。
  このとき $L/M$が\rm{Galois}となって、 $G = \gal(L/M)$を満たす。
  とくに $[L:M] = n!$となる。
\end{exam}

\begin{proof}
  $s_i := (T_1 , \dots , T_n の i次基本対称式)$とすると $s_i \in L$である。
  つまり、 $s_1 = T_1 + \cdots + T_n , s_2 = T_1 T_2 + T_1 T_3 + \cdots + T_{n-1} T_n , \cdots , s_n = T_1 \cdots T_n$となっている。
  $M_0 := K(s_1 , \dots , s_n)$とおくと基本対称式は文字を置換しても
  同じままなので $M_0$は $G$で固定される。
  よって $M_0 \subset M$である。

  ここで $T_1 , \dots , T_n$は解と係数の関係から
  $X^n - s_1 X^{n-1} + \cdots + (-1)^n s_n \in M_0[X]$の根になる。
  $T_1 , \dots , T_n$はそれぞれ異なるから命題 $(\mathrm{\ref{prop:9.2}})$から
  この多項式は分離的である。
  $L$はこの多項式の最小分解体なので定義 $(\mathrm{\ref{defi:galoispolynomial}})$
  から $L/M_0$は有限次\rm{Galois}拡大になる。
  命題 $(\mathrm{\ref{prop:galoispolynomial}})$から
  $[L:M_0] \leq n!$である。
  また、 $L/M$は\rm{Artin}の定理 $(\mathrm{\ref{theo:artin}})$から
  \rm{Galois}拡大で $G = \mathfrak{S}_n = \aut_M(L)$であり、
  \rm{Rem} $(\mathrm{\ref{rem:tokutyou}})$から $[L:M] = |\aut_M(L)| = |\mathfrak{S}_n| = n!$となる。
  よって $M_0 \subset M$と $[L:M_0] \leq n! = [L:M]$より $M_0 = M$となる。
  以上より $M$は $T_1 , \dots , T_n$の対称有理式の集合になり、
  $G = \mathfrak{S}_n = \gal(L/M)$で、 $[L:M] = n!$となる。
\end{proof}

\begin{fact}
  $n \geq 5$ならば $n$次交代群 $\mathfrak{A}_n$は非アーベル単純群なので
  非自明な正規部分群を持たない。
  命題 $(\mathrm{\ref{theo:koukanshi}})$より可解群となるための可解列に出てくる
  交換子群は正規部分群であるので $\mathfrak{A}_n$は可解群にならない。
  よって $\mathfrak{A}_n$を含む $\mathfrak{S}_n$は $n \geq 5$で非可解群。
  任意の $n$次分離多項式 $(\in \mathbb{Q}[X])$の\rm{Galois}群は命題 $(\mathrm{\ref{prop:galoispolynomial}})$より $\mathfrak{S}_n$の
  部分群に同型である。
  これより定理 $\mathrm{\ref{theo:kakai}}$から
  $5$次以上の一般代数方程式は解の公式を持たないことがわかる。
\end{fact}

\subsection{IGP (Inverse Galois Problem)}

  \rm{Galois}の逆問題 $(\mathrm{IGP \  Inverse \  Galois \  Problem})$とは
  $K:$体、 $G:$有限群が与えられたとき、
  $\gal(L/K) \cong G$となる\rm{Galois}拡大 $L/K$は作れるかというもの

\begin{fact}
  \rm{Hilbert}の既約性定理

  $X^n - s_1 X^{n-1} + \cdots + (-1)^n s_n$はほとんどの $(有限個の例外を除き)$
  $(s_1 , \dots , s_n) \in \mathbb{Q}^n$に対し既約で
  その\rm{Galois}群は $\mathfrak{S}_n$と同型。
\end{fact}

\subsection{無限次Galois拡大}

$L/K$が無限次 $(かもしれない)$\rm{Galois}拡大のとき
$\gal(L/K)$は $\mathrm{profinite} (副有限、射影有限)$群 $(有限群の射影極限になっている群)$である。
つまり、 $L'/K , L''/K (L' \subset L'')$を $L$に含まれる任意の有限次\rm{Galois}拡大とし、
制限写像 $\gal(L''/K) \longrightarrow \gal(L'/K) , \sigma \longmapsto \sigma|_{L'}$
による射影極限
\begin{eqnarray*}
  \gal(L/K) = \mathop{\varprojlim}\limits_{L'/K} \gal(L'/K) \subset \prod_{L'/K} \gal(L'/K)
\end{eqnarray*}
で定義される。
$\gal(L'/K)$は離散位相によって位相群になるので
$\gal(L/K)$にはその直積位相が入り、これを\rm{Krull}位相という。

\begin{theo}
  \rm{Galois}理論の基本定理の無限次版

  $L/K:$\rm{Galois}拡大、 $G := \gal(L/K)$とすると次の一対一対応がある。

  \begin{eqnarray*}
    \{ L/Kの部分体 \} & \overset{1:1}{\longleftrightarrow} & \{ Gの閉部分群 \} \\
    M & \longmapsto & \aut_M(L) = \gal(L/M) \\
    L^H & \leftlongmapsto & H \\
    \{ L/Kの部分体で K上有限次のもの \} & \overset{1:1}{\longleftrightarrow} & \{ Gの開部分群 (指数有限の部分群) \} \\
    \{ L/Kの部分体で K上有限次のものでかつ K上 \mathrm{Galois}になるもの \} & \overset{1:1}{\longleftrightarrow} & \{ Gの開正規部分群 \}
  \end{eqnarray*}
  他の性質は有限次のとき $(\mathrm{\ref{theo:galois}})$と同じ。
\end{theo}

\begin{defi}
  $K:$体、 $K^{\mathrm{sep}}:K$の分離閉包とするとき、
  命題 $(\mathrm{\ref{prop:9.11}})$から
  $K$の代数閉包の相対的分離閉包が $K^{\mathrm{sep}}$になるから
  ${}^\forall x \in K^{\mathrm{sep}}$は $K$上代数的かつ分離的なので
  $K^{\mathrm{sep}}/K$は\rm{Galois}であり、
  $G_K := \gal(K^{\mathrm{sep}}/K)$を $K$の
  \underline{絶対 \rm{Galois}群}という。
\end{defi}

すると、 $L'/K$を $K^{\mathrm{sep}}/K$に含まれる有限次\rm{Galois}拡大と
するとき $G_K = \mathop{\varprojlim}\limits_{L'/K} \gal(L'/K)$となり、
とくに ${}^\forall L'/K$に対し
$G_K \xlongrightarrow{sur} \gal(L'/K)$があるから
${}^\forall L'/K$に $G_K$が作用していると考えられる。

逆に $G_K$から $K$を作ることも考えられる。

\begin{theo}
  \rm{Neukirch} $-$ 内田 $(- \mathrm{Pop})$の定理

  $K_1 , K_2:$素体上有限生成な体。
  \begin{eqnarray*}
    G_{K_1} \cong G_{K_2} & \Rightarrow & K_1 \cong K_2 \\
    (位相群として同型) &  & (体として同型)
  \end{eqnarray*}
  が成り立つ。
  これの一般化である\rm{Grothendieck}予想もある。
\end{theo}

\subsection{いろいろなGalois拡大 (Abel拡大)}

\subsubsection{有限体}

以下では $K:$有限体、 $\Char(K) = p > 0$で
$[K:F_p] = f , |K| = p^f = q$とする。

\begin{lemm} \label{lemm12.4}
  体 $F$の乗法群 $F^\times$の有限部分群は巡回群。
\end{lemm}

\begin{proof}
  $G$を $F^\times$の有限部分群、 $N$を $G$の冪数とする。
  ただし、冪数とは $G$の元の位数の最小公倍数のことである。
  このとき正整数 $n_1 | n_2 | \cdots | n_r$によって
  $G \cong \Z/n_1 \Z \oplus \cdots \Z/n_r \Z$
  で $N = \mathrm{LCM} (n_1 , \dots , n_r) = n_r$となる。
  このとき ${}^\forall x \in G , x^N = 1$より
  $G$の元は $X^N - 1$の根であり、この多項式は $F$に高々 $N$個しか根を持たないので
  $|G| \leq N$となる。
  そして、 $|\Z/n_r \Z| = n_r = N$より
  $G = \Z/n_r \Z$となるしかなく、したがって $G$は巡回群になる。
\end{proof}

\begin{corl} \label{corl:12.5}
  $K$が $q$元体ならば $K^\times \cong \Z/(q-1)\Z$で
  位数 $q - 1$の巡回群となる。
\end{corl}

\begin{proof}
  $K^\times$は $0$を除いた $q-1$個の元の有限群なので成立。
\end{proof}

\begin{corl} \label{corl:12.6}
  位数 $q$の有限体 $K$は同型を除き一意に定まる。
  これを $\F_q$と書く。
\end{corl}

\begin{proof}
  有限体 $K$は素体として $\F_p$と同型な体を含む。
  $\Omega$を $\F_p$の代数閉包とすると $K/\F_p$が
  有限次拡大より代数拡大なので定理 $(\mathrm{\ref{theo:7.3}})$から
  $\Omega$に $K$を埋め込める。
  $K$の元は系 $(\mathrm{\ref{corl:12.5}})$より $X^{q-1} - 1$の根と $0$ですべて出しつくされるので
  $K = \{ x \in \Omega | x^{q} = x \}$と書ける。
  $\Omega$に埋め込めばすべてこの形に書けるので同型を除き一意に定まる。
\end{proof}

\begin{corl} \label{corl:12.7}
  各 $n \in \Z^+$に対し $\F_q$の $n$次拡大は
  同型を除きただ一つ存在しそれは $\F_{q^n}$である。
  とくに $\F_p$の代数閉包 $\Omega$の中では唯一つである。
\end{corl}

\begin{proof}
  $\F_q$の $n$次拡大は $\F_p$の $q^n$次拡大なので
  系 $(\mathrm{\ref{corl:12.6}})$を $q^n$について適用すれば良い。
  $\Omega$の中では $\F_{q^n} = \{ x \in \Omega | x^{q^n} = x \}$
  として書けるので唯一つに定まる。
\end{proof}

\begin{prop} \label{prop:finitegalois}
  $\F_{q^n}/\F_q$は\rm{Galois}拡大であり
  \begin{eqnarray*}
    \Z/n \Z & \longrightarrow & \gal(\F_{q^n}/\F_q) \\
    1 & \longmapsto & \phi_q
  \end{eqnarray*}
  で定める準同型写像は同型写像になる。
  ただし、 $\phi_q$は
  \begin{eqnarray*}
    \phi_q : \F_{q^n} & \longrightarrow & \F_{q^n} \\
    x & \longmapsto & x^q
  \end{eqnarray*}
  とする。
\end{prop}

\begin{proof}
  $\F_q$の任意の $n$次拡大体 $L$は系 $(\mathrm{\ref{corl:12.7}})$より
  $L \cong \F_{q^n}$となるので $L = \F_{q^n}$とする。
  定義から $\phi_q \in \aut(\F_{q^n})$である。
  ${}^\forall x \in \F_q$に対しては
  $\F_q = \{ x \in \Omega | x^q = x \}$より
  $\phi_q(x) = x^q = x$なので $\F_q$上恒等的だから
  $\phi_q \in \aut_{\F_q}(\F_{q^n})$となる。

  また、 $\phi_q$は位数 $n$になることを示す。
  つまり、 $\phi_q^i$が $i = 1 , \dots , n-1$で $\phi_q^i \neq \rm{Id}$となり、
  $i = n$で $\phi_q^i = \rm{Id}$となればよい。
  $\phi_q^i : x \longmapsto \phi_q^i(x) = x^{q^i}$であるからもしこれが恒等であるとすると ${}^\forall x \in \F_{q^n} , x^{q^i} = x$であるので
  $\F_{q^n}$は系 $(\mathrm{\ref{corl:12.7}})$の証明における $\{ x \in \Omega | x^{q^i} = x \} = \F_{q^i}$の部分集合になるから
  元の個数を考えれば $i = n$でそのようになることがわかる。
  したがって $\phi_q$は位数 $n$である。

  $G := \langle \phi_q \rangle \subset \aut(\F_{q^n})$とする。
  $\phi_q$の位数が $n$より $\langle \phi_q \rangle \cong \Z/n\Z$となる。
  $L/L^G$は\rm{Artin}の定理 $(\mathrm{\ref{theo:artin}})$より
  \rm{Galois}拡大で $[L:L^G] = |G| = n$となる。
  また、 $L = \F_{q^n}$より $[L:\F_q] = n$なので
  $[L:\F_q] = [L:L^G]$と $\F_q , L^G \subset L$より $\F_q = L^G$となるから
  $\F_{q^n}/\F_q$は\rm{Galois}拡大でその\rm{Galois}群は $\Z/n\Z$と同型。
\end{proof}

\begin{corl}
  $\F_q$の代数閉包を $\overline{\F_q}$とすると
  $\F_q$の絶対\rm{Galois}群 $G_{\F_q} := \gal(\overline{\F_q}/\F_q)$であり、
  このとき $G_{\F_q} \cong \hat{\Z} := \mathop{\varprojlim}\limits_{n} \Z/n\Z$となる。
  また、 $\Z_l := \mathop{\varprojlim}\limits_{m}\Z/l^m\Z$とするとき
  $\hat{\Z}$は $\prod_{l:素数} \Z_l$とも書ける。
\end{corl}

\begin{exam}
  形式的冪級数体 $K := \mathbb{C}((X))$の絶対\rm{Galois}群は
  $G_K \cong \hat{\Z}$である。
\end{exam}

\subsubsection{円分拡大}

\begin{defi} \label{defi:cyclotomic}
  $K:$体、 $n \in \Z^+$とする。

  $X^n - 1 \in K[X]$の $K$上の最小分解体を $K$の
  \underline{$n$分拡大 $(\mathrm{n-th \  cyclotomic \  extension})$}といい、
  ある $n$に対する $n$分拡大を\underline{円分拡大 $(\mathrm{cyclotomic \  extension})$}という。
  とくに $K = \Q$のとき $n$分体、円分体という。

  $p \mid n$のとき、 $n = p^e m$かつ $p \nmid m$となる $e , m \in \Z^+$が存在して、 $X^n - 1 = X^{p^e m} - 1 = (X^m - 1)^{p^e}$となるので
  $X^n - 1$の最小分解体と $X^m - 1$の最小分解体は一致するから
  以降は $p \nmid n$で考える。

  $X^n - 1 \in K[X]$が分離的であるときは定義 $(\mathrm{\ref{defi:galoispolynomial}})$より有限次\rm{Galois}拡大である。
\end{defi}

\begin{lemm} \label{lemm:pnmidm}
  $\Char(K) = p > 0$であるとき $X^n - 1 \in K[X]$について以下は同値。

  $(1)$
  $p \nmid n$

  $(2)$
  $X^n - 1$は分離的。
\end{lemm}

\begin{proof}
  命題 $(\mathrm{\ref{prop:9.3}})$の $(1) \Leftrightarrow (4)$より
  $X^n - 1$は分離的 $\Leftrightarrow X^n - 1 \notin K[X^p] \Leftrightarrow p \nmid n$なので成立。
\end{proof}

\begin{rem} \label{rem:cyclotomicgalois}
  補題 $(\mathrm{\ref{lemm:pnmidm}})$の同値からいま $p \nmid m$で考えてるので
  $X^n - 1$は分離的だから円分拡大は\rm{Galois}拡大になる。
\end{rem}

\begin{defi} \label{defi:rootofunity}
  $K$の元 $\zeta$が $1$の $n$乗根とは
  ある $n \in \Z^+$に対して $\zeta^n = 1$となることである。
  原始 $n$乗根とは $\zeta$の位数が $n$であることである。
  つまり原始 $n$乗根は $n$乗して初めて $1$になる $K$の元のこと。
\end{defi}

\begin{lemm} \label{lemm:rootgroup}
  $K$の代数閉包を $\Omega$とし、それに含まれる $1 (\in K)$の $n$乗根全体の集合を
  $\mu_n (:= \{ 1の n乗根 \in \Omega \})$とする。
  このとき $\mu_n$は $K$の積で群を成す。
  これは原始 $n$乗根をもち、それの冪乗で任意の $\mu_n$の元を表せる。
\end{lemm}

\begin{proof}
  結合法則は $K$より成り立つ。
  $\zeta_i , \zeta_j \in \mu_n$に対して $(\zeta_i \zeta_j)^n = \zeta_i^n \zeta_j^n = 1 \cdot 1 = 1$より $\zeta_i , \zeta_j \in \mu_n$なので積で閉じている。
  単位元は $1 \in K$が $1^n = 1$より存在している。
  $\zeta_i \zeta_i^{n-1} = \zeta_i^n = 1$から $\zeta_i^{-1} = \zeta_i^{n-1}$
  から逆元が存在するので $\mu_n$は群。

  とくにこれは $K^\times$の有限部分群なので補題 $(\mathrm{\ref{lemm12.4}})$から
  巡回群になるので生成元 $\zeta \in \mu_n$が存在する。
  これは位数 $n$なので定義 $(\mathrm{\ref{defi:rootofunity}})$から原始 $n$乗根である。
  したがって $|\mu_n| \leq n$と $\zeta$の位数が $n$より $|\mu_n| = n$で
  ${}^\forall x \in \mu_n$に対して $x = \zeta^i$となる
  $j \in \Z , 1 \leq i \leq n$が存在する。
\end{proof}

\begin{lemm} \label{lemm:rootcong1}
  補題 $(\mathrm{\ref{lemm:rootgroup}})$の文字を用いて
  $\mu_n \cong \Z/n\Z$が成り立つ。
\end{lemm}

\begin{proof}
  $\zeta$は $\mu_n$の生成元なので
  $\zeta_i := \zeta^i$とする。
  このとき
  \begin{eqnarray*}
    \phi : \mu_n & \longrightarrow & \Z/n\Z \\
    \zeta_i = \zeta^i & \longmapsto & i
  \end{eqnarray*}
  とすると $\zeta^i = \zeta^j$のとき $\zeta^i \zeta^{n-i} = \zeta^j \zeta^{n-i} \Leftrightarrow 1 = \zeta^{n+j-i} \Leftrightarrow 1 = \zeta^{j-i}$
  より $j - i \in n\Z$から $j = i \in \Z/n\Z$なので写像になっている。
  $i = j$のとき $\zeta^i = \zeta^j$より単射で
  ${}^\forall i \in \Z/n\Z$で $1 \leq i \leq n$だから
  $\zeta^i \in \mu_n$を取ればいいから全射。
  また、 $\phi(\zeta^i \zeta^j) = \phi(\zeta^{i+j}) = i + j ,
  \phi(\zeta^i) + \phi(\zeta^j) = i + j$より群準同型になっているため
  $\mu_n \cong \Z/n\Z$である。
\end{proof}

\begin{lemm} \label{lemm:rootcong2}
  補題 $(\mathrm{\ref{lemm:rootgroup}})$の文字を用いて
  $\aut(\mu_n) \cong (\Z/n\Z)^*$が成り立つ。
\end{lemm}

\begin{proof}
  $\mu_n$上で
  \begin{eqnarray*}
    \phi : \mu_n & \longrightarrow & \mu_n \\
    \zeta ( := \zeta_1) & \longmapsto & \phi(\zeta) \\
    \zeta_i & \longmapsto & \phi(\zeta_i) = \phi(\zeta)^i (1 \leq i \leq n)
  \end{eqnarray*}
  とすると $\mu_n$の元は $\zeta$の冪で全て表せるので $\phi$が一意に定まり、
  準同型になる。
  もし $\phi(\zeta)$が $\mu_n$の原始 $n$乗根デないとすると
  ある $ 1 \leq j \leq n$で $\phi(\zeta)^j = 1$となる。
  しかし、 $\phi$が準同型より $\phi(\zeta^j) = 1$から $\zeta^j = 1$となりこれは
  $\zeta$が原始 $n$乗根であることに矛盾するので $\phi(\zeta)$も原始 $n$乗根である。
  $\phi(\zeta) \in \mu_n$より $\phi(\zeta) = \zeta^a$となる $a \in \Z^+ , 1 \leq a \leq n$が存在している。
  $a = 1$となると $a$と $n$は互いに素であるから $a \in (\Z/n\Z)^*$である。
  $a \neq 1$のとき $a \mid m$であるとすると $ak = n$となる $k \in \Z^+ , 1 \leq k < n$があり $a = n/k$となる。
  $\phi(\zeta)^k = (\zeta^{n/k})^k = \zeta^n = 1$となり、
  これは $\phi(\zeta)$が原始 $n$乗根であることに矛盾するから $a \nmid m$
  なので $a \in (\Z/n\Z)^*$
  この $a_\phi := a$を取れば
  \begin{eqnarray*}
    \aut(\mu_n) & \longrightarrow & (\Z/n\Z)^* \\
    \phi & \longmapsto & a_\phi
  \end{eqnarray*}
  とできてこの $a$で $\phi$が一意に定まるから全単射である。
  $\phi \circ \varphi (\zeta) = \phi(\zeta^{a_\varphi}) = (\zeta^{a_\varphi})^{a_\phi} = \zeta^{a_\varphi + a_\phi}$と
  $\phi(\zeta) = \zeta^{a_\phi} , \varphi(\zeta) = \zeta^{a_\varphi}$より
  準同型になるので
  $\aut(\mu_n) \cong (\Z/n\Z)^*$が成立する。
\end{proof}

\begin{prop}
  $K_n$を $K$の $n$分拡大とすると\rm{Rem} $(\mathrm{\ref{rem:cyclotomicgalois}})$から $K_n/K$は\rm{Galois}なので
  $G_n := \aut_K(K_n) = \gal(K_n/K)$とする。
  $\zeta$を原始 $n$乗根とし、 $\sigma \in G_n$に対して
  $\sigma(\zeta) = \zeta^{a_\sigma}$となる $a_\sigma \in \Z^+$をとる。
  このとき
  \begin{eqnarray*}
    \chi_n : G_n & \longrightarrow & (\Z/n\Z)^* (\cong \aut(\mu_n)) \\
    \sigma & \longmapsto & a_\sigma
  \end{eqnarray*}
  は単射群準同型となる。
  この $\chi_n$を\underline{$n$次円分指標 \  $(\mathrm{cyclotomic \  character})$}といい\rm{Galois}表現の一種である。
\end{prop}

\begin{proof}
  $\sigma \in G_n$に対して補題 $(\mathrm{\ref{lemm:rootcong2}})$と
  同様に $\sigma(\zeta) = \zeta^{a_\sigma}$で一意に定まるから
  $\chi_n$は単射群準同型である。
\end{proof}

\end{document}
