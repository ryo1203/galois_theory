\documentclass[../master_galois_theory]{subfiles}
\begin{document}

\setcounter{section}{11}

\section{Galois拡大再論}

\subsection{Galois拡大}

\begin{prop} \label{prop:12.1}
  代数拡大 $L/K$について次は同値

  $(1)$
  $L/K$は\rm{Galois}

  $(2)$
  $L/K$は正規かつ分離的

  $(2)'$
  ${}^\forall x \in L$に対し、その最小多項式は分離的かつ
  $L[X]$において一次因子の積に分解される。

  $(3)$
  $L/K$はある分離多項式族 $(f_i)_{i \in I}$の最小分解体

  さらに、 $L/K$が有限次なら次も同値

  $(4)$
  $[L:K] = h_L(L) (:= |\aut_K(L)|)$
\end{prop}

\begin{proof}
  $\Omega$を $K$の代数閉包とする。

  $(2) \Leftrightarrow (2)'$は正規の定義 $(\mathrm{\ref{defi:normal}})$と系 $(\mathrm{\ref{corl:separable}})$
  と多項式の分離性の定義 $(\mathrm{\ref{prop:9.2}})$から明らか。

  $(1) \Rightarrow (2)$

  ${}^\forall x \in L$とその最小多項式 $f \in K[X]$をとる。
  また、 $Y_x := \{ \sigma(x) | \sigma \in \aut_K(L) \}$
  と定めるとこれは $x$の $\Omega$における共役元の集合の部分集合になり、
  $\sigma \in \aut_K(L)$から $Y \subset L$である。
  命題 $(\mathrm{\ref{prop:conjugate}})$の $(1) \Leftrightarrow (3)$から
  $x$の共役元はすべて $f$の根なので高々 $\deg(f)$個しかないので $Y_x$は有限集合。
  $g := \prod_{y \in Y_x}(X - y) , n := \deg(g)$とする。
  $y$はすべて異なるから単根なので $g$は分離的である。
  また、 $y$は $x$の共役元より $f$の根でもあるから $g$のすべての根は $f$の根より
  $g|f$となる。

  $y \in Y_x \subset L$よりその元から作られる基本対称式は $L$に含まれるので
  $g = \sum_{i=1}^n a_i X^i , a_i \in L$と書ける。
  $\sigma g = \sum_{i=1}^n \sigma(a_i) X^i$とすると係数だけに $\sigma$をかけているから $(\sigma g)(X) = \prod_{y \in Y_x}(X - \sigma(y))$となる。
  ここで $y \in Y_x$より $y = \tau(x) , \tau \in \aut_K(L)$となるものが存在する。
  $\aut_K(L)$は自己同型写像であるから $\sigma \circ \tau \in \aut_K(L)$より $\sigma(y) = \sigma \circ \tau(x)) \in Y_x$となる。
  ここで $Y_x$は有限集合であることと $\sigma$は体の準同型より単射なので
  それぞれの $y$は $\sigma$によりそれぞれ異なる $Y_x$の元に行く。
  したがって $(\sigma g)(X) = \prod_{y \in Y_x}(X - y) = g(X)$となるから
  $a_i$は ${}^\forall \sigma \in \aut_K(L)$によって動かされない。
  $L/K$が\rm{Galois}より $L^{\aut_K(L)} = K$より $a_i \in K$であるから
  $g \in K[X]$である。

  $g , f \in K[X]$で $g|f$より $f$の最小性から $f = g$なので
  任意の $x \in L$の最小多項式は
  $f = \prod_{y \in Y_x}(X - y)$と $L[X]$上で一次因子の積に分解されるので
  $L/K$は正規。
  また、 $g (= f)$は分離的でもあったので任意の最小多項式が分離的より
  系 $(\mathrm{\ref{corl:separable}})$より $L/K$は分離的であるので
  $L/K$は正規かつ分離的。

  $(2) \Rightarrow (1)$

  $L = K$のとき $L^{\aut_K(L)} = K^{\aut_K(K)} = K$で成立。
  $L \neq K$のとき $L \supsetneq K$であるから ${}^\forall x \in L - K$をとる。
  これがある $\sigma \in \aut_K(L)$で $\sigma(x) \neq x$となればよい。

  $x$の最小多項式を $f \in K[X]$とすると $x \in L - K$より $\deg(f) > 1$であり、
  仮定から $L/K$が分離的より系 $(\mathrm{\ref{corl:separable}})$から $f$が単根を持つので
  定義より分離的だから $f(y) = 0$で $y \neq x$であるような元 $y \in \Omega$が存在する。
  $y$の $K$上の最小多項式も $f$なので命題 $(\mathrm{\ref{prop:conjugate}})$の
  $(2) \Leftrightarrow (3)$から $\sigma(x) = y$となるような $\sigma \in \aut_K(\Omega)$が存在する。
  仮定から $L/K$は正規なので命題 $(\mathrm{\ref{prop:11.1}})$の $(1) \Leftrightarrow (3)$から $\sigma(L) = L$より
  $\sigma|_L \in \aut_K(L)$となる。
  この $\sigma$により $\sigma(x) = y \neq x$なので $x$は固定されないから
  固定されるのは $K$の元のみなので $L^{\aut_K(L)} = K$となり
  定義より $L/K$は\rm{Galois}である。

  $(2) \Leftrightarrow (3)$

  命題 $(\mathrm{\ref{prop:11.1}})$の $(1) \Leftrightarrow (5)$より
  「規 $\Leftrightarrow$ある多項式族 $(f_i)_{i \in I}$の最小分解体」が言えている。
  その多項式族は ${}^\forall x \in L$の最小多項式の族であったので
  系 $(\mathrm{\ref{corl:separable}})$より
  「分離的 $\Leftrightarrow$多項式族のすべての多項式が分離的」が言えている。

  $(2) \Leftrightarrow (4)$

  有限次拡大のとき系 $(\mathrm{\ref{corl:11.2}})$から
  「正規 $\Leftrightarrow [L:K]_s = h_L(L)$」が言えている。
  定義より「分離的 $\Leftrightarrow [L:K] = [L:K]_s$」なので
  「正規かつ分離的 $\Leftrightarrow [L:K] = [L:K]_s = h_L(L)$」となり示された。
\end{proof}

\subsection{多項式のGalois群}

\begin{defi} \label{defi:galoispolynomial}
  $K:$体、 $f \in K[X] - K:$分離多項式、 $L_f:f$の $K$上の最小分解体とするとき
  その根をすべて添加しているので命題 $(\mathrm{\ref{prop:6.7}})$から
  $L_f/K$は有限次だから
  命題 $(\mathrm{\ref{prop:12.1}})$の $(1) \Leftrightarrow (3)$から
  $L_f/K$は有限次\rm{Galois}拡大である。
  このとき $\gal(L_f/K)$を\underline{$f$の $K$上の\rm{Galois}群}という。
\end{defi}

\begin{prop} \label{prop:galoispolynomial}
  分離多項式 $f \in K[X] - K$にたいしてその最小分解体 $L_f$を考える。
  $\Omega$を $K$の代数閉包で $L_f$を含むもの、
  $W := \{ x \in \Omega | fの根 \}$とする。
  $f$は分離多項式なので $|W| = n := \deg(f)$となる。
  このとき $\gal(L_f/K)$は $W$に作用し、根の置換を引き起こす。
  したがって $W$の自己同型写像の群、つまり $W$の置換群を $\mathfrak{S}_W$とするとき $|W| = n$から $n$次対称群 $\mathfrak{S}_n$でもあり、
  \begin{eqnarray*}
    \gal(L_f/K) & \longrightarrow & \mathfrak{S}_W (= \mathfrak{S}_n) \\
    \sigma & \longmapsto & \sigma|_W
  \end{eqnarray*}
  という単射群準同型が存在する。 $(\mathfrak{S}_W に \gal(L_f/K)は埋め込める)$

  とくに $|\gal(L_f/K)| = [L_f:K] \leq n!$である。
\end{prop}

\begin{proof}
  ${}^\forall \sigma \in \gal(L_f/K) = \aut_K(L_f)$は
  $f(\sigma(x)) = \sigma(f(x)) = 0$より $\sigma(x) \in W$
  だから $\sigma(W) \subset W$なので
  \begin{eqnarray*}
    \sigma|_W : W & \longrightarrow & W \\
    x & \longmapsto & \sigma(x)
  \end{eqnarray*}
  となり $\sigma$は体の準同型より単射であって $|W| = n$で有限集合なので
  これは全単射である。
  したがって $\sigma|_W$は $W$上の全単射写像の群である $\mathfrak{S}_W$の元となる。
  $\sigma = \tau \in \gal(L_f/K)$のとき、 $\sigma|_W = \tau|_W$であるので
  $\gal(L_f/K) \longrightarrow \mathfrak{S}_W , \sigma \longmapsto \sigma|_W$は写像になっている。
  また、 $\sigma|_W = \tau|_W$のとき、
   $\aut_K(L_f)$の元としての $\sigma , \tau$は $K$を動かさないので
   最小分解体の定義から $L_f = K(W)$なので $W$の動かし方で定まるから $\sigma = \tau$である。
   したがって制限写像 $\gal(L_f/K) \longrightarrow \mathfrak{S}_W$は単射である。

   $L_f/K$は定義 $(\mathrm{\ref{defi:galoispolynomial}})$から有限次\rm{Galois}なので
   命題 $(\mathrm{\ref{prop:12.1}})$の $(1) \Leftrightarrow (4)$から
   $[L_f:K] = h_{L_f}(L_f) = |\aut_K(L_f)| = |\gal(L_f/K)|$である。
   ここで上述のことから $\gal(L_f/K)$は $\mathfrak{S}_W = \mathfrak{S}_n$に
   埋め込めるから $|\gal(L_f/K)| = [L_f:K] \leq |\mathfrak{S}_n| = n!$より示された。
\end{proof}

\begin{corl}
  一般の $n$次多項式 $f \in K[X]$の最小分解体 $L$の拡大次数は $n!$以下である。
\end{corl}

\begin{proof}
  命題 $(\mathrm{\ref{prop:galoispolynomial}})$で $f$は分離多項式とは限らないので
  $|W| \leq n$であるから $|\mathfrak{S}_W| \leq |\mathfrak{S}_n|$である。
  埋め込むことは同様にできるから $\gal(L_f/K)$を $\aut_K(L)$として
  $|\aut_K(L)| \leq |\mathfrak{S}_W| \leq |\mathfrak{S}_n| = n!$より成立。
\end{proof}

\begin{prop}
  分離多項式 $f \in K[X] - K$の根の集合 $W$とその元 $x , y \in W$に対して以下は同値。

  $(1)$
  $x$と $y$は $K$上共役。

  $(2)$
  $x$と $y$は同じ $\gal(L_f/K)-$軌道上に属する。

  $(3)$
  $x$と $y$は $f$の同じ既約成分の根。

  とくに $f$が既約であるためには $W \neq \emptyset$かつ $\gal(L_f/K)$が $W$に推移的に作用することが必要十分である。
  $(群 Gが集合 Xに推移的に作用するとは G-軌道 G(x) := \{ \sigma(x) | \sigma \in G \} とするとき G(x) = Xとなること)$
\end{prop}

\begin{proof}
  $\Omega$を $K$の代数閉包とする。

  $(1) \Leftrightarrow (2)$

  $f$が分離的なので $L_f/K$は有限次\rm{Galois}拡大であるから正規なので
  $\sigma \in \aut_K(\Omega) , \sigma(L_f) = L_f$を満たすから
  $\sigma|_{L_f} \in \aut_K(L_f) = \gal(L_f/K)$となる。
  また、 $\sigma \in \gal(L_f/K)$は系 $(\mathrm{\ref{corl:7.6}})$より
  $\tilde{\sigma} \in \aut_K(\Omega)$に拡張できる。
  これより
  \begin{eqnarray*}
    xと yが K上共役 & \Leftrightarrow & {}^\exists \sigma \in \aut_K(\Omega)
    , x = \sigma(y) \\
    & \Leftrightarrow & y \in \{ \sigma(x) | \sigma \in \gal(L_f/K) \} \\
    & \Leftrightarrow & yは xの \gal(L_f/K)-軌道に含まれる
  \end{eqnarray*}
  となる。

$(1) \Leftrightarrow (3)$

命題 $(\mathrm{\ref{prop:conjugate}})$の $(1) \Leftrightarrow (3)$より
$x$と $y$が $K$上共役 $\Leftrightarrow $ $x$と $y$の $K$上の最小多項式は同じ
なのでその最小多項式を $g \in K[X] - K$とすれば $g$は $f$の既約成分であるので
示された。

もし $\gal(L_f/K)$が $W (\neq \emptyset)$に推移的に作用するとすると、ある $f$の根 $x$に対して
その $\gal(L_f/K)-$軌道は $W$に一致するので任意の $f$の根は $(2) \Leftrightarrow (3)$から $f$の同じ既約成分の根になる。
したがって $f$の根はすべて $f$の既約成分の根になるから $f$は既約。
$f$が既約であるとき $(2) \Leftrightarrow (3)$から
すべての根はある $f$の根 $x$と同じ $\gal(L_f/K)-$軌道上に属するから
$W \subset \gal(L_f/K)-$軌道である。
また、 $x$の軌道はすべて $f$の根になるから $W \supset \gal(L_f/K)-$軌道より
$W = \gal(L_f/K)-$軌道となり推移的である。
\end{proof}

\begin{exam}
  $K:$体、 $L := K(T_1 , \dots , T_n) : n$変数の有理関数体とする。
  $G := \mathfrak{S}_n$として $T_i$の添字の置換とする。
  つまり、 $\sigma \in G$と $f = f(T_1 , \dots , T_n) \in L$に対して、
  $\sigma f := f(T_{\sigma(1)} , \dots , T_{\sigma(n)})$と作用させることとする。
  このとき、 $G$の元は $T_i$を写し、 $K$の元は動かさないので $L$の体の自己同型とみなせるので $G \subset \aut_{体}(L)$となる。

  $M := L^G$とおくとこれは $T_1 , \dots , T_n$の対称有理式の集合になる。
  このとき $L/M$が\rm{Galois}となって、 $G = \gal(L/M)$を満たす。
  とくに $[L:M] = n!$となる。
\end{exam}

\begin{proof}
  $s_i := (T_1 , \dots , T_n の i次基本対称式)$とすると $s_i \in L$である。
  つまり、 $s_1 = T_1 + \cdots + T_n , s_2 = T_1 T_2 + T_1 T_3 + \cdots + T_{n-1} T_n , \cdots , s_n = T_1 \cdots T_n$となっている。
  $M_0 := K(s_1 , \dots , s_n)$とおくと基本対称式は文字を置換しても
  同じままなので $M_0$は $G$で固定される。
  よって $M_0 \subset M$である。

  ここで $T_1 , \dots , T_n$は解と係数の関係から
  $X^n - s_1 X^{n-1} + \cdots + (-1)^n s_n \in M_0[X]$の根になる。
  $T_1 , \dots , T_n$はそれぞれ異なるから命題 $(\mathrm{\ref{prop:9.2}})$から
  この多項式は分離的である。
  $L$はこの多項式の最小分解体なので定義 $(\mathrm{\ref{defi:galoispolynomial}})$
  から $L/M_0$は有限次\rm{Galois}拡大になる。
  命題 $(\mathrm{\ref{prop:galoispolynomial}})$から
  $[L:M_0] \leq n!$である。
  また、 $L/M$は\rm{Artin}の定理 $(\mathrm{\ref{theo:artin}})$から
  \rm{Galois}拡大で $G = \mathfrak{S}_n = \aut_M(L)$であり、
  \rm{Rem} $(\mathrm{\ref{rem:tokutyou}})$から $[L:M] = |\aut_M(L)| = |\mathfrak{S}_n| = n!$となる。
  よって $M_0 \subset M$と $[L:M_0] \leq n! = [L:M]$より $M_0 = M$となる。
  以上より $M$は $T_1 , \dots , T_n$の対称有理式の集合になり、
  $G = \mathfrak{S}_n = \gal(L/M)$で、 $[L:M] = n!$となる。
\end{proof}

\end{document}
