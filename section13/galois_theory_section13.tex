\documentclass[../master_galois_theory]{subfiles}
\begin{document}

\setcounter{section}{12}

\section{Galois理論の基本定理の別の定式化}

$K:$体、 $K^\mathrm{sep}:K$の分離閉包、 $A:$\rm{etale} $K-alg$に対して
$\s(A) := \Hom_{K-alg}(A,K^\mathrm{sep})$とおく。
このとき $K$の絶対\rm{Galois}群 $G_K := \gal(K^\mathrm{sep}/K)$は $\s(A)$に以下のように作用する。
\begin{eqnarray*}
  G_K \times \s(A) & \longrightarrow & \s(A) \\
  (\sigma , f) & \longmapsto & \sigma f
\end{eqnarray*}
ただし $\sigma f$は
\begin{eqnarray*}
  \sigma f : A & \longrightarrow & K^\mathrm{sep} \\
  x & \longmapsto & (\sigma f)(x) := \sigma(f(x))
\end{eqnarray*}
である。
$G_K$は定義 $(\mathrm{\ref{defi:absolutegalois}})$から位相群であり、
この位相についてこの作用は連続になり、
これは各 $f \in \s(A)$の固定化群が開であることと同値になっている。

$A = K[X]/(f)$のとき $f$のある根 $\alpha_i$に対して $\alpha_i = X + (f)$とすることで $\s(A)$の写像が一つ定まるので $\s(A) \cong \{ f の根\}$が成り立ち、
$\s(A)$を多項式の根のように見ることができる。

逆に $S$を $G_K$が連続に作用する有限集合 $(G_K-集合)$とすると
$\A(S) := \map_G(S,K^\mathrm{sep}) := \{ f : S \longrightarrow K^\mathrm{sep} | f(\sigma(x)) = \sigma(f(x)) , {}^\forall \sigma \in G_K \}$
とおいたとき $\A(S)$は $K-alg$でさらに有限次\rm{etale}でもある。

以上のことから以下の定理が成り立つ。
\begin{theo}
  次の反圏同値がある。
  \begin{eqnarray*}
    (\mathrm{etale} K-algの圏) & \cong & (有限 G_K-集合の圏) \\
    A & \longmapsto & \s(A) \\
    \A(S) & \leftlongmapsto & S \\
    K-alg \  \mathrm{hom} : A_1 \rightarrow A_2 & \longleftrightarrow & G_K-集合の射 : S_1 \leftarrow S_2 \\
    単射 \mathrm{hom} : A_1 \hookrightarrow A_2 & \longleftrightarrow & 全射 : S_1 \twoheadleftarrow S_2 \\
    全射 \mathrm{hom} : A_1 \twoheadrightarrow A_2 & \longleftrightarrow & 単射 : S_1 \hookleftarrow S_2 \\
    A : 体 & \longleftrightarrow & Sは一つの\mathrm{orbit}からなる \\
    A:Kの\mathrm{Galois}拡大 & \longleftrightarrow & \mathrm{Stab}_S := Sの固定化群が G_Kで正規 \\
    A:Kの\mathrm{Galois}拡大のとき A/Kの中間体 & \longleftrightarrow & 全射 : (一点) \twoheadleftarrow Sの中間集合
  \end{eqnarray*}
\end{theo}

\clearpage

\end{document}
