\documentclass[../master_galois_theory]{subfiles}
\begin{document}

\setcounter{section}{13}

\section{Galois cohomology}

\subsection{群のcohomology}

\begin{defi} \label{defi:cochain}
  $G:$群、 $M:$加法 $(\mathrm{Abel})$群で
  $G$は $M$に加群としての作用をしているとする。
  ここで以下のように $G^n$から $M$への写像全体の集合を
  $C^n (n \in \Z_{\geq 0})$として定める。
  \begin{eqnarray*}
    C^n = C^n(G,M) := \{ f : G^n \longrightarrow M \} = \map(G^n,M)
  \end{eqnarray*}
  ただし $G^0 = \{ e \}$と考えることで $C^0 := M$と定める。
  この $C^n$の各元を\underline{ $n$コチェイン $(\mathrm{cochain})$}という。
  $C^n$上へは $f , g \in C^n$に対して
  $(f+g)(x) := f(x) + g(x)$と演算を定めることで $C^n$は加法群となる。
\end{defi}

\begin{defi} \label{defi:coboundary}
  $C^n$から $C^{n+1}$への以下のように定まる写像 $\partial$を考える。
  \begin{eqnarray*}
    \partial = \partial^n : C^n & \longrightarrow & C^{n+1} \\
    f & \longmapsto & \partial f
  \end{eqnarray*}
  ここで $\partial f : G^{n+1} \longrightarrow M$は
  $G$が $M$へ作用していることに注意して
  \begin{eqnarray*}
    \partial f(g_1 , \dots , g_{n+1}) & = & g_1 f(g_2 , \dots , g_{n+1}) \\
    & + & \sum_{i=1}^n (-1)^i f(g_1 , \dots , g_i g_{i+1} , \dots , g_{n+1}) \\
    & + & (-1)^{n+1} f(g_1 , \dots , g_n)
  \end{eqnarray*}
と定める。
このときこの $\partial (= \partial^n) : C^n(G,M) \longrightarrow C^{n+1}(G,M)$は加法群の準同型になり、
これを\underline{$n$次のコバウンダリー $(双対境界)$作用素 $(\mathrm{coboundary \  operator})$}とよぶ。
\end{defi}

\begin{prop} \label{prop:partialpartial}
  コバウンダリー作用素 $\partial$に対して $\partial^{n+1} \circ \partial^n = 0$が成り立つ。
\end{prop}

\begin{proof}
  $4 \leq n$でまず考える。

  $(\partial^{n+1} \circ \partial^n)(f)(g_1 , \dots , g_{n+2}) = \partial^{n+1}(\partial^n f)(g_1 , \dots , g_{n+2})$なので
  $f' := \partial^n f$として $\partial^{n+1}f'(g_1 , \dots , g_{n+2})$は
  \begin{eqnarray*}
    \partial^{n+1}f'(g_1 , \dots , g_{n+2}) & = & g_1 f'(g_2 , \dots , g_{n+2}) \\
    & + & \sum_{i=1}^{n+1} (-1)^i f'(g_1 , \dots , g_i g_{i+1} , \dots , g_{n+2}) \\
    & + & (-1)^{n+1} f'(g_1 , \dots , g_{n+1})
  \end{eqnarray*}
  である。
  $f'(g_1 , \dots , g_i g_{i+1} , \dots , g_{n+2}) = \partial^n f(g_1 , \dots g_i g_{i+1} , \dots , g_{n+2})$を $i$の値によって計算する。

  ・ $i = 1$のとき
  \begin{eqnarray*}
    \partial^n f(g_1 g_2 , \dots ,g_{n+2}) & = & g_1 g_2 f(g_3 , \dots , g_{n+2}) \\
    & + & (-1)^1 f((g_1 g_2) g_3 , g_4 , \dots , g_{n+2}) \\
    & + & \sum_{k=3}^{n+1} (-1)^{k-1} f(g_1 g_2 , g_3 , \dots , g_i g_{i+1} , \dots , g_{n+2}) \\
    & + & (-1)^{n+1} f(g_1 g_2 , g_3 , \dots , g_n)
  \end{eqnarray*}
  ・ $i = 2$のとき
  \begin{eqnarray*}
    \partial^n f(g_1 , g_2 g_3 , g_4 , \dots , g_{n+2}) & = & g_1 f(g_2 g_3 , g_4 , \dots , g_{n+2}) \\
    & + & (-1)^1 f(g_1 (g_2 g_3) , g_4 , \dots , g_{n+2}) \\
    & + & (-1)^2 f(g_1 , (g_2 g_3) g_4 , g_5 , \dots , g_{n+2}) \\
    & + & \sum_{k=4}^{n+1} (-1)^{k-1} f(g_1 , g_2 g_3 , g_4 , \dots , g_i g_{i+1} , \dots , g_{n+2}) \\
    & + & (-1)^{n+1} f(g_1 , g_2 g_3 , g_4 , \dots , g_{n+1})
  \end{eqnarray*}
  ・ $3 \leq i \leq n-1$のとき
  \begin{eqnarray*}
    \partial^n f(g_1 , \dots , g_i g_{i+1} , \dots , g_{n+2}) & = & g_1 f(g_2 , \dots , g_i g_{i+1} , \dots , g_{n+2}) \\
    & + & \sum_{k=1}^{i-2} (-1)^k f(g_1 , \dots , g_k g_{k+1} , \dots , g_i g_{i+1} , \dots , g_{n+2}) \\
    & + & (-1)^{i-1} f(g_1 , \dots , g_{i-2} , g_{i-1} (g_i g_{i+1}) , g_{i+2} , \dots , g_{n+2}) \\
    & + & (-1)^i f(g_1 , \dots , g_{i-1} , (g_i g_{i+1}) g_{i+2} , g_{i+3} , \dots , g_{n+2}) \\
    & + & \sum_{k=i+2}^{n+1} (-1)^{k-1} f(g_1 , \dots , g_i g_{i+1} , \dots , g_k g_{k+1} , \dots , g_{n+2}) \\
    & + & (-1)^{n+1} f(g_1 , \dots , g_i g_{i+1} , \dots , g_{n+1})
  \end{eqnarray*}
  ・ $i = n$のとき
  \begin{eqnarray*}
    \partial^n f(g_1 , \dots , g_n g_{n+1} , g_{n+2}) & = & g_1 f(g_2 , \dots , g_n g_{n+1} , g_{n+2}) \\
    & + & \sum_{k=1}^{n-2} (-1)^k f(g_1 , \dots , g_k g_{k+1} , \dots , g_n g_{n+1} , g_{n+2}) \\
    & + & (-1)^{n-1} f(g_1 , \dots , g_{n-1} (g_n g_{n+1}) , g_{n+2}) \\
    & + & (-1)^n f(g_1 , \dots , g_{n-1} , (g_n g_{n+1}) g_{n+2}) \\
    & + & (-1)^{n+1} f(g_1 , \dots , g_{n-1} , g_n g_{n+1})
  \end{eqnarray*}
  ・ $i = n+1$のとき
  \begin{eqnarray*}
    \partial^n f(g_1 , \dots , g_{n+1} g_{n+2}) & = & g_1 f(g_2 , \dots , g_{n+1} g_{n+2}) \\
    & + & \sum_{k=1}^{n-1} (-1)^k f(g_1 , \dots , g_k g_{k+1} , \dots , g_n , g_{n+1} g_{n+2}) \\
    & + & (-1)^n f(g_1 , \dots , g_{n-1} , g_n (g_{n+1} g_{n+2})) \\
    & + & (-1)^{n+1} f(g_1 , \dots , g_n)
  \end{eqnarray*}
となる。

また、 $g_1 f'(g_2 , \dots , g_{n+2})$と $(-1)^{n+2} f'(g_1 , \dots , g_{n+1})$は以下のようになる。
\begin{eqnarray*}
  g_1 \partial^n f(g_2 , \dots , g_{n+2}) & = & g_1 (g_2 f(g_3 , \dots , g_{n+2}) \\
  & + & \sum_{i=2}^{n+1} (-1)^{i-1} f(g_2 , \dots , g_i g_{i+1} , \dots , g_{n+2}) \\
  & + & (-1)^{n+1} f(g_2 , \dots , g_{n+1}) ) \\
  (-1)^{n+2} \partial^n f(g_1 , \dots , g_{n+1}) & = & (-1)^{n+2} (g_1 f(g_2 , \dots , g_{n+1}) \\
  & + & \sum_{i=1}^n (-1)^i f(g_1 , \dots , g_i g_{i+1} , \dots , g_{n+1}) \\
  & + & (-1)^{n+1} f(g_1 , \dots , g_n) )
\end{eqnarray*}

これを $\partial^{n+1}(\partial^n f)(g_1 , \dots , g_{n+2})$の式に代入すると
\begin{eqnarray*}
  \partial^{n+1}(\partial^n f)(g_1 , \dots , g_{n+2}
  % g_1 f'(g_2 , \dots , g_{n+2})
  & = & \{ g_1 g_2 f(g_3 , \dots , g_{n+2}) \\
  & + & \sum_{i=2}^{n+1} (-1)^{i-1} g_1 f(g_2 , \dots , g_i g_{i+1} , \dots , g_{n+2}) \\
  & + & (-1)^{n+1} g_1 f(g_2 , \dots , g_{n+1}) \} \\
  % i = 1
  & + & (-1)^1 \{ g_1 g_2 f(g_3 , \dots , g_{n+2}) \\
  & + & (-1)^1 f((g_1 g_2) g_3 , g_4 , \dots , g_{n+2}) \\
  & + & \sum_{k=3}^{n+1} (-1)^{k-1} f(g_1 g_2 , g_3 , \dots , g_i g_{i+1} , \dots , g_{n+2}) \\
  & + & (-1)^{n+1} f(g_1 g_2 , g_3 , \dots , g_n) \} \\
  % i = 2
  & + & (-1)^2 \{ g_1 f(g_2 g_3 , g_4 , \dots , g_{n+2}) \\
  & + & (-1)^1 f(g_1 (g_2 g_3) , g_4 , \dots , g_{n+2}) \\
  & + & (-1)^2 f(g_1 , (g_2 g_3) g_4 , g_5 , \dots , g_{n+2}) \\
  & + & \sum_{k=4}^{n+1} (-1)^{k-1} f(g_1 , g_2 g_3 , g_4 , \dots , g_i g_{i+1} , \dots , g_{n+2}) \\
  & + & (-1)^{n+1} f(g_1 , g_2 g_3 , g_4 , \dots , g_{n+1}) \} \\
  % 3 \leq i \leq n-1
  & + & \sum_{i=3}^{n-1} (-1)^i \{ g_1 f(g_2 , \dots , g_i g_{i+1} , \dots , g_{n+2}) \\
  & + & \sum_{k=1}^{i-2} (-1)^k f(g_1 , \dots , g_k g_{k+1} , \dots , g_i g_{i+1} , \dots , g_{n+2}) \\
  & + & (-1)^{i-1} f(g_1 , \dots , g_{i-2} , g_{i-1} (g_i g_{i+1}) , g_{i+2} , \dots , g_{n+2}) \\
  & + & (-1)^i f(g_1 , \dots , g_{i-1} , (g_i g_{i+1}) g_{i+2} , g_{i+3} , \dots , g_{n+2}) \\
  & + & \sum_{k=i+2}^{n+1} (-1)^{k-1} f(g_1 , \dots , g_i g_{i+1} , \dots , g_k g_{k+1} , \dots , g_{n+2}) \\
  & + & (-1)^{n+1} f(g_1 , \dots , g_i g_{i+1} , \dots , g_{n+1}) \} \\
  % i = n
  & + & \{ g_1 f(g_2 , \dots , g_n g_{n+1} , g_{n+2}) \\
  & + & \sum_{k=1}^{n-2} (-1)^k f(g_1 , \dots , g_k g_{k+1} , \dots , g_n g_{n+1} , g_{n+2}) \\
  & + & (-1)^{n-1} f(g_1 , \dots , g_{n-1} (g_n g_{n+1}) , g_{n+2}) \\
  & + & (-1)^n f(g_1 , \dots , g_{n-1} , (g_n g_{n+1}) g_{n+2}) \\
  & + & (-1)^{n+1} f(g_1 , \dots , g_{n-1} , g_n g_{n+1}) \} \\
  % i = n+1
  & + & \{ g_1 f(g_2 , \dots , g_{n+1} g_{n+2}) \\
  & + & \sum_{k=1}^{n-1} (-1)^k f(g_1 , \dots , g_k g_{k+1} , \dots , g_n , g_{n+1} g_{n+2}) \\
  & + & (-1)^n f(g_1 , \dots , g_{n-1} , g_n (g_{n+1} g_{n+2})) \\
  & + & (-1)^{n+1} f(g_1 , \dots , g_n) \} \\
  % (-1)^{n+2} f'(g_1 , \dots , g_{n+1})
  & + & \{ (-1)^{n+2} (g_1 f(g_2 , \dots , g_{n+1}) \\
  & + & \sum_{i=1}^n (-1)^i f(g_1 , \dots , g_i g_{i+1} , \dots , g_{n+1}) \\
  & + & (-1)^{n+1} f(g_1 , \dots , g_n) ) \}
\end{eqnarray*}
\end{proof}

\begin{defi} \label{defi:cohomology}
  以下のように $ n \in \Z_{\geq 0}$に対して定める $Z^n$を\underline{$n$\rm{-th} $(次)$ コサイクル $(双対輪体)$}といい、
  $B^n$を\underline{$n$\rm{-th} $(次)$ コバウンダリー $(境界輪体)$}という。
  \begin{eqnarray*}
    Z^n = Z^n(G,M) & := & \ker(\partial^n) \\
    B^n = B^n(G,M) & := & \im(\partial^{n-1}) \\
  \end{eqnarray*}
  ただし $B^0 := 0$とする。
  このとき命題 $(\mathrm{\ref{prop:partialpartial}})$から
  $\partial^n \circ \partial^{n-1} = 0$なので $\partial^n(\im(\partial^{n-1})) = 0$より $B^n \subset Z^n$が成り立っている。
  よって剰余群 $Z^n/B^n$が定義できて
  \begin{eqnarray*}
    H^n = H^n(G,M) := Z^n(G,M)/B^n(G,M)
  \end{eqnarray*}
  を $G$の $M$係数の\underline{$n$\rm{-th} $(次)$ コホモロジー群 $(\mathrm{cohomology})$}という。
\end{defi}

\begin{exam} \label{exam:cohomology0}
  $n=0$のときのコホモロジー群を考える。
  $Z^0 = \ker(\partial^0)$であり、
  定義から
  $\partial^0 : C^0 (= M) \longrightarrow C^1 , x \longmapsto \partial^0 x$と、
  $\partial^0 x(g) = gx - x$なので
  $Z^0 = \{ gx - x = 0 \Leftrightarrow gx = x | x \in M , {}^\forall g \in G \}$となる。
  $gx$は $M$の元への $G$の作用でありそれがどんな $g \in G$でも $x$になるから
  $M$の中で $G$によって固定されるので $Z^0 = M^G$である。
  $B^0 := 0$だったのでコホモロジー群 $H^0$は $H^0 = Z^0/B^0 = M^G$である。
\end{exam}

\begin{exam} \label{exam:cohomology1}
  $n=1$のときのコホモロジー群を考える。
  $Z^1 = \ker(\partial^1)$で
  $\partial^1 : C^1 \longrightarrow C^2 , f \longmapsto \partial^1 f$となって
  $\partial^1 f(g_1 , g_2) = g_1 f(g_2) - f(g_1 g_2) + f(g_1)$となるから
  $Z^1 = \{ f \in C^1 | g_1 f(g_2) - f(g_1 g_2) + f(g_1) = 0 \Leftrightarrow f(g_1 g_2) = g_1 f(g_2) + f(g_1) , {}^\forall g_1 , g_2 \in G \}$となる。
  $B^1 = \im(\partial^0) = \{ \partial^0 x | x \in M , \partial^0 x(g) = gx - x \}$となっている。
  いま作用が $G \times M \longrightarrow M , (g , x) \longmapsto gx = x$として自明なものであるときを考えると
  $Z^1 = \{ f \in C^1 | f(g_1 g_2) = f(g_1) + f(g_2) , {}^\forall g_1 , g_2 \in G \}$でこれは $G$から $M$への群準同型なので $Z^1 = \Hom_{群}(G,M)$となる。
  $B^1 = \{ \partial^0 x | x \in M , \partial^0 x(g) = gx - x = x - x = 0 \} = 0$となるから
  $n=1$のときのコホモロジー群 $H^1$は $H^1 = \Hom_{群}(G,M)$となる。
\end{exam}

\begin{fact}
  $G$加群 $M_i (1 \leq i \leq 3)$に対して以下の加群の完全列が存在するとする。
  \begin{eqnarray*}
    0 \longrightarrow M_1 \longrightarrow M_2 \longrightarrow M_3 \longrightarrow 0
  \end{eqnarray*}
  このとき以下のような無限の長さの完全列が存在する。
  \begin{eqnarray*}
    0 & \longrightarrow & H^0(G,M_1) \longrightarrow H^0(G,M_2) \longrightarrow H^0(G,M_3) \\
    & \longrightarrow & H^1(G,M_1) \longrightarrow H^1(G,M_2) \longrightarrow H^1(G,M_3) \\
    & \longrightarrow & H^2(G,M_1) \longrightarrow \cdots
  \end{eqnarray*}
\end{fact}

\subsection{Galois cohomology}

\begin{defi}
  $A$を群 $G$が作用する\rm{Abel}とは限らない群とする。
  このとき例 $(\mathrm{\ref{exam:cohomology0}})$より
  $0$次のコホモロジー群を
  $H^0(G,A) := A^G$としても矛盾しないのでそのように定義する。

  また、 $\alpha \in C^1(G,A)$を
  \begin{eqnarray*}
    \alpha : G & \longrightarrow & A \\
    g & \longmapsto & \alpha_g
  \end{eqnarray*}
  と定めると、 $A$の演算を非可換性を表すため積で書くことにすると
  \begin{eqnarray*}
    \partial^1(\alpha)(g , h) & = & g \alpha_h \cdot \alpha_{gh}^{-1} \cdot \alpha_g \\
    \alpha \in Z^1 = \ker(\partial^1) & \Leftrightarrow & {}^\forall g , h \in G , g \alpha_h \cdot \alpha_{gh}^{-1} \cdot \alpha_g = 1 \\
    & \Leftrightarrow & \alpha_{gh}^{-1} \cdot \alpha_g = (g \alpha_h)^{-1} \\
    & \Leftrightarrow & \alpha_g (g \alpha_h) = \alpha_{gh}
  \end{eqnarray*}
  となるから
  例 $(\mathrm{\ref{exam:cohomology1}})$より
  $1$次のコサイクルは
  $Z^1 = \{ \alpha \in C^1 | {}^\forall g , h \in G , \alpha_{gh} = \alpha_g \cdot g \alpha_h \}$
  となるのでそのように定義する。
\end{defi}

\begin{defi}
  群 $G$とそれが作用する非可換群 $A$の
  $1$次コサイクル $Z^1$について
  $\alpha , \beta \in Z^1$が\underline{\rm{cohomologous}} $(\alpha \sim \beta)$とは
  \[
  {}^\exists a \in A \  \mathrm{s.t.} \
  {}^\forall g \in G \  , \  \beta_g = a^{-1} \cdot \alpha_g \cdot ga
  \]
  となることであり、これは同値関係になる。
  $G$が恒等的な作用をするのであれば $ga = a$より
  これは $\alpha_g$と $\beta_g$が共役な関係になってることと同じになる。
  つまり共役から $ga$の分だけねじれているともみれる。
\end{defi}

\begin{proof}
  同値関係になることをしめす。

  まず、 ${}^\forall g \in G$と ${}^\forall a \in A$について
  $(ga)^{-1} = ga^{-1} , g(1) = 1$が成り立つことを示す。
  定義から $G$が $A$に加群のように作用するので
  $g(1) = g(1 \cdot 1) = g(1) \cdot g(1)$から
  $g(1) = g(1) \cdot g(1)^{-1} = 1$より成立。
  これを用いれば $1 = g(1) = g(a \cdot a^{-1}) = ga \cdot ga^{-1} \Leftrightarrow (ga)^{-1} = ga^{-1}$より成立。

  ・反射律

  $a = 1 \in A$としてとれば
  $\alpha_g = 1 \cdot \alpha_g \cdot 1 = 1^{-1} \cdot \alpha_g \cdot g(1)$
  が任意の $g \in G$で成り立つので $\alpha \sim \alpha$より反射律が成り立つ。

  ・対称律

  $\alpha \sim \beta$のときある $a \in A$で $\beta_g = a^{-1} \cdot \alpha_g \cdot ga$となっているので逆元をそれぞれかけて
  $\alpha_g = a \cdot \beta_g \cdot (ga)^{-1}$となっていて
  上で述べたことより $b := a^{-1} \in A$を取る時 $(ga)^{-1} = ga^{-1} = gb$から
  $\alpha_g = b^{-1} \cdot \beta_g \cdot gb$となるので $\beta \sim \alpha$より対称律がなりたつ。

  ・推移律

  $\alpha \sim \beta , \beta \sim \gamma$となっているとするとき
  ある $a , b \in A$で $\beta_g = a^{-1} \cdot \alpha_g \cdot ga$と
  $\gamma_g = b^{-1} \cdot \beta_g \cdot gb$となっている。
  $\beta_g$に代入すると $\gamma_g = b^{-1} \cdot (a^{-1} \cdot \alpha_g \cdot ga) \cdot gb = (b^{-1} a^{-1}) \cdot \alpha_g \cdot (ga \cdot gb) = (ab)^{-1} \cdot \alpha_g \cdot g(ab)$となり
  $ab \in A$なので $\alpha \sim \gamma$から推移律が成り立つ。
\end{proof}

\begin{defi}
  \rm{Galois cohomology}とは有限次\rm{Galois}拡大 $L/K$があるとき
  $G := \gal(L/K)$としてこれが作用する群 $M$についての
  コホモロジー群 $H^n(G,M)$のことである。
  とくに $M$として $L , L^n , GL_n(L)$等を考える。
  ただし $GL_n(L)$は $L$成分の $n$次正則行列全体の積による群であり、
  一般に $L$に作用する群を $G$としたとき $\sigma \in G$は $X = (x_{ij}) \in M_n(L) := (n次正方行列全体の集合)$に対して
  $\sigma(X) := (\sigma(x_{ij}))$と定める。
\end{defi}

\begin{prop}
  体 $L$と有限群 $G \subset \aut(L)$について以下が成り立つ。

  $(1)$
  ${}^\forall n \in \Z_{\geq 1}$について
  $H^n(G,L) = 0$となる。

  $(2)$
  ${}^\forall n \in \Z_{\geq 1}$について
  $H^1(G,GL_n(L)) = 1$となる。
  とくに $H^1(G,L^\times) = 1$となる。
  これは一つの成分だけの正則行列が $GL_1(L) = L^\times$となることからすぐ導かれる。
\end{prop}

\begin{proof}
  $(2)$

  一般に定義 $(\mathrm{\ref{defi:cohomology}})$から $B^1 \subset Z^1$だから
  $Z^1 \subset B^1$を示せば $B^1 = Z^1$から $H^1 = Z^1/B^1 = 1$が示される。
  まず、 $0$次コバウンダリー作用素 $\partial^0$に対して
  $B^1 = \im(\partial^0) = \{ \partial^0 X | X \in GL_n(L) \}$となっていて
  例 $(\mathrm{\ref{exam:cohomology1}})$の $B^1$から
  $\partial^0 X$は $GL_n(L)$での演算は積であることに注意すれば
  \begin{eqnarray*}
    \partial^0 X : G & \longrightarrow & GL_n(L) \\
    g & \longmapsto & \partial^0 X(g) = gX \cdot X^{-1}
  \end{eqnarray*}
  となっている。
  したがって ${}^\forall \alpha \in Z^1$に対して
  ${}^\forall g \in G , \alpha_g = \partial^0 X(g) = gX \cdot X^{-1}$となる
  $X \in GL_n(L)$が存在すればよい。
  いま、ある $X \in GL_n(L)$について
  \begin{eqnarray*}
    b := \sum_{h \in G} \alpha_h \cdot h(X)
  \end{eqnarray*}
  と定義すると $b \in GL_n(L)$である。
  $h \in G \subset \aut(L)$より\rm{Dedekind}の補題 $(\mathrm{\ref{lemm:dedekind}})$から $M$を $L$とみれば
  その対偶を取ることで
  $\alpha_h \in GL_n(L)$はより任意の $h \in G$で $\alpha_h \neq 0$となるから
  ある $x_{ij} \in L$が存在して $\sum_{h \in G} \alpha_h \cdot hx_{ij} \neq 0$
  となる。
\end{proof}

\end{document}
