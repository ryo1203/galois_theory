\documentclass[../master_galois_theory]{subfiles}
\begin{document}

\setcounter{section}{1}

\section{Galois理論の基本定理}

\subsection{Dedekindの補題}

\begin{defi}
  有限次拡大 $L/K$が \rm{Galois}拡大であるとは $L^{\aut_K(L)} = K$であること。

  このときの $\aut_K(L)$をとくに $\gal (L/K)$と記し、 $L/K$の \galois 群という。
\end{defi}

\begin{rem}
  $L^{\aut_K(L)}$は $K$を固定するような元で固定される $L$の元であるから $L^{\aut_K(L)} \supset K$は定義より明らか。
  それ以外に固定される元が無いということ。

  また、よくある \galois 拡大の定義は正規かつ分離な拡大というものでこれとの同値は後で示す。
\end{rem}

\galois 理論の基本定理を示すために準備を行う。

\begin{lemm} \label{lemm:grpdedekind}

  $S$:群 $L$:体とし、 $\sigma_1 , \dots , \sigma_n : S \longrightarrow L^\times$を相異なる群準同型とする。
  このとき $c_1 , \cdots c_n \in L$に対し以下が成り立つ。
  \[
  c_1 \sigma_1 (x) + \cdots + c_n \sigma_n (x) = 0 \  ({}^\forall x \in S)
  \Longrightarrow c_1 = \cdots = c_n = 0
  \]

\end{lemm}

\begin{proof}
  成り立たないと仮定し、ある $c_1 , \dots , c_n \in S$が成り立たないとするもののうち $n$が最小であるような最短の反例であるとする。
  まずこのとき $n \leq 2$である。
  $n = 1$のとき $c_1 \sigma_1 (x) = 0$であるが $\sigma_1 (x) \in L^\times = L - \{ 0 \}$から $c_1 = 0$となるからである。

  相異なる群準同型より写像として異なるということは $\sigma_n \neq \sigma_1$より ${}^\exists x_0 \in S , \sigma_n (x_0) \neq \sigma_1 (x_0)$となる。
  $x_0 x$を入れると準同型より

  \begin{eqnarray}
  c_1 \sigma_1 (x_0) \sigma_1 (x) + \cdots + c_n \sigma_n (x_0) \sigma_n (x) = 0 \label{eq:x_0x}
  \end{eqnarray}

  となる。これと $\sigma_n (x_0)$を式にかけたものは

  \begin{eqnarray}
    c_1 \sigma_n (x_0) \sigma_1 (x) + \cdots + c_n \sigma_n (x_0) \sigma_n (x) = 0 \label{eq:sigmax_0}
  \end{eqnarray}

  となりこれを辺々ひくと $c_n \sigma_n (x_0) \sigma_n (x)$が共通であるからそこが消えて、 $\sigma_1 (x_0) - \sigma_n (x_0) \neq 0$より

  \[
  c_1 ( \sigma_1 (x_0) - \sigma_n (x_0) ) \sigma_1 (x) + \cdots + c_{n-1} ( \sigma_{n-1} (x_0) - \sigma_n (x_0) ) \sigma_{n-1} (x) = 0
  \]

  となり $c_k ( \sigma_k (x_0) - \sigma_n (x_0) )$を新しい係数と見れば左辺は少なくとも全ての項が $0$になることは無いので $c_1 , \dots , c_n$の最短性に矛盾しているから $c_1 = \cdots = c_n = 0$である。

\end{proof}

\begin{lemm} \label{lemm:dedekind}
  \rm{Dedekind}の補題

  $M , L$:体とし、 $\sigma_1 , \dots , \sigma_n : M \longrightarrow L$が相異なる体の準同型とする。
  このとき $c_1 , \dots , c_n \in L$に対し、以下が成り立つ。

  \[
  c_1 \sigma_1 (x) + \cdots + c_n \sigma_n (x) = 0 \  ({}^\forall x \in M)
  \Longrightarrow c_1 = \cdots = c_n = 0
  \]

\end{lemm}

\begin{proof}
  乗法群に制限したものは$\sigma_i |_{M^\times} : M^\times \longrightarrow L^\times$でありこれは相異なる群準同型なので補題 \ref{lemm:grpdedekind}より成立。
\end{proof}

\begin{rem}
  写像 $\Hom_体 (M,L) \longrightarrow \Hom_{加法群} (M,L)$を 体の準同型をその加法群の準同型とみるというものにする。
  また、このとき $\Hom_{加法群} (M,L)$は $(\phi_1 + \phi_2)(x) = \phi_1 (x) + \phi_2 (x) , (c \phi)(x) = c (\phi (x)) \  c \in L$とすることで $L$の加法により$L$-ベクトル空間と見れる。
  そしてこの写像でそれぞれの元は変わらず変わるのは始域と終域の演算なので単射であり像は一次独立となることを補題 \rm{\ref{lemm:dedekind}}は述べている。
\end{rem}

\begin{lemm}
  \rm{Dedekind}の補題/$K$

  $L/M , M/K$:拡大で $\sigma_1 , \dots , \sigma_n : M \longrightarrow L$を相異なる $K$上の体準同型 $(\sigma_i |_K = \mathrm{id}_K)$とする。
  このとき $c_1 , \dots , c_n \in L$に対し、以下が成り立つ。

  \[
  c_1 \sigma_1 (x) + \cdots + c_n \sigma_n (x) = 0 \  ({}^\forall x \in M)
  \Longrightarrow c_1 = \cdots = c_n = 0
  \]
\end{lemm}

\begin{proof}
  \rm{Dedekind}の補題から明らか。
\end{proof}

\begin{rem} \label{rem:dedekindrem}
  これも \rm{\ref{lemm:dedekind}}と同様に $K$上の体準同型であることも考えれば写像 $\Hom_{Kの拡大} (M,L) \longrightarrow \Hom_{K-ベクトル空間} (M,L)$が単射で像は $L$上一次独立である。
\end{rem}

\subsection{Artinの定理}

\begin{lemm} \label{lemm:2.4}
  $M/K , L/K$:体の拡大として $M/K$が有限次拡大のとき $|\Hom_{Kの拡大} (M,L)|$は有限で
  $|\Hom_{Kの拡大} (M,L)| \leq [M:K]$が成り立つ。
\end{lemm}

\begin{proof}
  以下では $\Hom_{Kの拡大}$を $\Hom_K$と書く。

  まず、 $\Hom_K (M,K) \otimes_K L \cong \Hom_K (M,L)$を示す。

  $f \in \Hom_K(M,K) , l \in L$に対し $\varphi(f,l) : M \longrightarrow L , m \longmapsto f(m)l$とする。
  このときこれは $f \in \Hom_K(M,K)$から以下のように $K$線形写像であるから $\varphi(f,l) \in \Hom_K(M,L)$である。

  \begin{eqnarray*}
    \varphi (f,l) (m_1 + m_2) & = & f(m_1 + m_2)l = (f(m_1) + f(m_2))l = f(m_1)l + f(m_2)l = \varphi (m_1) + \varphi (m_2) \\
    \varphi (f,l) (km) & = & f(km)l = kf(m)l = k \varphi(m)
  \end{eqnarray*}

  そして $\phi : \Hom_K(M,K) \times L \longrightarrow \Hom_K(M,L) , (f,l) \longmapsto \phi(f,l) = \varphi(f,l)$
  とすると $\phi$は以下のように $L$-双線形写像になる。

  \begin{eqnarray*}
    \phi(f_1 + f_2 , l) (m) & = & (f_1 + f_2)(m)l = f_1(m)l + f_2(m)l = \phi(f_1,l) + \phi(f_2,l) = (\phi(f_1,l) + \phi(f_2,l))(m) \\
    \phi(f , l_1 + l_2) (m) & = & f(m)(l_1 + l_2) = f(m)l_1 + f(m)l_2 = \phi(f,l_1)(m) + \phi(f,l_2)(m) = (\phi(f,l_1) + \phi(f,l_2))(m) \\
    \phi(kf , l) (m) & = & (kf)(m)l = k(f(m))l = k \phi(f,l)(m) \\
    \phi(f,kl) (m) & = & f(m)kl = k(f(m))l = k \phi(f,l)(m)
  \end{eqnarray*}

  したがってテンソル積の普遍性から
  $\theta : \Hom_K (M,K) \otimes_K L \longrightarrow \Hom_K(M,L)$であり
  $\theta (f \otimes l) : M \longrightarrow L , m \longmapsto f(m)l$と定められたものが一意に定まる。

  今、有限次拡大であるので $M$の基底を $(m_i)$、その双対空間 $\Hom_K(M,K)$の基底つまり双対基底を $(f_i)$、 $L$の基底を $(l_j)$とできる。
  よって $z \in \Hom_K(M,K) \otimes_K L$は $z = \sum_{ij} a_ij (f_i \otimes l_j) , a_{ij} \in K$と書ける。
  そして定義から $\theta(z)(m) = \sum_{ij} a_ij (f_i (m) l_j)$となる。
  $m = m_i$とすると双対基底からクロネッカーのデルタから $f_i(m_j) = \delta_{ij}$となるので $\theta(z)(m_i) = \sum_j a_ij l_j$である。
  $\theta (z) = 0$になるとき、全ての $(m_i)$において $0$になるので $(l_j)$が基底より一次独立を考えれば ${}^\forall i , \sum_j a_ij l_j = 0 \Leftrightarrow a_ij = 0$となるから $z = 0$より $\ker (\theta) = 0$より $\theta$は単射。

  また、任意の $f \in \Hom_K(M,L)$に対して $z = \sum_i f_i \otimes f(m_i) $とおくと $\theta (z)(m) = \sum_i f_i (m) f(m_i)$から $m = m_i$とおけば双対基底より同様に $\theta (z)(m_i) = f(m_i)$であり $(m_i)$は基底なので $\theta (z) = f$となるから $\theta$は全射。

  よって $\theta$は全単射であり、 $K$-双線形写像より $\theta$は同型写像となるので
  $\Hom_K(M,K) \otimes_K L \cong \Hom_K(M,L)$が成り立つ。

  次に$\Hom_K(M,K) \otimes_K L \cong L^n$を示す。

  今 $[M:K] = n$とするとある基底を取れば $M$が $K$ベクトル空間より $M \cong K^n$とできるので $\Hom_K(M,K) \otimes_K L \cong \Hom_K(K^n,K) \otimes_K L$となる。
  また、 $\Hom_K(K^n,K)$は $M = K^n$の双対空間なので基底を移せるので $\Hom_K(K^n,K) \cong K^n$より $\Hom_K(K^n,K) \otimes_K L \cong K^n \otimes_K L$となる。

  そして $\phi : K^n \otimes_K L \longrightarrow L^n , (k_1 , \dots , k_n) \otimes l \longmapsto (k_1 l , \dots , k_n l)$とする。
  これは $(k_1 l , \dots , k_n l) = (k'_1 l' , \dots , k'_n l') \Leftrightarrow {}^\forall i , k_i l = k'_i l$であり
  $L$が体なので $l^{-1}$をかければ $k_i = k'_i$より $(k_1 , \dots , k_n) = (k'_1 , \dots , k'_n)$から $phi$は単射。
  そして、任意の $(l_1 , \dots , l_n) \in L^n$に対して $k_i = l_i l^{-1}$ととれば $\phi ((k_1 , \dots , k_n) \otimes l) = (l_1 , \dots , l_n)$より全射。
  構造も保たれるから $K^n \otimes_K L \cong L^n$となる。

  したがって同型から、 $[M:K] = n = \dim_L(L^n) = \dim_L(K^n \otimes_K L) = \dim_L(\Hom_K(M,K) \otimes_K L) = \dim_L(\Hom_K(M,L))$より
  $\dim_L(\Hom_K(M,L)) = [M:K]$となる。

  そして補題 \rm{\ref{rem:dedekindrem}}から単射で一次独立であることから $\Hom_{Kの拡大}(M,L)$は $\Hom_K(M,L)$に埋め込めるから $|\Hom_{Kの拡大}(M,L)| \leq |\Hom_K(M,L)| = [M:K]$より示された。
\end{proof}

\begin{theo}
  \rm{Artin}の定理

  $L/K$が有限次拡大のとき
  \begin{eqnarray*}
    L/Kが \mathrm{Galois}拡大 \Leftrightarrow K = L^G となる部分群 G \subset \aut(L)が存在する。
  \end{eqnarray*}
  このとき $G = \gal(L/K) , [L:K] = |G|$が成り立つ。

\end{theo}

\begin{proof}
  必要十分性を示す。

  $(\Rightarrow)$

  $G = \gal(L/K)$とすれば \galois 拡大の定義より成立。

  $(\Leftarrow)$

  $K = L^G$のとき $G$の元は $K$の元を固定するので $G \subset \aut_K(L)$であり、
  \rm{\ref{rem:hougan}}により包含関係が逆になり $L^G \supset L^{\aut_K(L)}$となる。
  $L^{\aut_K(L)}$は $K$の元で固定されるような元により固定される $L$の元なので $K$を含む。
  したがって以下のようになる。

  \[
  K = L^G \supset L^{\aut_K(L)} \supset K
  \]

  より $K = L^G = L^{\aut_K(L)} = K$から $ K = L^{\aut_K(L)}$より $L/K$は \galois 拡大。

  $L^G = L^{\aut_K(L)}$から $G = \aut_K(L)$とは言えないので以下のように示す。
  まず $[L:K] = |G|$を示す。

  補題 \rm{\ref{lemm:2.4}}から $G \subset \aut_K(L)$より
  $|G| \leq |\aut_K(L)| = |\Hom_K(L,L)| \leq [L:K]$となるので $|G| \geq [L:K]$が言えればよい。

  $|G| < [L:K]$と仮定する。

  $G = \{ \sigma_1 , \dots , \sigma_m \} , L$の $K$上の基底を $(w_1 , \dots , w_n)$とする。
  仮定より $m \leq n$なので $(n \times m)$の連立方程式系

  \begin{eqnarray*}
    \begin{cases}
        \sigma_1(w_1) x_1 + \cdots + \sigma_1(w_n) x_n  =  0 \\
        \vdots \\
        \sigma_m(w_1) x_1 + \cdots + \sigma_m(w_n) x_n  =  0
    \end{cases}
  \end{eqnarray*}

  が作られ、変数の数 $(n)$より式の数 $m$のほうが多いから非自明解が存在する。
  その解を $(c_1 , \dots , c_n) \in L^n$としそのうち $0$が一番多い最短の解を考え添字を並び替え $0$の解を後ろにまとめ、 $0$でない解 $c_i , (1 \leq i \leq r)$で連立方程式系を以下のようにできる。

  \begin{eqnarray} \label{eq:c}
    \begin{cases}
        c_1 \sigma_1(w_1) + \cdots + c_r \sigma_1(w_r) = 0 \\
        \vdots \\
        c_1 \sigma_m(w_1) + \cdots + c_r \sigma_m(w_r) = 0
    \end{cases}
  \end{eqnarray}

  まず、 \rm{\ref{lemm:grpdedekind}}のときと同様に $r \leq 2$である。
  また、 $c_r (\neq 0) \in L$で割って $c_r = 1$と置き直せる。
  そして ${}^\exists c_i \in L - K$となる。
  もし ${}^\forall c_i \in K$とすると $\sigma |_K = \rm{id}_K$より
  $c_i \sigma (w_i) = \sigma(c_i w_i)$と、準同型より
  $\sigma_1 (c_1 w_1 + \cdots + c_r w_r) = 0 \Rightarrow c_1 w_1 + \cdots c_r w_r = 0$となる。
  そして $(w_i)$は基底だから一次独立より $c_1 = \cdots = c_r = 0$となりこれは非自明解であることに矛盾する。
  よって $c_i$全てが $K$に入ることは無いから ${}^\exists c_i \in L - K$となりこれを $c_1$とおく。
  このとき $K$に入っていないから ${}^\exists \sigma \in G , \sigma(c_1) \neq c_1$が成り立つ。

  この $\sigma$を連立方程式全体に作用させると以下のようになる。

  \begin{eqnarray*}
    \begin{cases}
        \sigma(c_1) \sigma(\sigma_1(w_1)) + \cdots + \sigma(c_r) \sigma(\sigma_1(w_r)) = 0 \\
        \vdots \\
        \sigma(c_1) \sigma(\sigma_m(w_1)) + \cdots + \sigma(c_r) \sigma(\sigma_m(w_r)) = 0
    \end{cases}
  \end{eqnarray*}

  ここで $G$は有限なので $\sigma \sigma_i$は $i$を動かすことで $G$のすべての元を出し尽くすから、また添字を付け替えて方程式を並び替えて $\sigma \sigma_i$を $\sigma_i$として以下のようにして良い。

  \begin{eqnarray} \label{eq:sigmac}
    \begin{cases}
        \sigma(c_1) \sigma_1(w_1) + \cdots + \sigma(c_r) \sigma_1(w_r) = 0 \\
        \vdots \\
        \sigma(c_1) \sigma_m(w_1) + \cdots + \sigma(c_r) \sigma_m(w_r) = 0
    \end{cases}
  \end{eqnarray}

  $式(\ref{eq:c}) - 式(\ref{eq:sigmac})$とすると以下のようになる。

  \begin{eqnarray*}
    \begin{cases}
        (c_1 - \sigma(c_1)) \sigma_1(w_1) + \cdots + (c_r - \sigma(c_r)) \sigma_1(w_r) = 0 \\
        \vdots \\
        (c_1 - \sigma(c_1)) \sigma_m(w_1) + \cdots + (c_r - \sigma(c_r)) \sigma_m(w_r) = 0
    \end{cases}
  \end{eqnarray*}

  そして $c_1 - \sigma(c_1) \neq 0$と $c_r = 1$から $c_r - \sigma(c_r) = 1 - 1 = 0$より $r$の最短性に矛盾する。
  よって $|G| < [L:K]$は不適であるから $|G| \geq [L:K]$なので $|G| = [L:K]$が成り立つ。

  これより $G \subset \aut_K(L)$と一番外側の値が同じであるからその間の不等号も等号になるので $|G| = |\aut_K(L)| = [L:K]$より $G = \aut_K(L) = \gal(L/K)$も成り立つことがわかる。

\end{proof}

\begin{corl} \label{corl:galoisgeq}
  $L/K$:有限次拡大で $|\aut_K(L)| \geq [L:K]$ならば $L/K$は \galois 拡大。
\end{corl}

\begin{proof}
  $G = \aut_K(L)$とおく。
  Artinの定理から $K' = L^G$とすれば $G \subset \aut(L)$より $L/L^G$は \galois 拡大。
  したがって $[L:L^G] = |G|$となる。
  ここで $L^G$は $K$の元を固定するような元で固定される $L$の元なので $L^G \supset K$である。
  よって $L/L^G , L/K , L^G/K$はともに体の拡大であるから $[L:K] = [L:L^G][L^G:K]$が成り立ち、 $[L:L^G] = |G|$と仮定 $|G| \geq [L:K]$より
  $|G| \geq |G| [L^G:K] \Rightarrow [L^G:K] = 1$となる。
  よって $|G| = |\aut_K(L)| = [L:L^G] = [L:K]$である。

  補題 $(\rm{\ref{lemm:extensiondegree}})$より $L^G = K$となるので \galois 拡大の定義より $L/K$は \galois 拡大。
\end{proof}

\begin{rem} \label{rem:tokutyou}
  $|\aut_K(L)| \leq [L:K]$は補題 \rm{\ref{lemm:2.4}}から $M = L$とすれば $|\aut_K(L)| = |\Hom_{Kの拡大}(L,L)| \leq [L,K]$より $L/K$が有限次拡大なら \galois 拡大に限らず常に成り立つ。

  よって以下の \galois 拡大の特徴づけが言える。
  \[
  |\aut_K(L)| = [L/K] \Leftrightarrow L/K が \galois 拡大
  \]
\end{rem}

\begin{corl} \label{corl:2.7}
  $L/K$:有限次拡大のとき ${}^\forall L'/L (Lの拡大体)$で次が成り立つ。
  $L/K$:\galois $\Rightarrow \aut_K(L) (=\gal(L/K)) \xlongrightarrow{\sim} \Hom_{Kの拡大} (L,L')$
  つまり $\aut_K(L)$と $\Hom_K(L,L')$の間に同型写像が作れる。
\end{corl}

\begin{proof}
  終域がより大きいほうが写像の行き先が増え、 $L'/L$から $\aut_K(L) = \Hom_{Kの拡大}(L,L) \subset \Hom_{Kの拡大}(L,L')$である。
  そして $L/K$から $L'/K$も体の拡大であるので補題 \rm{\ref{lemm:2.4}}から
  $MをL , LをL'$とみなすことで$|\Hom_K(L,L')| \leq [L:K]$となる。
  また、 $L/K$が \galois 拡大より Artinの定理から $|\aut_K(L)| = [L:K]$なので
  $[L:K] = |\aut_K(L)| = |\Hom_K(L,L)| \leq |\Hom_K(L,L')| = [L:K]$と包含関係より $\aut_K(L) = \Hom_K(L,L) = \Hom_K(L,L')$である。
  よって $\aut_K(L)$と $\Hom_K(L,L')$の間には同型写像を作ることができる。
\end{proof}

\subsection{Galois理論の基本定理}

\begin{theo} \label{theo:galois}
  \galois 理論の基本定理

  $L/K$:有限次 \galois 拡大、 $G = \gal(L/K)$とおく。
  このとき以下が成立する。

  $(1)$
  $L/K$の任意の中間体 $M$に対し $L/M$は \galois 拡大であり、次の $1:1$対応がある。

  \begin{eqnarray*}
    \{ L/Kの中間体 \} & \overset{1:1}{\longleftrightarrow} & \{ Gの部分群 \} \\
    M & \longmapsto & \aut_M(L) = \gal(L/M) \\
    L^H & \leftlongmapsto & H
  \end{eqnarray*}

  $(2)$
  この対応で $M_i \longleftrightarrow H_i$のとき $(i = 1,2)$
  \[
  M_1 \subset M_2 \Leftrightarrow H_1 \supset H_2
  \]

  $(3)$
  $M \longleftrightarrow H$のとき ${}^\forall \sigma \in G$に対し
  \[
  \sigma(M) \longleftrightarrow \sigma H \sigma^{-1}
  \]

  $(4)$
  $M \longleftrightarrow H$のとき
  \[
  M/Kが \mathrm{Galois}拡大 \Longleftrightarrow H \lhd G (HがGの正規部分群)
  \]
  でありこのとき
  \[
  \gal(M/K) \cong G/H
  \]

\end{theo}

\begin{proof}

  ・$(1)$

  両側から写像で写して戻したときにもとに戻ることを示す。

  $H \longmapsto L^H \longmapsto \aut_{L^H}(L)$となるから $H = \aut_{L^H}(L)$を示す。
  $M = L^H$とおくと Artinの定理から $M = L^H$となる $H \subset \aut(L)$が存在しているので $L/M = L/L^H$は \galois であり、 $H = \gal(L/M) = \gal(L/L^H)$となるので $H = \aut_{L^H}(L)$が言えた。

  次に$M \longmapsto \aut_M(L) \longmapsto L^{\aut_M(L)}$となるから $M = L^{\aut_M(L)}$を示す。
  $H = \aut_M(L)$とすると $L^H \supset M$は定義より明らかでそのことから係数がより大きな範囲で取れることより $[L:L^H] \leq [L:M]$となる。

  $[L^H:K] \leq [M:K]$を示す。
  仮定より $L/K$が、Artinの定理より $L/L^H$が \galois 拡大なので \rm{Rem} $(\mathrm{\ref{rem:tokutyou}})$から $[L:K] = |G| , [L:L^H] = |H|$で
  $[L:K] = [L:L^H][L^H:K]$から $|G| = |H|[L^H:K]$となる。
  そして $H$が $G$の部分群より指数を $(G:H)$と書くこととすれば $|G| = (G:H)|H|$であるから $(G:H) = [L^H:K]$が言える。
  Lagrangeの定理から $r = (G:H)$としたとき $\phi , \varphi \in G$において同値関係 $\phi^{-1} \varphi \Leftrightarrow \phi \sim \varphi$による剰余類分割によって $G = \tau_1 H \cup \cdots \cup \tau_r H$とできる。
  ここで $\tau_i \in G$が $M$に制限されたとしても $\tau_i|_M$は相異なるといえる。
  これはもしある代表元同士、つまり同値ではない元において $\tau_i(x) = \tau_j(x) , {}^\forall x \in M$とすると自己同型写像であるから逆写像が考えられて $\tau_i^{-1}\tau_j|_M = \mathrm{id}_M$である。
  よってこの写像は $M$の元を固定するので $\tau_i^{-1}\tau_j \in H = \aut_M(L)$となる。
  これは同値関係の定義から $\tau_i \sim \tau_j$となるので同値ではない元を取ったことに矛盾する。
  したがって代表元は $M$に制限しても全て相異なる。
  このことから $M$に制限された $G$の元 $\tau|_M$は少なくとも $r = (G:H)$個あるため
  補題 $(\mathrm{\ref{lemm:2.4}})$から $r = (G:H) =[L^H:K] \leq |\Hom_{Kの拡大}(M,L)| \leq [M:K]$であるので $[L^H:K] \leq [M:K]$が示された。

  よっていま $[L^H:K] \leq [M:K] , [L:L^H] \leq [L:M]$が成り立っている。
  そして $[L:K] = [L:L^H][L^H:K] = [L:M][M:K]$から1つ目の不等式より $1/[L:L^H] \leq 1/[L:M]$となるので $[L:L^H] \geq [L:M]$も成り立つ。
  したがって $[L:L^H] = [L:M]$となる。
  $L^H \supset M$で拡大次数が等しいので補題 $(\rm{\ref{lemm:extensiondegree}})$から $L^{\aut_M(L)} = L^H = M$となる。

  よって両側から写像を送って戻したときにもとの元に戻ってくるためこの対応は $1:1$対応になっている。


  $1:1$対応より任意の中間体 $M$に対して $M = L^H$となるような $G$の部分群 $H$が存在し、それは上の議論より $H = \aut_M(L)$となる。
  したがって定義より $L/M$は \galois 拡大。
  実際はこのような$H$が存在することだけで Artinの定理から $L/M$が \galois 拡大であることがわかる。

  ・$(2)$

  双方とも定義より固定する元固定される元を考えれば明らかであるがここでは一つ一つ示していく。

  $(\Leftarrow)$

  $M_1$の任意の元 $x$をとる。
  $L/M_i$は \galois 拡大より $M_1 = L^{H_1} , M_2 = L^{H_2}$より ${}^\forall \sigma \in H_1 , \sigma (x) = x$である。
  $H_1 \supset H_2$より ${}^\forall \sigma \in H_2 \subset H_1 , \sigma (x) = (x)$となるから $x \in L^{H_2} = M_2$となるので $M_1 \subset M_2$となり成り立つ。

  $(\Rightarrow)$

  $H_2$の任意の元 $\sigma$をとる。
  $H_2 = \gal(L/M_2)$より ${}^\forall x \in M_2 , \sigma (x) = x$となり、
  $M_1 \subset M_2$より ${}^\forall x \in M_1 \subset M_2 , \sigma (x) = x$である。
  したがって $\sigma \in \gal(L/M_1) = H_1$より $H_1 \subset H_2$となり成り立つ。

  ・$(3)$

  ${}^\forall \sigma \in G$に対して $\sigma(M) \longmapsto \gal(L/\sigma(M)) , \sigma H \sigma^{-1} = \sigma \gal(L/M) \sigma^{-1}$より $1:1$対応から $\gal(L/\sigma(M)) = \sigma \gal(L/M) \sigma^{-1}$を示せばよい。

  ${}^\forall \tau \in \gal(L/M)$に対して $\sigma \tau \sigma^{-1} \in \sigma H \sigma^{-1}$であり、 $\tau|_M = \mathrm{id}_M$から
  ${}^\forall x \in M , \sigma \tau \sigma^{-1} (\sigma(x)) = \sigma \tau (x) = \sigma (x)$となる。
  よって$\sigma \tau \sigma^{-1}$は $\sigma(M)$上恒等写像になるので $\sigma \tau \sigma^{-1} \in \gal(L/\sigma(M))$より $\tau$の任意性から
  $\sigma \gal(L/M) \sigma^{-1} \subset \gal(L/\sigma(M))$である。

  また、 $g = \sigma^{-1} , N = \sigma(M)$とおく。
  このとき $\sigma^{-1} \gal(L/\sigma(M)) \sigma = g \gal(L/N) g^{-1}$となり、
  これと $\gal(L/g(N))$に対して上と全く同じことを考えれば
  $g \gal(L/N) g^{-1} \subset \gal(L/g(N))$となる。
  そして左右から $g , g^{-1}$をかけて、 $g = \sigma{-1}$から $g(N) = M$より
  $\gal(L/\sigma(M)) \subset \sigma \gal(L/M) \sigma^{-1}$である。

  以上より $\gal(L/\sigma(M)) = \sigma \gal(L/M) \sigma^{-1}$が示されたのでこの対応が成り立つ。

  ・$(4)$

  ${}^\forall \sigma \in G$に対して $(1) , (3)$より $H \lhd G \Leftrightarrow \sigma H \sigma^{-1} = H \Leftrightarrow \sigma(M) = M$であるから
  $\sigma(M) = M \Leftrightarrow M/K$が \galois 拡大を示せば良い。

  $(\Rightarrow)$

  ${}^\forall \sigma \in G , \sigma(M) = M$のとき $\sigma|_M : M \longrightarrow M$となるから $\sigma$は $M$の $K$上自己同型写像。
  これより$\pi : G \longrightarrow \aut_K(M) , \sigma \longmapsto \sigma|_M$という写像が作れてこれは $G$の元を $M$に制限しているだけなので $G$の構造を保つから群準同型写像である。
  $M \Leftrightarrow H$の対応があるから $\ker(\pi) = \{ \sigma \in G | \sigma|_M = \mathrm{id}_M \} = \aut_M(L) = H$より準同型定理から
  $G/H \cong \im(\pi) \subset \aut_K(M)$となる。
  よって $|G/H| = |\im(\pi)| \leq |\aut_K(M)|$と $(1)$の話から $|G/H| = (G:H) = [M:K]$なので $[M:K] \leq |\aut_K(M)|$となるため、系 $(\mathrm{\ref{corl:galoisgeq}})$から $M/K$は \galois 拡大である。

  そして有限次 \galois 拡大より $[M:K] = |G/H| = |\aut_K(M)|$でこれらは有限であり、
  自然な準同型$\theta : G/H \longrightarrow \aut_K(M) , \sigma H \longmapsto \sigma|_M$は
  $\ker(\theta) = \{ \sigma H \in G/H | \sigma|_M = \mathrm{id}_M \} = \{ \sigma H | \sigma \in H \} = H$となるので単射。
  したがって $\theta$は同型写像なので $\gal(M/K) \cong G/H$が示された。

  $(\Leftarrow)$

  $M/K$が \galois 拡大とすると $L/M$の拡大に対して系 $(\mathrm{\ref{corl:2.7}})$から $\aut_K(M) = \Hom_K(M,L)$となる。
  よって $\Hom_K(M,L) \subset G$より ${}^\forall \sigma (\in G) : M \longrightarrow L$は $\sigma \in \Hom_K(M,L) = \aut_K(M)$だから
  $K$上の $M$自己同型写像となるので $\sigma(M) = M$となる。

\end{proof}

\clearpage


\end{document}
