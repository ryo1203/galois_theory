\documentclass[../master_galois_theory]{subfiles}
\begin{document}

\setcounter{section}{5}

\section{代数拡大}

\subsection{代数的、超越的}


$K:$体、 $A:K-$代数とする。

\begin{defi}
  $x \in A$が $K$上\underline{代数的、代数的数 \  $(\mathrm{algebraic})$}とは
  \[
  {}^\exists f \  (\neq 0) \  \in K[X]:K係数多項式 \  s.t. \  f(x) = 0
  \]
  となることで
  代数的でないときこれを\underline{超越的、超越的数 \  $(\mathrm{transcendental})$}という。
\end{defi}

\begin{prop} \label{prop:yuugenzigen}
  $x \in A$に対して以下は同値

  $(1)$ $1 , x , x^2 , \cdots$が $K$上一次独立ではない

  $(2)$ $K[x]$が有限次元

  $(3)$ $x$は $K$上代数的
\end{prop}

\begin{proof}
  $3 \Rightarrow 1$

  $x$が代数的なので、ある $f = \sum_{i=0}^n a_i X^i \in K[X] \  (0 \neq a_i \in K)$
  において $f(x) = \sum_{i=0}^n a_i x^i = 0$より
  $1 , x , x^2 , \cdots$は一次独立ではない。

  $1 \Rightarrow 3$

  一次独立でないのである有限な $m$で $\sum_{i=0}^m a_i x^i = 0$となる
  全ては $0$ではない $a_i \in K$が存在するのでこれを $f = \sum_{i=0}^m a_i X^i$
  とすれば $f \in K[X] , f(x) = 0$となるため $x$は $K$上代数的である。

  $2 \Leftrightarrow 3$

  $x \in A$に対し写像 $\phi : K[X] \longrightarrow A , X \longmapsto x$
  は環準同型であり、
  ${}^\exists f \in K[X] , \ker(\phi) = (f)$
  となる。
  このとき $x:$代数的 $\Leftrightarrow$ $f \neq 0$が定義より言える。
  したがって環準同型定理より
  $\im{\phi} = K[x] \cong K[X]/(f)$となる。
  そして $K[X]/(f)$は $\deg(f) = n$以上の次数の多項式を割り算によりその次数以下にするから
  $K[X]/(f) = \{ a_0 + a_1 x + \cdots + a_{n-1} x^{n-1} | a_i \in K \}$
  で表せるので $K[x]$も同型より有限次元である。

  とくに $1 , x , \cdots , x^{n-1}$は $n-1$次以下の $K[x]$の元が一次結合で
  表わせ、一次独立であるから
  $K$上の $K[x]$における基底となる。

\end{proof}

\begin{defi}
  $x$が $K$上代数的数のとき $f(x) = 0$となる $f (\neq 0) \in K[X]$
  のうち次数が最小で \rm{monic} $(最高次の係数が1)$であるものを
  $x$の $K$における\underline{最小多項式 \  $(\mathrm{minimal \  polynomial})$}
  という。
  $\deg(f)$を $x$の次数ともいう。

  $f \in K[X]$に対して $f = gh \Rightarrow f = g または f = h$となるとき
  $f$を既約多項式という。
\end{defi}

\begin{exam}
  $a \in \mathbb{Q}$で平方数でないものにおいて
  $\sqrt{a} \in \mathbb{C}$の $\mathbb{Q}$の
  最小多項式は $X^2 - a \in \mathbb{Q}[X]$である。

  $e , \pi$は $\mathbb{Q}$上超越的である。
\end{exam}

\begin{defi}
  $K:$可換環、 $A:K-alg$のとき
  $x \in A$が $K$上\underline{整 \  $(\mathrm{integral})$}とは
  \[
  {}^\exists f \  (\neq 0) \  \in K[X]:K係数 \mathrm{monic}多項式 \  s.t. \  f(x) = 0
  \]
  となること。
\end{defi}

\begin{exam}
  $\sqrt{2} , 1/\sqrt{2}$は $X^2 - 2 , X^2 - 1/2$を考えれば $\mathbb{Q}$上整。

  しかし、 $1/\sqrt{2}$は $\mathbb{Z}$上で代数的であるが
  $2X^2 - 1 \in \mathbb{Z}[X]$の根で \rm{monic}にならないので
  $\mathbb{Z}$上整ではない。
\end{exam}

\begin{prop} \label{prop:6.1}
  $K:$体、 $A:K-alg$で $x \in A$が代数的、その最小多項式を $f \in K[X]$とする。

  このとき以下が成立。

  $(1)$
  $g \in K[X]$について $g(x) = 0 \Leftrightarrow f | g$

  $(2)$
  $K[X]/(f) \xlongrightarrow{\sim} K[x] , X ( \mod f) \longmapsto x$
  とできてとくに
  $1 , x , \cdots , x^{n-1}$は $K[x]$の基底 $(n = \deg f)$

  $(3)$
  $x \in A^\times \Leftrightarrow f(0) \neq 0$
  でありこのとき $x^{-1} \in K[x]$
\end{prop}

\begin{proof}
  $(1)$

  \rm{Euclid}の割り算から $g = q \cdot f + r$となる
  $q , r \in K[X] , \deg r < \deg f$がある。
  $g(x) = 0$より $q(x)f(x) + r(x) = r(x) = 0$となるが
  $\deg f$の最小性から $r = 0$であるので $g = q \cdot f$
  となるため $f | g$である。

  逆は $f | g \Rightarrow g = f \cdot (xの多項式)$で $f(x) = 0$より従う。

  $(2)$

  命題 $(\mathrm{\ref{prop:yuugenzigen}})$の $(2)$より従う。

  $(3)$

  $\Rightarrow$

  $f = X^n + a_{n-1} X^{n-2} + \cdots + a_1 X + a_0$とする。
  $x \in A^\times$より $f(x) = 0$から
  \begin{eqnarray*}
    - \frac{a_0}{x} & = & -(x^{n-1} + a_{n-1}x^{n-2} + \cdots + a_1)
  \end{eqnarray*}
  であり $\deg f$の最小性からこの右辺は $\neq 0$なので
  $- a_0 / x \neq 0 \Rightarrow a_0 \neq 0$より
  $f(0) = a_0 \neq 0$となる。

  $\Leftarrow$

  $f(0) = a_0 \neq 0$とすると $a_0 \in K^\times$より
  \begin{eqnarray*}
    1 = x \cdot \frac{-(x^{n-1} + a_{n-1} x^{n-2} + \cdots + a_1)}{a_0}
  \end{eqnarray*}
  となりこの $-(x^{n-1} + a_{n-1} x^{n-2} + \cdots + a_1)/a_0$は
  $K[x]$の元であり $x$の逆元 $x^{-1}$になるので $x \in A^\times$と
  $x^{-1} \in K[x]$が言えた。
\end{proof}

\subsection{代数拡大}

\begin{defi}
  体の拡大 $L/K$が\underline{代数的 \  $(\mathrm{algebraic})$}とは
  ${}^\forall x \in L$が $K$上代数的であること。

  \underline{超越的 \  $(\mathrm{transcendental})$}とは
  代数的でないこと
\end{defi}

\begin{rem} \label{rem:6.4}
  $L/K:$有限次拡大 $\Rightarrow$ $L$が代数的
\end{rem}

\begin{proof}
  ${}^\forall x \in L$に対して $1 , x , x^2 , \cdots , x^n , \cdots$
  を考えると $[L:K]$が有限よりこれは $K$上一次独立でないから
  ある有限な $n$で $x^n + a_{n-1}x^{n-1} + \cdots + a_1 x + a_0 = 0$
  となるような全てが $0$ではない $a_0 , \dots , a_{n-1} \in K$が存在する。
  よって $f = X^n + a_{n-1}X^{n-1} + \cdots + a_1 X + a_0$とすれば
  これは $x$を根にもつ $f \in K[X]$より $x$は代数的でしたがって $L$は代数的。
\end{proof}

一般に逆は成り立たない。

\begin{exam}
  $\mathbb{Q}(\sqrt{2} , \sqrt{3} , \cdots) / \mathbb{Q}$は代数的だが
  有限次ではない。
\end{exam}

\begin{fact}
  後に示す $x \in K$の最小多項式 $f$に対して $[K(x):K] = \deg_K f$を認めれば
  $[\mathbb{Q}(\sqrt{p_1} , \cdots , \sqrt{p_n}):\mathbb{Q}] = 2^n$が示される。
  上記の例ではこれを用いれば有限次ではないことがわかる。
\end{fact}

\begin{lemm} \label{lemm:6.3}
  $A:K-alg$で整域とする。
  このとき
  $x \in A$が $K$上代数的ならば $x$は $K[x]$で可逆。
\end{lemm}

\begin{proof}
  $x$の最小多項式を $f$とすると
  命題 $(\mathrm{\ref{prop:6.1}})$の $(2)$より
  $K[x] \xlongrightarrow{\sim} K[X]/(f)$である。
  $x \in A$より $K[x] \subset A$より $K[x]$も整域だから
  $K[X]/(f)$も整域。
  したがって $(f)$は素イデアルなので $f$は既約多項式より $f(0) \neq 0$である。
  これより命題 $(\mathrm{\ref{prop:6.1}})$の $(3)$から
  $x \in A^\times , x^{-1} \in K[x]$となる。
\end{proof}

\begin{prop} \label{prop:6.6}
  $L/K$において次は同値。

  $(1)$
  $L/K$は代数的

  $(2)$
  $L/K$の任意の部分 $K-alg$は体。
\end{prop}

\begin{proof}
  $(1) \Rightarrow (2)$

  任意の部分 $K-alg , A$をとる。
  これは $A \subset L$より整域であるので補題 $(\mathrm{\ref{lemm:6.3}})$より
  ${}^\forall x \in A \subset L$に対して $L/K$が代数的で
  $x$が代数的なので $x$は $K[x] \subset A$で可逆。
  したがって $A$は体。

  $(2) \Rightarrow (1)$

  $L$の任意の元 $x$をとる。
  このとき $K[x]$は $K-alg$より仮定から体なので $x^{-1} \in K[x]$をもつ。
  よってある $n$次多項式で $x^{-1} = a_n x^n + \cdots + a_1 x + a_0$と書ける。
  $1 = x \cdot x^{-1} = a_n x^{n+1} + \cdots a_0 x$で $a_n \in K^\times$より
  $f = X^{n-1} + \cdots a_0 / a_n X - 1 / a_n$とすればこれは
  $f \in K[X]$で $f(x) = 0$となるから $x$は $K$上代数的。
  よって $L/K$は代数的。
\end{proof}

\begin{prop} \label{prop:6.7}
  $L/K$において $x \in L$が $K$上代数的ならば
  その最小多項式を $f$として
  $K[x] = K(x) \cong K[X]/(f)$であり、
  $[K(x):K] = \deg_K (x)$
  となる。
\end{prop}

\begin{proof}
  命題 $(\mathrm{\ref{prop:6.6}})$と $(\mathrm{\ref{prop:6.1}})$より
  $K[x]$は体であり
  $K[x] \cong K[X]/(f)$で $\dim_K K[x] = n = \deg f$が成り立つ。
  よって体 $K(x) = \{ q(x) | q(X) \in K[X] \}$の定義より
  $K(x) = K[x]$となる。
  そして $\dim_K K[x] = \dim_K K(x) = [K(x):K] = n = \deg_K f$である。
\end{proof}

\begin{corl} \label{corl:6.8}
  $L/K:$有限次拡大は
  $L = K(a_1 , \dots , a_r) , (a_i \in L)$の形で
  $K \subset K(a_1) \subset \cdots \subset K(a_1 , \dots , a_r) = L$
  と体の拡大の列ができる。

  $a_i$の $K(a_1 , \dots , a_{i-1})$上の拡大次数を $n_i$とし
  最小多項式を $f_i \in K(a_1 , \dots , a_{i-1})[X]$とすると
  $[L:K] = n_1 \cdots n_r$で
  $\{ a_1^{\nu_1} \cdots a_r^{\nu_r} | 0 \leq \nu_i \leq n_i \}$は
  $L$の $K$上の基底となる。

  \begin{eqnarray*}
    L \cong \left( \left( \left( \frac{K[X_1]}{(f_1)} \right) [X_2]/(f_2) \right) \cdots \right) [X_r]/(f_r)
  \end{eqnarray*}
  が成り立つ。
\end{corl}

\begin{proof}
  命題 $(\mathrm{\ref{prop:6.7}})$を繰り返し用いれば良い。
\end{proof}

\begin{lemm} \label{lemm:daisuuwa}
  $K$上代数的数 $x , y$に関して、 $x + y , xy , x - y$も代数的であり、
  $y$が $0$で無いのなら $x/y$も代数的である。
\end{lemm}

\begin{proof}
  $x + y , xy , x - y , xy \in K(x,y)$であり、
  $x,y$の最小多項式をそれぞれ $f , g$とするとともに有限次。
  したがって $K(x,y) = K(x)(y)$は拡大次数が最小多項式の次数と等しいことから
  有限次拡大である。
  よって $K$上の代数拡大であるのでそこに含まれる元は $K$上代数的。
\end{proof}

\begin{prop} \label{prop:6.9}
  $L/M/K$を拡大の列とするとき以下が成り立つ。
  \[
  L/Kが代数的 \Leftrightarrow L/M , M/Kがともに代数的
  \]
\end{prop}

\begin{proof}
  $(\Rightarrow)$は $M \supset K$より明らか。

  $(\Leftarrow)$

  ${}^\forall x \in L$が $K$上代数的であることを示す。
  $x$は $M$上代数的なので ${}^\exists f = \sum_{i=0}^n a_i X^i \in M[X] , f(x) = 0$となる。
  また、 $a_i \in M$よりこれは $K$上代数的であるので
  命題 $(\mathrm{\ref{prop:6.7}})$で $L$を $M$と、 $x$を $a_i$とみれば
  $K' = K[a_0 , \dots , a_n]$は体で $K(a_0 , \dots , a_n)$と等しい。
  したがって$K$の有限次拡大であり $f \in K'[X]$で $x$は $K'$上代数的である。
  同様に命題 $(\mathrm{\ref{prop:6.7}})$から $K'[x] \cong K'[X]/(f)$となる。
  ここでこの右辺は命題 $(\mathrm{\ref{prop:6.1}})$の $(2)$から
  $\dim_{K'} K'[X]/(f) = n$なので左辺は $K'$上有限次拡大。
  そして $K'$は $K$上有限次拡大であったので $K'[x]$は $K$上有限次拡大。
  したがって $\mathrm{Rem} (\mathrm{\ref{rem:6.4}})$より
  $K'[x]/K = K[a_1 , \dots , a_n , x]/K$は代数拡大なので
  $x \in K'[x] \subset L$は $K$上代数的。
\end{proof}

\begin{prop}
  $M_1/K , M_2/K:$代数拡大 $\Rightarrow$ 任意の合成拡大 $(L,u_1,u_2)$は $K$上代数的
\end{prop}

\begin{proof}
  ${}^\forall x \in M_1$は $K$上代数的より
  最小多項式 $f = \sum_{i=0}^n a_i X^i , f(x) = 0$が存在する。
  そして $u_1$は $K-$準同型より $0 = u_1(f(x)) = u_1(\sum_{i=0}^n a_i x^i) = \sum_{i=0}^n u_1(a_i) u_1(x)^i = \sum_{i=0}^n a_i u_1(x)^i = f(u_1(x))$
  となるから $u_1(x)$は $K$上代数的になる。
  $u_2$も同様に考えると $u_1(M_1) , u_2(M_2)$は $K$上代数的である。
  補題 $(\mathrm{\ref{lemm:daisuuwa}})$よりこの集合間の四則演算は全て代数的なので
  $L = K(u_1(M_1) , u_2(M_2))$は代数的である。
\end{proof}

\begin{prop}
  $M_1/K , M_2/K:$
\end{prop}

\begin{defi}
  $L/K:$拡大とする。

  $K$の $L$の中での\underline{相対的代数閉包 \  $(\mathrm{relative \  algebraic \  closure})$} $M$とは
  \[
  M := \{ x \in L | x は K上代数的\}
  \]
  となるもの。
  これを $\overline{K}$と書くこともある。

  また、 $K$が\underline{ $L$の中で $(相対的に)$閉じている}とは
  $K = M$となること。
\end{defi}

\begin{prop}
  上の定義における相対的代数閉包 $M$は体。
\end{prop}

\begin{proof}
  補題 $(\mathrm{\ref{lemm:daisuuwa}})$より和と積について $M$は閉じている。

  $K(x) \subset M$であり、 $K(x)$は $x$を含む最小の $L$の部分体より $x^{-1} , -x \in K(x) \subset M$なので逆元も存在する。
\end{proof}

\begin{exam}
  $K$の $K(X) \  (Xは変数)$の中での相対的代数閉包は $X$は変数なので
  それが含まれると $K$上代数的でなくなるため $K$である。

  $\mathbb{R}$の $\mathbb{C}$の中での相対的代数閉包は $\mathbb{C}$と一致するが
  $\mathbb{Q}$の $\mathbb{C}$の中での相対的代数閉包 $\overline{\mathbb{Q}}$は
  $\mathbb{C}$と一致しない。
\end{exam}

\clearpage

\end{document}
