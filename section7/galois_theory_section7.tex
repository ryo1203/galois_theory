\documentclass[../master_galois_theory]{subfiles}
\begin{document}

\setcounter{section}{6}

\section{代数閉体、分解体、代数閉包}

\subsection{代数閉体}

\begin{prop} \label{prop7.1}
  体 $K$について次は同値。

  $(\mathrm{AC}1)$
    ${}^\forall f \in K[X] - K$は $K[X]$において一次の積に分解できる。

  $(\mathrm{AC}2)$
    ${}^\forall f \in K[X] - K$は $K$において少なくとも一つの根を持つ。

  $(\mathrm{AC}3)$
    任意の $K[X]$の既約多項式は一次。

  $(\mathrm{AC}4)$
    $K$の代数拡大は $K$のみ。
\end{prop}

\begin{proof}

  $(1) \Rightarrow (2)$

  一次の積に分解できればそれが根になるので明らか。

  $(2) \Rightarrow (1)$

  $f$のある根を $k \in K$とすると $f(X) = (X - k)g(X)$となる $g \in K[X]$がある。
  この $g$に対しても同様なことをして繰り返せば
  $f = (X - k_1)(X - k_2) \cdots (X - k_n)$と一次の積に分解できる。

  $(1) \Leftrightarrow (3)$

  $K$上の既約多項式はそれ以上 $K[X]$上で分解できない多項式なので全ての $f \in K[X] - K$が一次に分解できるので既約多項式は一次。
  また、分解は既約多項式まで分解できるので一次の積に分解できる。

  $(3) \Rightarrow (4)$

  任意の代数拡大 $L/K$をとると
  ${}^\forall x \in L$に対し最小多項式 $f \in K[X]$がある。
  これは既約多項式なので $(3)$より $f(x) = x - k , k \in K$となっているから
  $x = k \in K$より $L = K$なので代数拡大は $K$のみ。

  $(4) \Rightarrow (1)$

  任意の $f \in K[X] - K$における任意の既約成分を $g$とする。
  $g$のある一つの根を $x$とするとこの元は $K$上代数的であるから
  $[K(x):K] = \deg_K g$で有限次拡大なので
  $K(x) \cong K[X]/(g)$は $K$上の代数拡大。
  $(4)$よりこれは $K$なので $\deg_K g = \dim_K (K[X]/(g)) = \dim_K K = 1$
  だから $\deg_K g = 1$より一次式になる。
  よって任意の既約成分が一次式になるので $f = (一次の積)$となる。
\end{proof}

\begin{defi}
  体$K$が上記の命題 $(\mathrm{\ref{prop7.1}})$の $(\mathrm{AC}1) \sim (\mathrm{AC}4)$を成り立たせる、つまり全てを満たすとき
  $K$を\underline{代数閉体 \  $(\mathrm{algebraically \  closed})$}という。

  相対的代数閉包とはことなり $K$を含む上の体が最初からは無い。
\end{defi}

\begin{exam}
  代数学の基本定理は$\mathbb{C}$が代数閉体であることを述べている。
\end{exam}

\begin{prop}
  $\Omega/K:$拡大、 $\Omega :$代数閉体とする。
  このとき $K$の $\Omega$の中での相対的代数閉包 $\overline{K}$は代数閉体。
\end{prop}

\begin{proof}
  $\overline{K}$が $(\mathrm{AC}2)$を満たすことを示す。

  ${}^\forall f = \sum_{i=0}^n a_i X^i \in \barK[X] - \barK \subset \Omega[X] - \Omega$は
  $\Omega$が代数閉体よりある根 $x \in \Omega$が存在する。
  $a_i \in \barK$より $K$上代数的だからそれぞれの最小多項式の次数を考えれば
  $K' = K(a_0 , \dots , a_n)$は $K$上有限次拡大。
  $x$は $K'$上代数的より
  $K'(x)$は $K'$上有限次拡大。
  この有限次拡大を合わせれば $K(a_0 , \dots , a_n)(x)/K = K(a_0 , \dots , a_n , x)/K$は有限次拡大なので代数拡大。
  よって $x$は $K$上代数的なので $x \in \barK$より $\barK$に少なくとも一つの根を持っている。
\end{proof}

\begin{theo}
  \rm{Steinitz}の定理

  $L/K , \Omega/K:$拡大とし、 $L$は代数拡大、 $\Omega$は代数閉体とする。
  このときある $K$の拡大の準同型写像 $\varphi : L \longrightarrow \Omega$が存在する。
  $(任意の代数拡大は \Omega に埋め込める)$
\end{theo}

\begin{proof}
  系 $(\mathrm{\ref{corl:scholium}})$から $\Omega' := (L \otimes_K \Omega)/\gm$は $L , \Omega$の拡大体である。

  この拡大の準同型を $\phi : L \longrightarrow \Omega' , \psi : \Omega \longrightarrow \Omega'$とするとれは体の準同型より単射。
  単射準同型なのでそれぞれの像において構造を保つことを考えれば
  $\phi(L)$は $\phi(K)$上代数拡大、 $\psi(\Omega)$は代数閉体。
  よって $\psi(K)$上代数的な元を $\psi(\Omega)$はすべて含む。
  また写像が可換より $\psi(K) = \phi(K)$なので
  $\psi(\Omega)$は $\phi(K)$上代数的な元をすべて含む。
  よって $\phi(L) \subset \psi(\Omega)$で単射なので
  $\psi^{-1}\phi : L \longrightarrow \Omega$となる体の準同型をつくれる。
\end{proof}

\subsection{分解体}

\begin{defi}
  $K:$体で $(f_i)_{i \in I}:$多項式の族 $(f_i \in K[X] - K)$に対し、
  $K$の拡大体 $L$が $(f_i)_{i \in I}$の
  \underline{最小分解体 \  $(\mathrm{minimal \  spilitting \  field})$}
  $(もしくはここでは \mathrm{MS}体)$とは以下の条件を満たすものである。

  $(1)$
  ${}^\forall i \in I , f_i$は $L[X]$で一次の積に分解される。
  $(ここではこの条件が成り立つものを分解体という)$

  $(2)$
  $L = K({}^\forall i \in I , {}^\forall f_i の根)$となる、
  つまり $f_i$の根で $K$上生成される最小の体であること。
\end{defi}

\begin{rem}
  $(f_i) = (f_1 , \dots , f_n)$のように有限個の多項式の場合、
  $f = f_1 \cdots f_n$とすると
  $(f_i)$の \rm{MS}体 $=$ $f$の \rm{MS}体である。
\end{rem}

\begin{prop}
  $K$上の多項式の族 ${}^\forall (f_i)_{i \in I}$に対しその \rm{MS}体は存在し、
  それは $K$上の同型を除き一意的である。
\end{prop}

\begin{proof}
  $(f_i)_{i \in I} = (f)$のときを考える。
  体上の多項式なので最高次の係数を $1$にしても一般性を失わない。
  $f = X^n + \sum_{i = 1}^n a_i X^{n-i} , a_i \in K$とおき、
  $A_i := K[X_1 , \dots , X_n]/ I$を考えるとこれは $K-alg$である。
  ただしここで $I$とは $K[X_1 , \dots , X_n]$において
  $s_k - (-1)^k a_k \  (k = 1 , \dots , n)$で生成されるイデアルとする。
  $s_k$は $X_1 , \dots , X_n$の $k$次基本対称式でありつまり
  $(X - X_1) \cdots (X - X_n) = X^n - s_1 X^{n-1} + \cdots + (-1)^n s_n$
  を満たすものである。
  $x_j := X_j \mod I = X_j + I$とおき、 $x_j$の $k$次基本対称式を上と同様に $t_k$とするとき
  $t_k = s_k + I$であり、
  $s_k - (-1)^k a_k \in I \Leftrightarrow s_k + I = (-1)^k a_k + I$
  なので
  \begin{eqnarray*}
    (X - x_1) \cdots (X - x_n) & = & X^n - t_1 X^{n-1} + \cdots + (-1)^n t_n \\
    & = & X^n + \sum_{i=1}^n (-1)^i t_i X^{n-i} \\
    & = & X^n + \sum_{i=1}^n (-1)^i (s_i + I) X^{n-i} \\
    & = & X^n + \sum_{i=1}^n (-1)^i ((-1)^k a_k + I) X^{n-i} \\
    & = & X^n + \sum_{i=1}^n (a_k + I) X^{n-i} \\
    & = & X^n + (a_1 + I) X^{n-1} + \cdots + (a_n + I)
  \end{eqnarray*}
  となる。
  これより $f$は $A_i[X]$において $f = (X - x_1) \cdots (X - x_n)$と分解される。
  $A_i$の任意の極大イデアル $\gm$に対し $L_i = A_i/\gm$とおくとこれは体で
  標準的射像を考えれば$f$は $L[X]$上で一次の積に分解される。
  $f$の根が $x_1 , \dots , x_n (= X_1 , \dots , X_n \mod I)$より
  $A_i$の作り方からこの $L_i$は $(f)$の\rm{MS}体である。

  一般の $(f_i)_{i \in I}$に対しては
  $L := (\otimes_{i \in I} A_i) / (極大イデアル)$とするとこれは体で
  $(f_i)$の \rm{MS}体なので存在性が示された。

  一意性は定義 $(2)$より従う。
\end{proof}

\subsection{代数閉包}


\end{document}
