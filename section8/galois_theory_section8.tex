\documentclass[../master_galois_theory]{subfiles}
\begin{document}

\setcounter{section}{7}

\section{etale代数}

\subsection{対角化}

以下ではとくに述べない限り $K$を可換体とする。

\begin{theo} \label{theo:8.1}
  $A:K-alg$と $L/K:$拡大としたときに
  集合 $\mathscr{H} := \Hom_{K-alg}(A,L)$は
  $L-$ベクトル空間 $\Hom_{K-vect.sp}(A,L)$
  の中で $L$上一次独立。
\end{theo}

\begin{proof}
  $A$を $K-vect.sp$として見ればこれは加法群であるので
  \rm{Dedekind}の補題から従う。
\end{proof}

\begin{lemm} \label{lemm:dimension}
  $\dim_L (\Hom_{K-vect.sp}(A,L)) = [\Hom_{K-vect.sp}(A,L):L] = [A:K]$が成り立つ。
\end{lemm}

\begin{proof}
  $A_{(L)} := L \otimes_K A$としてその双対空間を $(A_{(L)})^* := \Hom_L(A_{(L)},L)$とする。
  以下簡単のため $\Hom_{K-vect.sp}(A,L)$を $\Hom(A,L)$と書く。
  $\overline{・} : (A_{(L)})^* \longrightarrow \Hom(A,L) , u \longmapsto \overline{u}$で
  $\overline{u} : A \longrightarrow L , x \longmapsto \overline{u}(x) = u(1 \otimes x)$とすればこの $\overline{・}$は同型であり双対空間であることから
  $\dim_L A_{(L)} = \dim_L (A_{(L)})^* = \dim_L \Hom(A,L)$である。
  $\dim_L A_{(L)} = \dim_K A$より従う。
\end{proof}

\begin{corl} \label{corl:8.2}
  上の状況において$h(L) (= h_A(L)) := |\Hom_{K-alg}(A,L)| \leq [A:K]$が成り立つ。
\end{corl}

\begin{proof}
  $\Hom_{K-alg}(A,L)$は $\Hom_{K-vect.sp}(A,L)$で一次独立より
  $h(L) \leq \dim_L (\Hom_{K-vect.sp}(A,L))$である。
  補題 $(\mathrm{\ref{lemm:dimension}})$の $\dim_L (\Hom_{K-vect.sp}(A,L)) = [A:K]$より従う。
\end{proof}

\begin{defi}
  $K-alg$の $A$が\underline{対角化可能 \  $(\mathrm{diagonalizable})$}とは
  ${}^\exists n \geq 1 , A \cong K^n$であること。
  とくに $n = [A:K]$である。
  $K^n$は成分ごとの演算を行う直積代数である。
\end{defi}

\begin{proof}
  $n = [A:K]$であることは $A$を $K-$ベクトル空間と見ることからわかる。
\end{proof}

\begin{defi}
  $A$が拡大 $L/K$により\underline{対角化される \  $(\mathrm{diagonaled \  by \  L})$}とは
  $L-alg$の $L \otimes_K A$が対角化可能であること。
\end{defi}

\begin{defi}
  $A$が $K$上\underline{\rm{etale}}とは
  ${}^\exists 拡大 L/K$により対角化されること。
\end{defi}

\begin{rem} \label{rem:kitei}
  $(e_1 , \dots , e_n)$が $K^n (\cong A)$の標準基底とすると成分ごとの演算を行うから
  $e_i^2 = e_i , e_i e_j = 0 (i \neq j) , e_1 + \cdots + e_n = 1_A$となる。
\end{rem}

\begin{prop} \label{prop:8.3}
  有限次 $K-alg \  A$について次は同値
  $(n = [A:K]とする)$

  $(1)$
  $A$は対角化可能。

  $(2)$
  $A$の $K$上の基底 $(e_1 , \dots , e_n)$で
  $e_i^2 = e_i , e_i e_j = 0 (i \neq j)$
  を満たすものが存在する。

  $(3)$
  $\Hom_{K-alg}(A,K)$は $\Hom_{K-vect.sp}(A,K)$を生成する。
\end{prop}

\begin{proof}
  $(1) \Rightarrow (2)$は \rm{Rem} $(\mathrm{\ref{rem:kitei}})$より成立。

  $(2) \Rightarrow (1)$

  $A_i = K e_i$とすると $A_i \cong K$で
  $A = \{ k_1 e_1 + \cdots + k_n e_n | k_i \in K \} = A_1 \times \cdots \times A_n \cong K^n$
  より対角化可能。

  $(3) \Rightarrow (1)$

  有限次 $K-alg$なので $\Hom_{K-alg}(A,K) = \{ \pi_1 , \dots , \pi_n \}$とする。
  これは定理 $(\mathrm{\ref{theo:8.1}})$より一次独立で仮定から全体を張るので
  $\Hom_{K-vect.sp}(A,K)$の基底になる。
  そしてそれを並べた $K-$代数の準同型 $\pi := (\pi_1 , \dots , \pi_n) : A \longrightarrow K^n , a \longmapsto (\pi_1 (a) , \dots , \pi_n (a))$とする。
\end{proof}

\begin{corl} \label{corl:8.4}
  系 $(\mathrm{\ref{corl:8.2}})$における $|\Hom_{K-alg}(A,L)| \leq [A:K]$について
  \[
    |\Hom_{K-alg}(A,L)| = [A:K] \Leftrightarrow AはLで対角化される。
  \]
\end{corl}

\begin{proof}
  $\pi : \Hom_{K-vect.sp}(A,L) \longrightarrow \Hom_{L-vect.sp}(L \otimes_K A , L)
  , u \longmapsto \pi u$とし、
  $L-$線形写像で
  $\pi u : A_{(L)} \longrightarrow L , (1 \otimes x) \longmapsto (\pi u)(1 \otimes x) := u(x)$とする。
  $\pi$ は準同型で $\pi u = 0 \Rightarrow {}^\forall x \in A , u(x) = 0 \Leftrightarrow u = 0$で単射。
  ${}^\forall v \in \Hom_{L-vect.sp}(A_{(L)} , L) , u(x) := v(1 \otimes x)$とおけば
  $\pi u = v$となるので全射より $\pi$は同型なので
  $\dim_L \Hom_{K-vect.sp}(A,L) = \dim_L \Hom_{L-vect.sp}(L \otimes_K A,L)$が成立する。

  また、始域と終域を代数の準同型に制限して $\pi : \Hom_{K-alg}(A,L) \longrightarrow \Hom_{L-alg}(L \otimes_K A , L)$でも同様に全単射になるから
  $|\Hom_{K-alg}(A,L)| = |\Hom_{L-alg}(L \otimes_K A , L)|$である。

  命題 $(\mathrm{\ref{prop:8.3}})$の $(1) \Leftrightarrow (3)$で
  $A$を $L \otimes_K A$で置き換えて、
  補題 $(\mathrm{\ref{lemm:dimension}})$も用いれば
  \begin{eqnarray*}
      AはLで対角化される & \Leftrightarrow & L \otimes_K A は対角化可能 \\
      & \Leftrightarrow & \Hom_{L-alg}(A_{(L)} , L) は \Hom_{L-vect.sp}(A_{(L)},K)を生成する。(基底になる) \\
      & \Leftrightarrow & |\Hom_{L-alg}(A_{(L)} , L)| = \dim_L \Hom_{L-vect.sp}(A_{(L)},K) \\
      & \Leftrightarrow & |\Hom_{K-alg}(A,L)| = |\Hom_{L-alg}(A_{(L)} , L)| \\
      & \  & = \dim_L \Hom_{L-v.s}(A_{(L)},L) = \dim_L \Hom_{K-v.s}(A,L) = [A:K] \\
      & \Leftrightarrow & |\Hom_{K-alg}(A,L)| = [A:K]
  \end{eqnarray*}
\end{proof}

\begin{prop} \label{prop:8.5}
  $K-alg \  A$について次は同値。

  $(1)$
  $A$は $K$上\rm{etale}である。
  $(: \Leftrightarrow {}^\exists 拡大により対角化される)$

  $(2)$
  $A$は $K$の ${}^\exists $\underline{有限次}拡大により対角化される。

  $(3)$
  $A$は $K$の ${}^\forall $\underline{代数閉}な拡大により対角化される。

  $(4)$
  $A$は $K$の ${}^\exists $\underline{代数閉}な拡大により対角化される。
\end{prop}

\begin{proof}
  $(3) \Rightarrow (4) \Rightarrow (1)$は明らか。

  $(1) \Rightarrow (2) \Rightarrow (3)$を示す。

  $(1) \Rightarrow (2)$

  $(1) : \Leftrightarrow {}^\exists L/K$により対角化される。
  系 $(\mathrm{\ref{corl:8.4}})$から
  $|\Hom_{K-alg}(A,L)| = [A:K] = n$となる。
  $\Hom_{K-alg}(A,L) = \{ \phi_1 , \dots , \phi_n \}$とすると
  $\phi_i(A)$は $L$の部分体で対角化可能だから
  $\phi_i(A) \otimes_K A \subset L \otimes_K A \cong K^n$より
  $\phi_i(A)$は $K$上 $n$次以下。
  よって$M := (\phi_i(A)たちの合成) (\subset L)$も $K$の有限次拡大となり、
  $\im(\phi_i) \subset M$より終域を制限することができるから
  $\Hom_{K-alg}(A,M) = \{ \phi_1 , \dots , \phi_n \}$である。
  系 $(\mathrm{\ref{corl:8.4}})$より $|\Hom_{K-alg}(A,M)| = [A:K]$だから
  $A$は $K$上有限次拡大の $M$で対角化されるから $(2)$が示された。

  $(2) \Rightarrow (3)$

  $A$はある有限次拡大 $M$で対角化されるとする。
  有限次拡大より\rm{Rem} $(\ref{rem:6.4})$から $M$は代数拡大でもある。
  また、$K$の任意の代数閉体 $\Omega$をとると定理 $(\mathrm{\ref{theo:7.3}}) (\mathrm{Steinitz}の定理)$から $M$は $\Omega$に埋め込める。
  よって
  $\Hom_{K-alg}(A,M) \subset \Hom_{K-alg}(A,\Omega)$である。
  ここで対角化されるので系 $(\mathrm{\ref{corl:8.4}})$から
  $|\Hom_{K-alg}(A,M)| = [A:K]$になることと
  系 $(\mathrm{\ref{corl:8.2}})$から
  $|\Hom_{K-alg}(A,\Omega)| \leq [A:K]$より
  $|\Hom_{K-alg}(A,M)| = |\Hom_{K-alg}(A,\Omega)| = [A:K]$となる。
  よって $A$は任意の代数閉体 $\Omega$で対角化される。
\end{proof}

\subsection{\rm{etale}代数の部分代数}

以下では\rm{etale}代数$A = K^n$とし、
その標準基底を $\{ e_1 , \dots , e_n \}$とする。

\begin{prop} \label{prop:subalgebra}
  $[n] := \{ 1 , \dots , n \}$でこれを共通部分が無いように
  $[n] = I_1 \bigsqcup \cdots \bigsqcup I_r \  (I_j \neq \emptyset)$と分割する。
  $I \subset [n]$に対して
  $e_I := \sum_{i \in I} e_i$とする。
  $[n] = I_1 \bigsqcup \cdots \bigsqcup I_r$に対し、
  $A_{(I_1 , \dots , I_r)} := K e_{I_1} + \cdots + K e_{I_r}$は
  $A$の部分 $K-alg$である。

  そして $A$の任意の部分 $K-alg$は対角化可能で
  $A_{(I_1 , \dots , I_r)}$のもので尽き、
  とくに有限個である。
\end{prop}

\begin{proof}
  $e_{I_j}$が $A_{(I_1 , \dots , I_r)}$の標準基底になること。

  $A_{(I_1 , \dots , I_r)}$の定義より全体を張り、
  一次独立性も保つ。
  $e_i$は標準基底より打ち消し合って冪等元より $I_k \neq I_l$とするとき
  \begin{eqnarray*}
    e_{I_k}^2 = \left( \sum_{i \in I_k} e_i \right)^2 = \sum_{i \in I_k} e_i^2 = e_{I_k} \\
    e_{I_k} e_{I_l} = \left( \sum_{i \in I_k} e_i \right) \left( \sum_{i \in I_l} e_i \right) = 0 \\
    e_{I_1} + \cdots + e_{I_r} = \sum_{i \in [n]} e_i = 1 \\
  \end{eqnarray*}
  より標準基底になるのでそれで $K$上張られている $A_{(I_1 , \dots , I_r)}$は
  $A$の部分 $K-alg$であり、命題 $(\mathrm{\ref{prop:8.3}})$の $(2)$から
  対角化可能である。

  また、 $B$を $A$の任意の部分代数とするとき
  射影
  \begin{eqnarray*}
    v_i : A (= K^n) & \longrightarrow & K \\
    \begin{pmatrix}
      a_1 \\
      \vdots \\
      a_n
    \end{pmatrix}
    & \longmapsto & a_i
  \end{eqnarray*}
  の定義域を $B$に制限したものを考える。
  これを再度 $v_i$とおくときこれは $v_i \in \Hom_{K-alg}(B,K)$である。
  \begin{eqnarray*}
    \alpha =
    \begin{pmatrix}
      a_1 \\
      \vdots \\
      a_n
    \end{pmatrix}
    \beta =
    \begin{pmatrix}
      b_1 \\
      \vdots \\
      b_n
    \end{pmatrix}
    \in B ,
    k \in K
  \end{eqnarray*}
  とするとき
  $v_i (\alpha + \beta) = a_i + b_i = v_i (\alpha) + v_i (\beta) ,
  v_i (\alpha \beta) = a_i b_i = v_i (\alpha) v_i (\beta) ,
  v_i (k \alpha) = k a_i = k v_i (\alpha) ,
  v_i (1_{K^n}) = 1_K$
  より $v_i \in \Hom_{K-alg}(B,K)$である。
  そして定義より $v_i(e_j) = \delta_{ij}$なので
  $\{ v_1 , \dots , v_n \}$は $\Hom_{K-vect.sp}(B,K)$を生成する。
  つまり、 $f \in \Hom_{K-vect.sp}(B,K)$に対して
  $v_i (c_1 e_1 + \cdots + c_n e_n) = c_i$より
  $f = f(e_1)v_1 + \cdots + f(e_n)v_n$とすればよい。

  したがって $\Hom_{K-alg}(B,K)$の元 $v_1 , \dots , v_n$が
  $\Hom_{K-vect.sp}(B,K)$を生成するので
  命題 $(\mathrm{\ref{prop:8.3}})$から任意の部分代数は対角化可能である。
  また、基底として $(\varepsilon_1 , \dots , \varepsilon_m)$で
  $\varepsilon_i^2 = \varepsilon , \varepsilon_i \varepsilon_j = 0 (i \neq j)$となるものが存在する。

  この基底は $A$の元なので $e_1 , \dots , e_n$で作られるが
  冪等性と総和が $1$になることを考えれば $e_{I_1} , \dots , e_{I_r}$で出し尽くされる。
  したがって全ての部分代数は $A_{(I_1 , \dots , I_r)}$であり、
  $[n]$の分割を考えれば部分代数は有限個。
\end{proof}

\begin{prop} \label{prop:ideal}
  各 $I \subset [n]$に対し
  $\ga_I := \sum_{i \in I}K e_i$
  とするとこれは $A$のイデアルになる。
  そして $A$のイデアルはこれに尽き、とくに有限個である。
\end{prop}

\begin{proof}
  $\ga_I$は明らかに $A$のイデアルになる。

  $A$のイデアル $\ga$が ${}^\forall i \in I , e_i \in \ga$で
  ${}^\forall j \in J := [n] - I , e_j \notin \ga$となっているとする。
  定義より明らかに $\ga_I \subset \ga$である。
  $x = x_1 e_1 + \cdots + x_n e_n \in \ga , x_i \in K$と
  $j \in J$に対して
  $e_j \in A , x \in \ga$から $x e_j \in \ga$なので
  $x e_j = x_1 e_1 e_j + \cdots + x_n e_n e_j = x_j e_j \in \ga$となる。
  ここで $x_j = 0$のとき $x_j e_j = 0_{K^n} \in \ga$である。
  $x_j \neq 0$のとき $x_j e_j \in \ga$とすると
  $K$が体より $x_j^{-1}$が存在して、 $x_j^{-1} e_j \in A$であるから
  イデアルより $x_j^{-1} e_j x_j e_j = e_j \in \ga$となり、これは矛盾。
  したがって $x_j e_j \in \ga$であるときは $x_j = 0$である。
  これより $x \in \ga$は $\sum_{i \in I} x_i e_i$とかけるから
  $\ga \subset \ga_I$なので $\ga = \ga_I$。
  任意のイデアルはそれが含んでいる標準基底によってのみ決まるから
  $\ga_I$で全てであり $I$のとり方より有限個である。
\end{proof}

\begin{rem}
  $A = K^n$のイデアル $\ga$はそれ自身は $K-alg$の構造を持つが、
  一般に$A$の部分 $K-alg$ではない。

  また、 $\ga = \ga_I$は $A$のイデアル $\gb = \ga_J = \ga_{[n] - I}$の商 $K-alg \  A/\gb$と同型である。
\end{rem}

\begin{proof}
  $K-alg$の構造を持つことは
  $\ga = \ga_I$で $I = \{ 1 , \dots , k (\neq n) \}$とすると
  \begin{eqnarray*}
    \phi : K & \longrightarrow & \ga \\
    k & \longmapsto &
    \begin{pmatrix}
      k \\
      \vdots \\
      k \\
      0 \\
      \vdots \\
      0
    \end{pmatrix}
  \end{eqnarray*}
  とするとこれは環準同型で $K$が可換体より $\im(\phi) \subset (\ga の中心)$より
  $K-alg$になる。
  一般の $I$についても同様。

  また、$\ga$の単位元 $1_\ga$は $(\underbrace{1 , \dots , 1}_{k} , 0 , \dots , 0)$
  であり、これは $A$の単位元 $1_A = (1 , \dots , 1)$と一致しないので
  $A$の部分 $K-alg$ではない。

  $I = \{ 1 , \dots , k \}$として考える。
  このとき $J = \{ k+1 , \dots , n \}$である。
  $\psi : A \longrightarrow \ga , (a_1 , \dots , a_n) \longmapsto (a_1 , \dots , a_k)$とすると
  これは $K-alg$準同型で全射であり、 $\ker(\psi) = (a_{k+1} , \dots , a_n) = \gb$であるから準同型定理より
  $A/\gb \cong \ga$となるので示された。
\end{proof}

\begin{prop}
  \rm{etale} $K-alg \  A$は部分 $K-alg$及びイデアルを有限個しか持たない。
\end{prop}

\begin{proof}
  \rm{etale}より ${}^\exists L/K$により $L \otimes_K A \cong L^n$であるので
  命題 $(\mathrm{\ref{prop:subalgebra}})(\mathrm{\ref{prop:ideal}})$より
  $L \otimes_K A$の部分代数とイデアルは有限個。
  よって $A \subset L \otimes_K A$の部分代数とイデアルも有限個。
\end{proof}

\begin{rem}
  $\Hom_{K-alg}(A,L)$は $A$の素イデアルの集合 $\mathrm{Spec}(A)$の "$L-$有理点"の集合である。
\end{rem}

\begin{exam}
  $A := K[X,Y]/(f)$で $f(X,Y) = X^3 + 1 - Y^2$とする。
  このとき $\phi \in \Hom_{K-alg}(A,L)$を取り、
  $x = X (\mod f) , y = Y (\mod f) , \phi(x) = a , \phi(y) = b$とする。
  すると $A$で $f(X,Y) = 0_A$から $\phi(f) = f(a,b) = 0$より
  この $a,b$が $f$の $L$上の有理点になる。
  $\phi$は準同型より $x,y$の送り先 $a,b$のみで
  $\phi (g(X,Y) + f \  (\in A)) = g(a,b)$と定まるので
  $\Hom_{K-alg}(A,L) \cong \{ (a,b) \in L^2 | f(a,b) = 0 \}$
  という同型が定まる。
\end{exam}

\subsection{分離次数}

$A:$有限次 $K-alg$で ${}^\forall L/K$に対して $h_A(L) := |\Hom_{K-alg}(A,L)|$
とおく。
このとき系 $(\mathrm{\ref{corl:8.2}})$より $h(L) \leq n = [A:K]$が成り立っている。

\begin{lemm} \label{lemm:8.8}
  $\Omega/K:$拡大、 $\Omega:$代数閉体とするとき
  ${}^\forall L/K$に対し $h(L) \leq h(\Omega)$
\end{lemm}

\begin{proof}
  $L': K$の相対的代数閉包とすると
  ${}^\forall \phi \in \Hom_{K-alg}(A,L)$において
  $A$の $K$上の基底を $(e_1 , \dots , e_n)$とする。
  このとき${}^\forall x = a_1 e_1 + \cdots + a_n e_n \in A$と書けて
  $\phi(x) = a_1 \phi(e_1) + \cdots + a_n \phi(e_n)$となる。
  $\{ \phi(e_1) , \dots , \phi(e_n) \}$の部分集合が $\phi(A)$の基底になるので
  $[\phi(A):K] \leq [A:K] = n$より $\phi(A)/K$は有限次拡大より代数拡大である。
  よって $\phi(A) \subset L'$だから $\phi \in \Hom_{K-alg}(A,L')$なので
  $h(L) = h(L')$になる。
  定理 $(\mathrm{\ref{theo:7.3}})$より $L'$は代数閉体と見た $\Omega$に埋め込めるので終域が小さくなるから
  $\Hom_{K-alg}(A,L) = \Hom_{K-alg}(A,L') \subset \Hom_{K-alg}(A,\Omega)$より
  $h(L) \leq h(\Omega)$である。
\end{proof}

\begin{defi}
  $[A:K]_s := \max_{L/K} h_A(L) = h_A(\Omega) \  (\Omega:Kの代数閉包 , 補題(\mathrm{\ref{lemm:8.8}})から言える。)$

  を $A$の $K$上の\underline{分離次数 \  $(\mathrm{separable \  degree})$}という。
  系 $(\mathrm{\ref{corl:8.2}})$から $[A:K]_s \leq [A:K]$が言える。
\end{defi}

\begin{defi}
  ・$K-alg \  A$が\underline{分離的}とは
  $[A:K]_s = [A:K]$となること。

  ・とくに有限次拡大 $L/K$が\underline{分離的}とは $K-alg$として
  $L$が分離的であること。

  ・代数拡大 $L/K$が\underline{分離的}とは ${}^\forall$有限次部分体 $(中間体)$が分離的であること。
\end{defi}

\begin{prop} \label{prop:8.9}
  $A,B :$有限次 $K-alg$、 $L/K:$拡大、 $A_{(L)} = L \otimes_K A$とする。

  $(1)$
  $[A \otimes_K B:K]_s = [A:K]_s [B:K]_s$

  $(2)$
  $[A_{(L)}:L]_s = [A:K]_s$

  $(3)$
  $C:$有限次 $L-alg$で $L/K$が有限次のとき
  $[C:K]_s = [C:L]_s [L:K]_s$
\end{prop}

\begin{proof}
  $(1)$

  $\Omega$を $K$の代数閉包とする。
  定義より $[A \otimes_K B:K]_s = h_{A \otimes_K B}(\Omega) ,
  [A:K]_s = h_A(\Omega) , [B:K]_s = h_B(\Omega)$である。
  そして
  \begin{eqnarray*}
    * :  \Hom_{K-alg}(A , \Omega) \times \Hom_{K-alg}(B , \Omega)
     & \longrightarrow &  \Hom_{K-alg}(A \otimes_K B , \Omega) \\
     (v , u) & \longmapsto & v * u \\
     v * u : A \otimes_K B & \longrightarrow & \Omega \\
     a \otimes b & \longmapsto & v(a) u(b)
  \end{eqnarray*}
  と定める。
  まず、 $v' * u' = v * u$のとき ${}^\forall a \otimes b \in A \otimes_K B , v'(a) u'(b) = v(a) u(b)$で $\Omega$の元だから $a \otimes b \neq 0$で
  逆元が存在するから $v'(a) = v(a) , u'(b) = u(b)$より $(v',u') = (v,u)$なので単射。

  $u_1 : A \longrightarrow A \otimes_K B , a \longmapsto u_1(a) = a \otimes 1$と
  $u_2 : B \longrightarrow A \otimes_K B , b \longmapsto u_2(b) = 1 \otimes b$とすると
  ${}^\forall a \otimes b (\in A \otimes_K B) = u_1(a) u_2(b)$となる。
  任意の $w \in \Hom_{K-alg}(A \otimes_K B , \Omega)$に対し、
  $v_i = w \circ u_i$とすると準同型より
  $w(a \otimes b) = w(u_1(a) u_2(b)) = w \circ u_1(a) w \circ u_2(b) = v_1(a) v_2(b)$となる。
  よって $w$に対して $v_1 \in \Hom_{K-alg}(A,\Omega) , v_2 \in \Hom_{K-alg}(B,\Omega)$をとれば $w = v_1 * v_2$となるので全射。
  したがって $*$は全単射だから $h_{A \otimes_K B}(\Omega) = h_A(\Omega) h_B(\Omega)$より成立。

  $(2)$

  $\Hom_{K-alg}(A,\Omega)$と $\Hom_{L-alg}(L \otimes_K A , \Omega)$の間は
  系 $(\mathrm{\ref{corl:8.4}})$での $\pi$を用いれば
  $L$を $\Omega$として見てもよく、これは全単射であるから
  $[L \otimes_K A : L]_s = [A:K]_s$が成り立つ。

  $(3)$

  $S = \Hom_{K-alg}(L,\Omega) , T = \Hom_{K-alg}(C,\Omega)$とする。
  $\sigma \in S$に対して $T_\sigma = \{ f \in T | {}^\forall \alpha \in L , f(\alpha) = \sigma(\alpha) \}$とするとこれは以下で示されるように $T$を分割する。
  $\sigma , \tau \in S$に対して $f \in T_\sigma \cap T_\tau$としたとき
  ${}^\forall \alpha \in L , f(\alpha) = \sigma(\alpha) = \tau(\alpha)$より
  $\sigma = \tau$から $T_\sigma = T_\tau$となる。
  また、 ${}^\forall g \in T$に対して
  $\sigma := g|_L$とすれば $\sigma(\alpha) = g(\alpha)$だから
  $g \in T_\sigma$であるので確かに $T$を分割する。

  $\sigma$は体の準同型より単射だから
  $\Omega$の中に $\sigma(L)$として $L$を埋め込めるからその2つを同一視することで
  $K$の代数閉包 $\Omega$を $L$の代数閉包と見ることもできる。
  このとき ${}^\forall f \in T_\sigma$は定義より
   ${}^\forall \alpha \in L , f(\alpha) = \sigma(\alpha) \in \sigma(L) \cong L$で $\sigma(L) \cong L$上恒等写像になる。
   したがって $f \in \Hom_{L-alg}(C,\Omega)$より
   $T_\sigma \subset \Hom_{L-alg}(C,\Omega)$である。
   また、この $\sigma(L)$と $L$の同一視から
   $\alpha \in L$は $\Omega$の中で
   $\alpha = \sigma(\alpha) \in \sigma(L)$であるので
   $\phi \in \Hom_{L-alg}(C,\Omega)$に対して
   $\phi(\alpha) = \alpha = \sigma(\alpha)$より
   $\phi \in T_\sigma$だから $\Hom_{L-alg}(C,\Omega) \subset T_\sigma$である。
   これより $T_\sigma = \Hom_{L-alg}(C,\Omega)$から
   $|T_\sigma| = [C:L]_s$となるので分割であることも考えれば
   $|T| = [C:K]_s = \sum_{\sigma \in S} |T_\sigma| = |S||T_\sigma| = [C:L]_s [L:K]_s$より示された。
\end{proof}

\begin{note}
  分離次数ではなく拡大次数でも $(1) \sim (3)$と同様のことが成り立つ。
\end{note}

\begin{prop} \label{prop:8.10}
  $A :$有限次 $K-alg$について
  \[
  Aが K上分離的 \Leftrightarrow Aは K上 \mathrm{etale}
  \]
\end{prop}

\begin{proof}
  系 $(\mathrm{\ref{corl:8.4}})$と
  命題 $(\mathrm{\ref{prop:8.5}})$から
  \begin{eqnarray*}
    Aが K上分離的 & \Leftrightarrow & [A:K]_s = [A:K] \\
    & \Leftrightarrow & |\Hom_{K-alg}(A,\Omega)| = [A:K] \\
    & \Leftrightarrow & Aはある Kの代数閉包 \Omega で対角化される。 \\
    & \Leftrightarrow & Aは K上 \mathrm{etale}
  \end{eqnarray*}
\end{proof}

\begin{corl}
  次の $3$つが成り立つ

  $(1)$
  $A \otimes_K B$が $\mathrm{etale}/K$
  $\Leftrightarrow A$と $B$がともに\rm{etale}

  $(2)$
  $A/K:\mathrm{etale} \Leftrightarrow A_{(L)}/L:\mathrm{etale}$

  $(3)$
  $C/L/K$のとき $C$が $K$上\rm{etale} $\Leftrightarrow$
  $C$が $L$上\rm{etale}でかつ $L$が $K$上\rm{etale}
\end{corl}

\begin{proof}
  代数が分離的であるときその分離次数は拡大次数と等しいという定義と
  その拡大次数を常に超えないということから
  命題 $(\mathrm{\ref{prop:8.9}})(\mathrm{\ref{prop:8.10}})$と
  拡大次数について命題 $(\mathrm{\ref{prop:8.9}})$が成り立つことより
  示される。
\end{proof}

\subsection{微分加群}

\begin{defi}
  $A/K$の微分加群 $\Omega_{A/K}$とは
  $I := \ker(A \otimes_K A \longrightarrow A , a \otimes b \longmapsto ab)$としたとき
  $\Omega_{A/K} := I/I^2$と定義される。
  定義より $\Omega_{A/K} \subset B := A \otimes_K A /I^2$は明らか。
  ここで $A \otimes_K A$を $a \longmapsto a \otimes 1$と見ることで $A$加群と考える。
  そして $d : A \longrightarrow \Omega_{A/K} , a \longmapsto d(a) (=da) := 1 \otimes a - a \otimes 1 (\mod I^2)$とする。
  このとき $a , b \in A$と $k \in K$に対して
  \begin{eqnarray*}
    d(ab) & = & b d(a) + a d(b) \\
    d(k) & = & 0
  \end{eqnarray*}
  が成り立つ。
  実際、 $d(ab) = 1 \otimes ab - ab \otimes 1 = (1 \otimes a)(1 \otimes b) - (a \otimes 1)(b \otimes 1) = (1 \otimes b)(1 \otimes a - a \otimes 1) + (a \otimes 1)(1 \otimes b - b \otimes 1) + (1 \otimes b)(a \otimes 1) - (a \otimes 1)(1 \otimes b) = b d(a) + a d(b)$であり、
  $d(k) = 1 \otimes k - k \otimes 1 = k(1 \otimes 1) - k(1 \otimes 1) = 0$より成立。
\end{defi}

\begin{exam}
  $A = K[X]$とすると $\Omega_{A/K} = A \cdot dX (= d(X) = 1 \otimes X - X \otimes 1)$であり
  $d : A \longrightarrow \Omega_{A/K} , f \longmapsto f' dX$となる
\end{exam}

\begin{exam} \label{exam:8.12}
  $A = K[X]/(f)$のとき
  $\Omega_{A/K} = A/(f') dX = K[X]/(f)/(f') dX = K[X]/(f,f') dX$となる。
\end{exam}

\begin{prop} \label{prop:8.12}
  有限次 $K-alg \  A$について
  \[
  Aは K上で \mathrm{etale} \Leftrightarrow \Omega_{A/K} = 0
  \]
  が成り立つ。
\end{prop}

\begin{corl}
  $A = K[X]/(f)$で $\mathrm{etale}/K$ $\Leftrightarrow (f,f') = 1 \  (fとその形式微分 f'によって作られるイデアルが 1_Kを含む \Leftrightarrow fとf'が互いに素)$
\end{corl}

\begin{proof}
  例 $(\mathrm{\ref{exam:8.12}})$と命題 $(\mathrm{\ref{prop:8.12}})$から
  $\Omega_{A/K} = 0 \Leftrightarrow  (f,f') = K[X] = 1$より成り立つ。
\end{proof}

\subsection{被約}

\begin{defi}
  可換環 $A$が\underline{被約 \  $(\mathrm{reduced})$}とは
  $0$以外の冪零元を持たないこと。 $(\Leftrightarrow a \neq 0 なら a^2 \neq 0)$
  $K-alg \  A$が被約とはそれが可換環として被約であること。
\end{defi}

\begin{exam}
  体、整域は被約。
  被約 $\times$被約 $=$被約

  図形的には重なっていないことと見ることができる。
\end{exam}

\begin{lemm} \label{lemm:idempotent}
  可換環$A$に冪等元 $e (\neq 0 , 1)$が存在するとき
  $A = A_1 \times A_2$となる $\{ 0 \} , A$ではない
  部分環 $A_1 , A_2$が存在する。
\end{lemm}

\begin{proof}
  $e' := 1 - e$とするとこれも
  $(1 - e)(1 - e) = 1 - 2e + e^2 = 1 - e$より冪等元である。
  また、 $e e' = e (1 - e) = e - e^2 = 0$をみたす。
  $A_1 := Ae , A_2 := Ae'$とするとこれは 和と積について閉じているから
  $A$の部分環になっていて以下の準同型写像を考える。
  \begin{eqnarray*}
    \phi : A & \longrightarrow & A_1 \times A_2 \\
    x & \longmapsto & (ex , e'x)
  \end{eqnarray*}
  ここで $\phi(x) = 0 \Leftrightarrow ex = 0 \land e'x = (1 - e)x = 0 \Rightarrow (1 - e)x + ex = 0 \Rightarrow x = 0$より $\ker(\phi) = \{ 0 \}$から単射。
  また、 ${}^\forall (ea , e'b)$に対して
  $ea + e'b \in A$をとると
  $\phi(ea + e'b) = (e(ea + e'b) , e'(ea + e'b) ) = (ea , e'b)$より全射。
  したがって同型写像になるから $A \cong A_1 \times A_2 (= Ae \times Ae')$となる。
\end{proof}

\begin{lemm} \label{lemm:determinant}
  $M:$有限生成 $A$加群で $\ga$を $A$のイデアルとして
  $\phi$を $M$の $A$加群の自己準同型であり $\phi(M) \subset \ga M$を満たすとする。
  このとき $M$の生成系を $(x_1 , \dots , x_n)$として
  $\phi$はある $a_i \in \ga$により
  $\phi^n + a_1 \phi^{n-1} + \cdots + a_n = 0$となる。
\end{lemm}

\begin{proof}
  定義から ${}^\forall i , \phi(x_i) \in \ga M$から
  $\phi(x_i) = \sum_{j = 1}^n a_{ij} x_j$となる $a_{ij}$が存在するので以下の式変形で
  \begin{eqnarray*}
    \phi(x_i) - \sum_{j = 1}^n a_{ij} x_j & = & 0 \\
    \sum_{j = 1}^n \delta_{ij} \phi(x_i) - a_{ij} x_j & = & 0 \\
    \sum_{j = 1}^n (\delta_{ij} \phi - a_{ij}) x_j & = & 0
  \end{eqnarray*}
  となる。
  ここで $n$次正方行列 $A := (\delta_{ij} \phi - a_{ij})$を考えることが出来て
  これは $\vec{x} := {}^T(x_1 , \dots , x_n)$に対して
  $A \vec{x} = 0$となる。
  $\vec{x}$の元は $M$の生成系から $0$ではないのでこの連立方程式は
  非自明解を持つからその行列式
  $\det A = \det (\delta_{ij} \phi - a_{ij}) = 0$である。
  この行列式は $\phi$の $n$次式になり、
  $n$次部分は $A$の対角線上の積のみであるので $n$次の径数は $1$で
  その他の係数は $a_{ij} \in \ga$の積の和だから
  ある $a_i \in \ga$で
  $\phi^n + a_1 \phi^{n-1} + \cdots + a_n = 0$となるので示された。
\end{proof}

\begin{corl} \label{corl:determinant}
  $M$を有限生成 $A$加群として、
  $\ga$を $\ga M = M$となる $A$のイデアルとする。
  このとき $x M = 0$と $x \equiv 1 (\mod \ga)$となる $x \in A$が存在する。
\end{corl}

\begin{proof}
  補題 $(\mathrm{\ref{lemm:determinant}})$で $\phi = \mathrm{id}_M$とすると
  $\phi(M) = M = \ga M$から条件を満たしているので
  $\phi^n + a_1 \phi^{n-1} + \cdots + a_n = 1 + a_1 + \cdots + a_n = 0$となる
  $a_1 , \dots , a_n \in \ga$が存在する。
  ここで $x = 1 + a_1 + \cdots + a_n$とおくと $x \in A$であり、
  $x M = 0 M = 0$と
  $a_1 + \cdots + a_n \in \ga$から $x \equiv 1 (\mod \ga)$となる。
\end{proof}

\begin{prop}
  有限次 $K-alg \  A$において次は同値。

  $(1)$
  $A$は被約。

  $(2)$
  $A$はある有限次体拡大 $L_i/K$により
  $A \cong L_1 \times \cdots \times L_n$となる。
\end{prop}

\begin{proof}
  $(2) \Rightarrow (1)$

  体は被約であることより明らか。

  $(1) \Rightarrow (2)$

  $A$が体であれば $K$の拡大体と見ることで $L := A$で成立する。

  $A$が体でないとして $A$の次元について帰納法を用いる。
  $[A:K] = 1$のとき $A \cong K$から $K$が可換体より成立。
  $[A_i:K] = m \leq n$の $K-alg \  A_i$について題意が満たされているとする。
  $[A:K] = n$のとき
  もし $A$が冪等元 $e (\neq 0 , 1)$を持っていれば
  補題 $(\mathrm{\ref{lemm:idempotent}})$より
  $A \cong A_1 \times A_2$となる $0$でも $A$でもない部分環が存在して
  部分環であることから $[A_1:K] , [A_2:K] \leq [A:K]$となるので帰納法の仮定から
  $A$は題意を満たす。

  よって $A$が冪等元 $e (\neq 0 , 1)$を持っていることを示す。
  $\ga (\neq (0) , (1))$は $A$のイデアルで
  $K-$ベクトル空間としてみたときに次数が最小のものとする。
  $\ga \neq (0)$より $0$でない $x \in \ga$がとれる。
  $A$が被約より $x^2 \neq 0$なので $\ga^2 \neq \{ 0 \}$であり
  $\ga^2 \subset \ga$を満たす。
  そして $\ga$の次数の最小性から $\ga^2 = \ga$となる。

  $A$が有限次 $K-alg$よりそのイデアル $\ga$も有限生成であるので
  系 $(\mathrm{\ref{corl:determinant}})$において $M = \ga$としてもよく、
  $\ga$自身は $\ga \ga = \ga$を満たしているので
  $x M = 0 , x \equiv 1 (\mod \ga)$となる $x \in A$が存在する。
  $e := 1 - x$とすると $e \equiv 0 (\mod \ga)$より $e \in \ga$で
  $(1 - e) \ga = 0$となる。
  したがって ${}^\forall x \in \ga , (1 - e)x = 0 \Leftrightarrow ex = x$
  となるから $x = e \in \ga$をとると $e^2 = e$から冪等元 $e$が存在して
  $\ga = Ae$となる。
  $\ga \neq (0) , (1)$から $e \neq 0 , 1$であるので
  $A$は冪等元 $e (\neq 0 , 1)$を持っているので示された。
\end{proof}

\end{document}
