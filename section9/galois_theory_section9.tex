\documentclass[../master_galois_theory]{subfiles}
\begin{document}

\setcounter{section}{8}

\section{分離的代数拡大}

\subsection{多項式の分離性}

\begin{prop}
  代数拡大 $L/K$について次は同値。

  $(1)$
  $L/K:$分離的。

  $(2)$
  $L/K$の ${}^\forall$部分拡大 $M/K$は分離的。
\end{prop}

\begin{proof}
  定義 $(\mathrm{\ref{defi:separable}})$から明らか。
\end{proof}

\begin{prop} \label{prop:9.2}
  $f \in K[X] - K$について以下は同値。

  $(1)$
  $(f , f') = 1 \  (\Leftrightarrow f とその形式微分 f' が互いに素)$

  $(2)$
  $f$の判別式 $\mathrm{disc}(f) \neq 0 \  ( f = \prod_{i = 1}^n (X - \alpha_i) のとき \mathrm{dics}(f) := \prod_{i < j}(\alpha_i - \alpha_j)^2 と定義する)$

  $(3)$
  $K$のある拡大 $L$上で $f$は相異なる一次式の積になる。

  $(4)$
  $f$の任意の根は単根 $(重解でない)$

  $(5)$
  $K[X]/(f)$は $\mathrm{etale}/K \  (\Leftrightarrow K上分離的)$
\end{prop}

\begin{proof}
  $(5) \Leftrightarrow (1)$

  系 $(\mathrm{\ref{corl:8.12}})$で示した。

  $(2) \Leftrightarrow (3) \Leftrightarrow (4)$

  明らか。

  $(1) \Rightarrow (2) \  (\deg f > 1のときを考える)$
  対偶 $\mathrm{disc}(f) = 0 \Rightarrow (f,f') \neq 1$を示す。

  $\mathrm{dics}(f) = 0$よりある $0 \leq i < j \leq n$があり
  $\alpha_i = \alpha_j$となる。
  $i = 1 , j = 2$としても一般性を失わない。
  これは $f$の根なので $f = (X - \alpha_1)^2 Q(X)$となる $Q(X) \in K[X]$が存在する。
  よって $f' = 2(X - \alpha_1)Q(X) + (X - \alpha_1)^2 Q'(X) = (X - \alpha_1)(2 Q(X) + (X - \alpha_1)^2 Q'(X))$となるから
  $f , f'$は共通の $\alpha_1$という根を持つので互いに素でないから
  $(f,f') \neq 1$となる。

  $(2) \Rightarrow (1) \  (\deg f > 1のときを考える)$
  対偶 $(f,f') \neq 1 \Rightarrow \mathrm{disc}(f) = 0$を示す。

  $(f,f') \neq 1$よりある $\alpha$があってそれを
  $f = (X - \alpha) Q_1(X) , f' = (X - \alpha) Q_2(X)$として共通根として持つ。
  この二つから $f' = Q_1(X) + (X - \alpha)Q_1'(X) = (X - \alpha) Q_2(X)$より
  $(X - \alpha)(Q_1'(X) - Q_2(X)) = Q_1(X)$となるから
  $f = (X - \alpha)^2 (Q_1'(X) - Q_2(X))$より重根をもつ。
  したがって根の差の積である $\mathrm{disc}(f) = 0$である。

  $\deg f = 1$のときは $f$の根は $0$より常に $\mathrm{disc}(f) = 0$となるからこの命題には不適。
\end{proof}

\begin{defi}
  これらが成り立つとき $f$を\underline{分離的}という。
\end{defi}

\begin{prop} \label{prop:9.3}
  既約多項式 $f \in K[X]$について次は同値。

  $(1)$
  $f$は分離的。

  $(2)$
  $f$は $({}^\exists L に)$少なくとも一つの単根をもつ。

  $(3)$
  $f' \neq 0$

  $(4)$
  $\Char(K) = 0$か、または $\Char(K) = p > 0$で $f \notin K[X^p]$
\end{prop}

\begin{proof}
  $(1) \Rightarrow (2)$は命題 $(\mathrm{\ref{prop:9.2}})$で示した。

  $(2) \Rightarrow (3)$

  $\alpha$を $f$の単根とする。
  $f'(\alpha) = 0$とすると命題 $(\mathrm{\ref{prop:9.2}})$の $(2) \Rightarrow (1)$の証明より
  $f = (X - \alpha)^2 Q(X)$となるから $\alpha$が単根に矛盾するので $f'(\alpha) \neq 0$である。
  よって $f' \neq 0$

  $(3) \Rightarrow (1)$

  体上の多項式より $f$を\rm{monic}としてよい。
  $\alpha$を $f$の任意の根とする。
  $f$が既約多項式で\rm{monic}より $f$は最小多項式であるから
  その次数の最小性と $f' \neq 0$より
  $f'$は多項式で $f'(\alpha) \neq 0$であるから $\alpha$は単根。
  これが任意の $f$の根について成り立つから $f$は分離的。

  $(3) \Leftrightarrow (4)$

  $f = \sum_{i=0}^n a_i X^i \in K[X]$について
  \begin{eqnarray*}
    f' & = & \sum_{i=0}^n a_i i X^{i-1} = 0 \\
    & \Leftrightarrow &
     \begin{cases}
      a_1 = \cdots = a_n = 0 & (\Char(K) = 0) \\
      a_i = 0 \  (p \nmid i) & (\Char(K) = p > 0)
     \end{cases} \\
    & \Leftrightarrow &
    \begin{cases}
      f = a_0 & (\Char(K) = 0) \\
      f = \sum a_{pk} X^{pk} \in K[X^p] & (\Char(K) = p > 0)
    \end{cases}
  \end{eqnarray*}
  より、 既約多項式は$f \in K[X]-K$で否定を考えれば成立。
\end{proof}

\begin{corl}
  体 $K$について次は同値。

  $(1)$
  $K$は完全体

  $(2)$
  任意の既約多項式 $f \in K[X]$は分離的

  $((3) {}^\forall L/K:代数拡大は分離的)$
\end{corl}

\begin{proof}
  $(1) \Leftrightarrow (2)$のみ示す。

  $\Char(K) = 0$のとき命題 $(\mathrm{\ref{prop:9.3}})$の
  $(1) \Leftrightarrow (4)$から ${}^\forall$既約多項式 $f \in K[X]$は分離的。

  $\Char(K) = p > 0$のとき
  \[
  Kが完全体 \Leftrightarrow {}^\forall f \in K[X^p] - K は可約
  \]
  を示す。
  これより、既約ならば $f \notin K[X^p] - K$が言えて
  命題 $(\mathrm{\ref{prop:9.3}})$の $(4) \Leftrightarrow (1)$より
  既約ならば分離的が言える。

  $(\Rightarrow)$

  $f = \sum a_i X^{pi} \in K[X^p] - K$で
  $K^p := \{ x^p | x \in K \} \  (p乗元の集合)$とする。
  $K$が完全体なので $\mathrm{Frobenius}$が全射だから
  $K = K^p$なので ${}^\forall a_i \in K$に対して
  ${}^\exists b_i \in K , a_i = b_i^p \in K^p = K$である。
  したがって $\Char(K) = p > 0$に注意すれば
  $f = \sum b_i^p X^{pi} = (\sum b_i X^i)^p$より
  $\sum b_i X^i \in K[X]$で分解できるから $f$は可約。

  $(\Leftarrow)$
  対偶の $K:非完全 \Rightarrow {}^\exists f \in K[X^p] - K は既約$を示す。

  $K:$非完全とする。
  このとき $K^p \neq K$から ${}^\exists a \in K^\times - K^p$が取れる。
  ここで $f = X^p 0 a \in K[X]$は既約になる。

  $b$を $f$の根 $(b^p = a)$とし、
  $g$を $b$の $K$上の最小多項式とする。
  最小性から $g \mid f$で $\Char(K) = p > 0$より
  $f = (X - b)^p$となるから
  $g = (X - b)^d \  (d^e = p)$と書ける。
  $f = g^e$の形になり、 $p$が素数から
  $d = p$または $d = 1$になる。
  $d = 1$とすると $g \in K[X]$より $b \in K$であり、 $a = b^p \in K^p$から
  $a \in K^\times - K^p$に矛盾する。
  よって $d = p$で $f = g$となるから $f$は既約。
  これより既約な $f \in K[K^p] - K$が存在するので対偶が示された。
\end{proof}

\subsection{元の分離性}

\begin{defi}
  $L/K:$拡大としたとき、
  $K$上代数的な元 $x \in L$が\underline{$K$上分離的}とは
  体の拡大$K(x)/K$が分離的であること。
\end{defi}

\begin{prop} \label{prop:9.5}
  $x \in L:K$上代数的な元、 $f:x$の最小多項式とするとき、次は同値。

  $(1)$
  $x$は $K$上分離的。

  $(2)$
  $f$は分離多項式。

  $(3)$
  $x$は $f$の単根。

  $(4)$
  $K[X]/(f)$は $K$上\rm{etale} $(\Leftrightarrow K上分離的)$
\end{prop}

\begin{proof}
  $x$が $K$上代数的なので命題 $(\mathrm{\ref{prop:6.7}})$から
  $K(x) = K[X]/(f)$となる。

  $x$が $K$上分離的なとき定義から $K(x)/K$が分離的なので
  $K[X]/(f)$が $K$上分離的である。
  そして命題 $(\mathrm{\ref{prop:9.2}})$の $(5) \Leftrightarrow (4)$より $f$の任意の根は単根より $x$は $f$の単根であり、
  $f$は分離多項式である。
\end{proof}

\begin{corl} \label{corl:9.6}
  $x \in L$が ${}^\exists g \in K[X]$の単根ならば $x$は $K$上分離的。
\end{corl}

\begin{proof}
  $x$の最小多項式を $f$としたとき
  最小性から $f \mid g$より $f = gh$となる $h \in K[X]$が存在する。
  このとき $h$が $x$を根として持っているとすると $f$の最小性に矛盾するから
  $h(x) \neq 0$である。
  したがって $f = gh$は $x$を単根としてもつので命題 $(\mathrm{\ref{prop:9.5}})$から $x$は $K$上分離的。
\end{proof}

\begin{corl}
  $x \in L$が $K$上分離的ならば $L/K$の任意の中間体 $M$でも分離的。
\end{corl}

\begin{proof}
  $x$の $M$上の最小多項式を $f_M$とし、
  $K$上の最小多項式を $f_K$とする。
  このとき $K[X] \subset M[X]$から $M[X]$上で $f_M \mid f_K$となる。
  $x$は $K$上分離的なので $f_K$の単根であるから
  系 $(\mathrm{\ref{corl:9.6}})$で $g = f_K \in M[X]$と見れば
  $x$は $M$上分離的である。
\end{proof}

\begin{prop} \label{prop:7.3.6}
  拡大$L/K$について以下は同値。

  $(1)$
  $L$は $K$上代数的かつ分離的。

  $(2)$
  $L$の任意の元 $x$は $K$上代数的かつ分離的。

  $(3)$
  $L$は $K$上代数的かつ分離的な元のある部分集合 $S (\subset L)$によって $K$上生成される。
  $(L = K(S)となる)$
\end{prop}

\begin{proof}
  $(1) \Rightarrow (2)$
  $L/K$が代数的なので $L$の任意の元は $K$上代数的。
  分離的であることから、 $L/K$の任意の有限次部分拡大が分離的である。
  ${}^\forall x \in L$は代数的元なので
  命題 $(\mathrm{\ref{prop:6.7}})$より $K(x)/K$は有限次部分拡大。
  したがって $K(x)/K$が分離的だから定義より $x$は分離的。

  $(2) \Rightarrow (3)$
  仮定より $L$の任意の元は $K$上代数的かつ分離的なので
  $S = L$ととれて、 $K(L) = L$より成立する。

  $(3) \Rightarrow (1)$
  任意の $x \in L$は $S$のある有限部分集合 $S'$によって
  $x \in K(S')$となり、 $K(S')$は有限次拡大より $x$は $K$上代数的。
  よって $L$は $K$上代数的。
  $M$を任意の $K$の有限次部分拡大とする。
  このとき有限次拡大なので系 $(\mathrm{\ref{corl:6.8}})$から
  $M = K(x_1 , \dots , x_m)$となる元 $\{ x_1 , \dots , x_m \}$がある。
  仮定より $x_i \in S$は分離的かつ代数的なのでその最小多項式を $f_i$としたとき、
  命題 $(\mathrm{\ref{prop:9.5}})$から
  $K(x_i) \cong K[X]/(f)$は $K$上\rm{etale}である。
  したがって系 $(\mathrm{\ref{corl:8.9}})$より
  $K(x_1) \otimes \cdots \otimes K(x_m)$も $K$上\rm{etale}である。
  そして、 $M$は\rm{Rem} $(\mathrm{\ref{rem:quotientalgebra}})$より
  $K(x_1) \otimes \cdots \otimes K(x_m)$の商 $K-alg$の部分代数と同型。
  したがって $M$も\rm{etale}より $M$は分離的であるので任意の有限次部分拡大が分離的なので
  $L/K$は代数的かつ分離的。
\end{proof}

\begin{corl} \label{corl:separable}
  代数拡大$L/K$において次は同値。

  $(1)$
  $L/K$は分離的。

  $(2)$
  ${}^\forall x \in L$は $K$上の最小多項式の単根。
\end{corl}

\begin{proof}
  命題 $(\mathrm{\ref{prop:7.3.6}})$の $(1) \Leftrightarrow (2)$から成立する。
\end{proof}

\begin{prop}
  $(1)$
  $L/K$がある集合 $S$によって $L = K(S)$とするとき
  \[
    Sの任意の元が K上分離的 \Rightarrow
    L/K は分離的
  \]

  $(2)$
  $L_1/K , L_2/K \  (\subset {}^\exists L)$に対して
  $L_1 , L_2$の合成体を $L_1 L_2$とすると、
  \[
  L_1 L_2 / Kが分離的 \Leftrightarrow L_1/K , L_2/Kがともに分離的
  \]

  $(3)$
  $L/M/K$のとき
  \[
  L/Kが分離的 \Leftrightarrow L/M , M/Kが分離的
  \]

  $(4)$
  $L/K , K'/K$についてその合成体 $L' := L K'$について
  \[
  L/Kが分離的 \Rightarrow L'/K'が分離的
  \]
\end{prop}

\end{document}
